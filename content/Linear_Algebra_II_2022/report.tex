\documentclass{article}

%%%% Chinese
% fontset = none 可以为后续自定义字体
\usepackage[UTF8, scheme=plain]{ctex}
% % 设置英文字体
% \setmainfont[BoldFont={Ubuntu Bold}, ItalicFont={Ubuntu Italic}]{Ubuntu}
% \setmainfont{Microsoft YaHei}
% \setsansfont{Comic Sans MS}
% \setmonofont{Courier New}
% % 设置中文字体
% \setCJKmainfont[Mapping = fullwidth-stop]{SimSun} %选项将。映射为.
% \setCJKmainfont{SimSun}
% \setCJKmonofont{Source Code Pro}
% \setCJKsansfont{YouYuan}

%%% 习题课讲义排版
\usepackage{xparse}
\newtoks\patchtoks    % helper token register
\def\longpatch#1%     % worker macro
  {\let\myoldmac#1%
   \long\def#1##1{\patchtoks={##1}\myoldmac{\the\patchtoks}}}
\longpatch{\phantom}
\NewDocumentEnvironment{solution}{ +b}{%
  \ifsolution
      \textbf{解答}\quad #1
  \else
      \phantom{\parbox{\textwidth}{#1}}
  \fi
  }{\par}
\newif\ifsolution
% \solutiontrue %添加此句将输出答案,否则输出答案所需的空白

%%%%% MATH
\usepackage{amsmath, amssymb, amsthm}
\usepackage{physics}
\newtheorem{theorem}{Theorem} % 定义定理环境等
\newtheorem{lemma}{Lemma}
\newtheorem{definition}{Definition}
\newtheorem*{remark}{Remark}
\numberwithin{equation}{section} % 公式分节标号

%%%%% Paper size
\usepackage{geometry}
\geometry{a4paper,left=2cm,right=2cm,bottom=2.5cm,top=2.5cm}

%%%%% Text
\usepackage{bm}
\usepackage{ulem} % underlining
\usepackage{enumitem} % customizing lists
\usepackage{titling} % 标题调整
\pretitle{           % 标题左对齐
  \begin{flushleft}
}
\posttitle{
  \end{flushleft}
}
\preauthor{
  \begin{flushleft}
}
\postauthor{
  \end{flushleft}
}
\predate{
  \begin{flushleft}
}
\postdate{
  \end{flushleft}
  \noindent\vrule height 1.0pt width \textwidth
  \vskip .75em plus .25em minus .25em% increase the vertical spacing a bit, make this particular glue stretchier
}
\usepackage{color}
\usepackage{setspace}
\renewcommand{\baselinestretch}{1.3}

%%%%% Ref
\usepackage[
  giveninits=true,       % 作者姓名首单词缩写, 并大写
  date = year,           % 日期只是年份
  % bibstyle = mystyle,  % 参考文献样式(文章最后的文献条目)
]{biblatex}
\renewbibmacro{in:}{} % 去掉文章前面的in
\DeclareFieldFormat[article,incollection,unpublished]{title}{#1} % No quotes for article titles
\DeclareFieldFormat[thesis]{title}{\mkbibemph{#1}} % Theses like book titles
\DeclareFieldFormat{journaltitle}{#1\isdot}
\DeclareFieldFormat{pages}{#1} % 去掉页码前面的pp
% 使用shortjournal代替journaltitle
\DeclareSourcemap{
  \maps[datatype=bibtex]{
    \map[overwrite]{ % Notice the overwrite: replace one field with another
      \step[fieldsource=shortjournal,fieldtarget=journaltitle]
    }
  }
}
\renewcommand*{\bibpagespunct}{\addspace} % This tells biblatex to only put a space right before the pages, no other punctuation.
\renewcommand*{\newunitpunct}{\addcomma\space}
\addbibresource{report.bib}
\usepackage[colorlinks,
            linkcolor=red,
            anchorcolor=red,
            citecolor=blue
            ]{hyperref}

%%%% Graphic
\usepackage{graphics} %color
\usepackage{tikz}
\usetikzlibrary{decorations.pathreplacing,calligraphy} % tikz 花括号

%%%%% Table
\usepackage{tabularx} %table column fix width
\newcolumntype{L}[1]{>{\raggedright\arraybackslash}p{#1}}
\newcolumntype{C}[1]{>{\centering\arraybackslash}p{#1}}
\newcolumntype{R}[1]{>{\raggedleft\arraybackslash}p{#1}}
\usepackage{booktabs} % three lines table
\usepackage{multirow} % multi row; multi column self-contained

%%%%% Figure
\usepackage{graphicx}  % minipage
\usepackage{subfig}    % subfig

%%%%% Code and other
% \usepackage[numbered,framed]{matlab-prettifier} %matlab
% \lstset{
%   style              = Matlab-editor,
%   basicstyle         = \mlttfamily,
%   escapechar         = ",
%   mlshowsectionrules = true,
% }
\usepackage{listings}
\usepackage{algorithm}
\usepackage{algorithmic}
\algsetup{
  indent=3em,
  linenosize=\small,
  linenodelimiter=.
}

%%%%%%% New command
%\newcommand{\char}[0]{\mathrm{char}\,}
\newcommand{\Ker}[0]{\mathrm{Ker}\,}
\newcommand{\Img}[0]{\mathrm{Im}\,}

\title{{\huge \bfseries
线性代数A(II)习题课讲义03
       }}
\author{Caiyou Yuan}
\date{\today}

\begin{document}
\maketitle
% include, new page
\paragraph{8.1}
\begin{itemize}
    \item 域, 以及域$F$上的线性空间
    \item 基和维数
    \begin{itemize}
		\item[a.] 所有非零线性空间均有基
		\item[b.] 线性空间中的任意一组线性无关的向量可以扩充为基
	\end{itemize}
    \item 过渡矩阵
\end{itemize}

\paragraph{例题}
\begin{itemize}
  \item[1.]
  \begin{itemize}
    \item [(a)] 把域$F$看成是$F$上的线性空间,求它的一个基和维数;
    \item [(b)] 把复数域$C$看成是实数域$R$上的线性空间,求它的一个基和维数;
    \item [(c)] 把实数域$R$看成是有理数域$Q$上的线性空间,证明:对于任意大于1的正整数$n$,
  $$1, \sqrt[n]{3}, \sqrt[n]{3^2}, \cdots, \sqrt[n]{3^{n-1}}$$
  是线性无关的.\,(提示: 已知$g(x) = x^n-3$是Q上的不可约多项式)
    \item [(d)] 证明: 实数域$R$作为有理数域$Q$上的线性空间是无穷维的.
\end{itemize}
\end{itemize}
\vspace{2cm}

\begin{itemize}
  \item[2.] 设$V$是域$F$上的n维线性空间,域$F$包含域$E$,\,$F$可看作域$E$
  上的$m$维线性空间
  \begin{itemize}
    \item [(a)] 求证:V可以成为域$E$上的线性空间
    \item [(b)] 证明: 求V作为域E上线性空间的维数
  \end{itemize}
\end{itemize}
\vspace{1.5cm}

\begin{itemize}
  \item[3.] (Complexification of real vector space)
  设$V$是数域R上的$n$维线性空间,设$V_C = \{(u,v), u,v\in V\}$,
  定义$V_C$上的加法
  $$(u_1,v_1) +(u_2, v_2) = (u_1+u_2, v_1+v_2)$$
  以及C上的数乘
  $$(a+bi)(u,v)=(au-bv, av+bu)$$
  \begin{itemize}
    \item [(a)] 求证:$V_C$是一个复线性空间
    \item [(b)] 计算$V_C$的维数
  \end{itemize}
\end{itemize}
\vspace{1cm}

\begin{itemize}
  \item[4.] (对偶空间)设$V$是数域K上的$n$维线性空间,
  考虑复数域$C$上的线性空间$C^V$(从$V$到$R$的函数全体)中具有下述性质的函数组成的子集$W$:
  $$f(\alpha+\beta)=f(\alpha) + f(\beta),\quad \forall \alpha, \beta \in V,$$
  $$f(k\alpha)=kf(\alpha),\quad \forall \alpha \in V, k \in K.$$
  \begin{itemize}
    \item [(a)] 求证:$W$是一个复线性空间
    \item [(b)] 求$W$的一个基和维数;设$f\in W$, 求$f$在这个基下的坐标
  \end{itemize}
\end{itemize}
\vspace{2cm}

\begin{itemize}
  \item[5.] (零化多项式和最小多项式)
  设A是数域K上的一个非零n阶矩阵,说明$K[A]$是$K$上的一个线性空间.
  $K[A]$至多多少维?\footnote{Hamilton-Cayley定理告诉我们,K[A]至多n维.}
\end{itemize}
\vspace{1cm}

\begin{itemize}
  \item[6.] 设递推方程
  $$u_n = au_{n-1} +bu_{n-2}, \quad n \ge 2,$$
  其中$a,b$都是非零复数. 若N上的一个复值数列$u_n$满足上述递推关系,则称为上述递推方程的解.
  一元多项式$f(x)=x^2 - ax -b$称为上述递推方程的特征多项式.
  求证
  \begin{itemize}
    \item [(a)] 上述递推方程的解集$W$是一个复线性空间
    \item [(b)] 设$\alpha$是一个非零复数,则$\alpha^n \in W$ 当且仅当$f(\alpha)=0$
    \item [(c)] 设$\alpha$是一个非零复数,则$n\alpha^n \in W$ 当且仅当$f(\alpha)=0, f'(\alpha)=0$.
    \item [(d)] 若$f(x)$有两不同的根$\alpha_1, \alpha_2$, 则任意$u_n \in W$, 可以表示为
    $$ u_n = C_1 \alpha_1^n +C_2 \alpha_2^n$$
    其中$C_1, C_2$是常数;
    \item [(e)] 若$f(x)$有二重根$\alpha$, 则任意$u_n \in W$, 可以表示为
    $$ u_n = C_1 \alpha^n +C_2 n\alpha^n$$
    其中$C_1, C_2$是常数;
  \end{itemize}
\end{itemize}
\vspace{6cm}


\paragraph{8.2}
\begin{itemize}
    \item 子空间
    \item 子空间的维数定理
    $$\dim V_1 + \dim V_2 = \dim (V_1 + V_2) + \dim (V_1 \cap V_2)$$
    \item 直和
\end{itemize}

\paragraph{例题}
% \begin{itemize}
% \item[1.] 设$V_1, V_2, V_3$都是域F上的有限维线性空间$V$的子空间
% \begin{itemize}
%     \item[(a)] 求证:
%     $$V_3 \cap V_1 + V_3 \cap V_2 \subset V_3 \cap (V_1 + V_2)$$ 
%     \item[(b)] 如果$V_3 \subset V_1 + V_2$,
% 试问$(V_3\cap V_1) + (V_3\cap V_2) = V_3$是否总成立?如果再加上条件$V_1 \subset V_3$呢?
%     \item[(c)] 求证,
%     \begin{equation}
%     \nonumber
%     \begin{aligned}
%     \dim V_1 + \dim V_2 + \dim V_3 \ge
%     &\dim(V_1 + V_2 + V_3)\\
%     &+\dim(V_1 \cap V_2) +\dim(V_1 \cap V_3) +\dim(V_2 \cap V_3)\\
%     &-\dim(V_1 \cap V_2 \cap V_3)
%     \end{aligned}
%     \end{equation}  
%   \end{itemize}
% \end{itemize}
% \vspace{3cm}

% \begin{itemize}
%   \item[2.] 设$V_1, \cdots, V_s$都是域F上线性空间V的子空间,
%   证明$V_1 + V_2 + \cdots + V_s$是直和
%   \begin{itemize} 
%     \item[(a)] 当且仅当,
%     $$V_i \cap \left(\sum_{j\ne i} V_j \right) = 0, \quad i=1,2,\dots,s$$
%     \item[(b)] 当且仅当,$V$中有一向量$\alpha$可以唯一表示为
%     $$\alpha = \sum_{i=1}^s \alpha_i, \quad \alpha_i \in V_i$$
%     \item[(c)] 当且仅当,
%     $$\dim(V_1 + V_2 + \cdots + V_s) = \dim V_1 + \cdots \dim V_s$$ 
%   \end{itemize}
% \end{itemize}
% \vspace{3cm}

% \begin{itemize}
%   \item[3.] 设A是数域K上的一个n阶矩阵,$\lambda_1, \lambda_2, \cdots, \lambda_s$
%   是A的全部不同的特征值,用$V_{\lambda_i}$表示A的属于$\lambda_i$的特征子空间. 证明:
%   A可以对角化的充分必要条件是
%   $$K^n = V_{\lambda_1} \oplus V_{\lambda_2} \oplus \cdots \oplus V_{\lambda_s}$$
% \end{itemize}
% \vspace{1cm}

\begin{itemize}
  \item[1.] 设$V_1, V_2, \cdots, V_s$是域$F$上线性空间V的$s$个真子空间,证明:如果$char F \ne 0$,
  $V_1\cup V_2\cup \cdots \cup V_s \ne V$.
  \vspace{2cm}

  \item[2.] 在数域K上的线性空间$K^{M_n(K)}$中,如果$f$满足,
  对于任意的$A=(\alpha_1, \alpha_2,\cdots, \alpha_n)$,任意的n维列向量$\alpha$, 以及任意$k\in K$, $j\in \{1,2,\cdots, n\}$,
  有 
  $$f(\alpha_1, \cdots, \alpha_{j-1}, \alpha_{j} + \alpha, \alpha_{j+1}, \cdots, \alpha_n)
  = f(\alpha_1, \cdots, \alpha_{j-1}, \alpha_{j}, \alpha_{j+1}, \cdots, \alpha_n) + 
    f(\alpha_1, \cdots, \alpha_{j-1}, \alpha, \alpha_{j+1}, \cdots, \alpha_n)$$
  $$f(\alpha_1, \cdots, \alpha_{j-1}, k\alpha_{j}, \alpha_{j+1}, \cdots, \alpha_n)
  = kf(\alpha_1, \cdots, \alpha_{j-1}, \alpha_{j}, \alpha_{j+1}, \cdots, \alpha_n)
  $$
  那么称$f$是$M_n(K)$上的列线性函数. 同理如果$g(A^T)$是列线性函数,则称$g(A)$是行线性函数。
  记所有的列/行线性函数组成的集合分别记为$V_1$和$V_2$.
  \begin{itemize} 
    \item[(a)] 证明: $V_1, V_2$都是$K^{M_n(K)}$的子空间
    \item[(b)] 分别求$V_1, V_2$的一个基和维数
    \item[(c)] 分别求$V_1 \cap V_2,\,V_1 + V_2$的一个基和维数 
  \end{itemize}
\end{itemize}
\vspace{6cm}

\paragraph{8.3}
\begin{itemize}
    \item 线性空间的同构
    \item 有限维线性空间同构的充要条件
\end{itemize}

\paragraph{例题}
\begin{itemize}
  \item[1.]令
  \begin{equation}
  \nonumber
  H = \left\{
    \begin{bmatrix}
      z_1& z_2\\
     -\bar{z}_2& \bar{z}_1\\
    \end{bmatrix},
    z_1, z_2 \in C
      \right\}
  \end{equation}
  \begin{itemize}
    \item[(a)] H对于矩阵的加法,以及实数和矩阵的乘法构成一个实线性空间
    \item[(b)] 给出H的一个基和维数
    \item[(c)] 证明: $H$与$R^4$同构,并写出$H$到$R^4$的一个同构映射  
  \end{itemize}
\end{itemize}
\vspace{2cm}

\begin{itemize}
  \item[2.]设$A \in M_n(K)$, 令$AM_n(K) = \{AB, B\in M_n(K) \}$.
  \begin{itemize}
    \item[(a)] 证明$AM_n(K)$是数域$K$上线性空间$M_n(K)$的子空间
    \item[(b)] 设A的列向量组$\alpha_1, \cdots, \alpha_n$的一个极大线性无关组为
    $\alpha_{j_1}, \cdots, \alpha_{j_r}$,证明$AM_n(K)$和$M_{r\times n}(K)$同构,并
    写出一个同构映射. 
    \item[(c)] 证明: $\dim [AM_n(K)] = \rank(A)n$.
  \end{itemize}
\end{itemize}
\vspace{4cm}

\paragraph{8.4}
\begin{itemize}
    \item 商空间
    \item $\dim{(V/W)} = \dim V - \dim W$
\end{itemize}

\paragraph{例题}
\begin{itemize}
    \item [1.] 设$U,W$都是域F上线性空间$V$的子空间, 证明$(U+W)/W \cong U/(U\cap W)$
    \vspace{2cm}
    % \item [2.] 设$V$是域F上的一个$n$维线性空间, ($n\ge 3$). $U$是$V$的一个2维子空间,用
    % $\Omega_1$表示V中包含U的所有$n-1$维子空间组成的集合,用$\Omega_2$表示商空间$V/U$的所有
    % $n-3$维子空间组成的集合,令
    % \begin{equation}
    %     \nonumber
    %     \begin{aligned}
    %     \sigma:\quad &\Omega_1 \longrightarrow \Omega_2\\
    %                  & W \longrightarrow W/U.
    %     \end{aligned}
    % \end{equation}
    % 证明: $\sigma$是双射.
    % \vspace{3cm}
\end{itemize}

\paragraph{9.1-4}
\begin{itemize}
    \item 线性映射,线性变换,线性函数
    \item 线性映射的核与像:
    \begin{itemize}
        \item[1.] $\dim{(\Ker A)} + \dim{(\Img A)} = \dim{V}$
        \item[2.] 有限维线性变换, 单等价于满
    \end{itemize}
    \item 线性映射的矩阵表示:
    \begin{itemize}
        \item[1.] $Hom(V,V') \cong M_{s \times n}(F)$
        \item[2.] 相似矩阵 $\Longleftrightarrow$ 线性变换在不同基下的表示 
    \end{itemize}
    \item 线性映射的行列式,秩,迹,特征值,特征向量等
\end{itemize}

\paragraph{例题}
\begin{itemize}
	\item [1.] 设$A \in Hom(V,V)$, 证明对于任意的$k$, 
	$$\rank{A^k} - \rank{A^{k+1}} \ge \rank{A^{k+1}} - \rank{A^{k+2}}$$
	\vspace{3cm}

	\item [2.] 设$f$是$M_n(K)$上的线性函数,且对于任意的$A,B \in M_n(K),
	f(AB)=f(BA)$, 求证$f=c \mathrm{tr}$, 其中$c$是某一常数,
	$\mathrm{tr}$是迹算子.
	\vspace{2cm}

	\item [3.] (Frobenius秩不等式) 设$V,U,W,M$都是域F上的线性空间,
    并且$V,U$都是有限维的,设$A \in Hom(V,U), B \in Hom(U,W), C \in Hom(W,M)$.
    证明,$$\rank(CBA) \ge \rank(CB) + \rank(BA) - \rank(B).$$
    \vspace{2cm}

    \item [4.] (幂零矩阵的矩阵表示)设V是域$F$上的n维线性空间,A是V上的一个线性变换. 如果
    $A^{n-1} \ne 0, A^n = 0$, 那么在V中存在一个基,使得A在此基下的矩阵为
    \begin{equation}
    \nonumber
    \begin{bmatrix}
        0& 1& 0& \cdots &0\\
        0& 0& 1& \cdots &0\\
        \vdots& \vdots& \vdots& \cdots & 1\\
        0& 0& 0& \cdots &0\\
    \end{bmatrix}
    \end{equation}
    \vspace{3cm}

    \item [5.] 设$V$和$V'$分别是域$F$上n维,s维线性空间,A是$V$到$V'$
    的一个线性映射,证明存在$V$的一个基和$V'$的一个基,使得A在这对基下的矩阵为,
    \begin{equation}
    \nonumber
    \begin{bmatrix}
        I_r& 0\\
        0&   0\\
    \end{bmatrix}
    \end{equation}
    其中$r = \rank(A)$.
    \vspace{3cm}

    \item [6.](两两正交的幂等变换的充要条件)
    设$A_i \in M_n(K)$, $i=1,2,\cdots,s$,其中$K$是数域.
    令$A=A_1 + A_2 +\cdots + A_s$.
    \begin{itemize}
        \item[(1) ] 证明: 如果
              $$ \rank(A) = \rank(A_1) + \rank(A_2) \cdots + \rank(A_s),$$
              那么
              $$AM_n(K) = A_1M_n(K) \oplus A_2M_n(K) \oplus \cdots \oplus A_s M_n(K).$$
        \item [(2) ] 证明: $A_1, A_2, \cdots, A_n$是两两正交的幂等矩阵,
              当且仅当$A$是幂等矩阵,并且
              $$ \rank(A) = \rank(A_1) + \rank(A_2) \cdots + \rank(A_s).$$
    \end{itemize}
    \vspace{5cm}

    % \item [7. ] (对角化和数域相关) 设$f(x)=x^n +a_{n-1}x^{n-1} +\cdots +a_1 x +a_0$ 是有理数域Q上的一个不可约
    % 多项式,$n>1$,$\omega$是$f(x)$的一个复根. 把$C$看成是$Q$上的线性空间,令
    % \begin{equation}
    %     \nonumber
    %     \begin{aligned}
    %     \mathbf{B}:\quad &Q[x] \longrightarrow C\\
    %                      &g(x) \longrightarrow g(\omega).
    %     \end{aligned}
    % \end{equation}
    % \begin{itemize}
    %     \item[(1) ] $\mathbf{B}$是不是一线性映射?若是,给出$\Img \mathbf{B}$的一个基和维数;
    %     \item[(2) ] 令$\mathbf{A}(z) = \omega z, \forall z \in \Img \mathbf{B}$, 
    %                 $\mathbf{A}$是不是$\Img \mathbf{B}$上的线性变换?如果是,求$\mathbf{A}$在(1)中基下的矩阵A;
    %     \item[(3) ] $\mathbf{A}$是否可以对角化?
    %     \item[(4) ] 把矩阵$A$看成是复数域上的矩阵,该矩阵可以对角化么?
    % \end{itemize}
    % \vspace{5cm}
\end{itemize}


\end{document}
