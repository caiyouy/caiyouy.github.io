\paragraph{9.5}
\begin{itemize}
    \item 不变子空间
    \item 若$f(x)=f_1(x)f_2(x), (f_1(x),f_2(x))=1$, 则
     $$\Ker f(A) = \Ker f_1(A) \oplus \Ker f_2(A)$$
\end{itemize}

\paragraph{例题}
\begin{itemize}
	\item [1.] 设W是V上线性变换$A$的不变子空间
	\begin{itemize}
		\item[(a)] 若$A$可逆, $A|_W$可逆么?
		\item[(b)] 若$A$可对角化, $A|_W$可对角化么? 
	\end{itemize}
	\vspace{3cm}

    \item [2.] (同时对角化的充分必要条件) 设$A_1, \cdots, A_m$是域$F$上线性空间$V$上的线性变换,每一个都可以对角化,
    证明: 他们可以在某一组基下同时对角化的充分必要条件是他们的乘积可交换,即$A_i A_j = A_j A_i$.
    \vspace{4cm}

    \item [3.] 设$A$是域$F$上有限维线性空间$V$上的一个线性变换,
    \begin{itemize}
        \item [(a) ] 设$f(x),g(x) \in F[x]$,
		$(f(x),g(x))=d(x), \left[f(x),g(x)\right]=m(x)$, 证明,
        $$\Ker d(A) = \Ker f(A) \cap \Ker g(A)$$
        $$\Ker m(A) = \Ker f(A) + \Ker g(A)$$
		\vspace{2cm}

        \item [(b) ] 设$A$的最小多项式为$m(x)$, 而$f(x),g(x) \in F[x]$, 
		$f|m$, $g|f$, 但$f,g$并不相伴, 证明,$\Ker g(A) \subsetneq \Ker f(A)$.
        \vspace{3cm}
		
		\item [(c)] 若$f$为非零多项式, 证明$\Ker f(A)={0}$ 的充分必要条件是$f$和$m$互质.
		\vspace{3cm}

		\item [(d)] 若$f$为非零首一多项式,且$f|m$, 证明$f$是$A|_{\Ker f(A)}$的最小多项式.
		\vspace{3cm}
    \end{itemize}
    \vspace{4cm}
\end{itemize}


\paragraph{9.6-7}
\begin{itemize}
    \item Hamilton-Cayley定理: 对于线性变换$A$的特征多项式$f(x)$,有$f(A)=0$
    \item 线性变换的最小多项式
    \begin{itemize}
        \item [1. ] 如果$V$能分解成一些非平凡不变子空间的直和,
        $$V = W_1 \oplus W_2 \cdots \oplus W_s$$
        则$A$的最小多项式$m(\lambda) = [m_1(\lambda), m_2(\lambda), \cdots, m_s(\lambda)]$,其中
        $m_j(\lambda)$是$W_j$上变换$A|_{W_{j}}$的最小多项式.
        \item [2. ] $A$可以对角化的充分必要条件是,$A$的最小多项式$m(\lambda)$可以在$F[\lambda]$
        中分解成不同的一次因式的乘积.
    \end{itemize}
\end{itemize}

\paragraph{例题}
\begin{itemize}
    \item [1.] 设$A,B$分别是$n,m$阶复矩阵. 证明:矩阵方程$AX-XB=0$只有零解的充分必要条件是$A$和$B$没有公共特征值.
    \vspace{3cm}

	\item [2.] 设$A \in M_{n\times n}(K)$可以对角化, $M_{n\times n}(K)$
	上的线性变换$\mathcal{A}: X \rightarrow AX-XA$, 那么$\mathcal{A}$可以对角化么?
	\vspace{3cm}

    % \item [2.] Hamilton-Cayley定理对于实数域上的线性变换成立么?
    % \vspace{3cm}
    \item [3.] 设$A$是域F上的$n$维线性空间$V$上的线性变换,A有$s$个不同的特征值,可以对角化,
    属于特征值$\lambda_i$的特征子空间的维数是$n_i$.
    \begin{itemize}
        \item [(a) ] 证明,
        $$\dim C(A) = \sum_{i} n_i^2$$
        \item [(b) ] 说明若$s<n$,
        $$F[A] \subsetneq C(A)$$
        若$s=n$,
        $$F[A] = C(A)$$
    \end{itemize}
\end{itemize}


