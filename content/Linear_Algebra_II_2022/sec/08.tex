约定这里所提及的线性空间均是域$F$上的有限维线性空间.
\section{线性空间的张量积}
\begin{definition}
    对于线性空间$U,V$, 若存在线性空间$T$和双线性映射$\otimes: U\times V \rightarrow T$,
    满足: 
    \begin{enumerate}
        \item[$*$] 对于任一线性空间$W$, 以及双线性映射$f: U\times V \rightarrow W$, 
    都存在唯一的线性映射$F: T\rightarrow W$, 使得$f = F \circ \otimes$,
    \end{enumerate}
    则称二元组$(T, \otimes)$是$U,V$的张量积.
\end{definition}

\begin{remark}
    说明$U, V$的张量积在同构的意义下是唯一的, 即
    若$(T_1, \otimes_1)$, $(T_2, \otimes_2)$均为$U,V$的张量积,
    则$T_1$和$T_2$同构.
    所以我们用$U\otimes V$表示$U,V$的张量积, 
    这里省略了双线性映射$\otimes$.
\end{remark}
\vspace{6cm}

\begin{remark}
说明$U,V$的张量积是存在的.
取$T$为$U,V$上的双线性函数空间, 即$T = L(U^*,V^*;F)$.
\end{remark}
\vspace{5cm}

\begin{remark}
    证明上述性质$*$ 等价于
    \begin{enumerate}
        \item [$*_1$] $T = span( \Img \otimes)$
        \item [$*_2$] 对于任一线性空间$W$, 以及双线性映射$f: U\times V \rightarrow W$, 
    都存在线性映射$F: T\rightarrow W$, 使得$f = F \circ \otimes$,
    \end{enumerate}
\end{remark}
\vspace{6cm}

\begin{remark}
    $Im \otimes$是$T$的子空间么?若是给出证明,若不是举出反例.
\end{remark}
\vspace{4cm}

\begin{remark}
    证明$U \otimes V$和$V \otimes U$同构.
\end{remark}
\vspace{5cm}

\section{多个线性空间的张量积}
\begin{definition}
    对于线性空间$U_i(i=1,\cdots, N)$, 
    若存在线性空间$T$和$N$重线性映射$\otimes: U_1\times \cdots \times U_N
     \rightarrow T$,
    满足: 
    \begin{enumerate}
        \item[$*$] 对于任一线性空间$W$, 以及$N$重线性映射$f: U_1\times \cdots \times U_N
        \rightarrow W$, 都存在唯一的线性映射$F: T\rightarrow W$, 使得$f = F \circ \otimes$,
    \end{enumerate}
    则称二元组$(T, \otimes)$是$U_i(i=1,\cdots, N)$的张量积.
\end{definition}
\begin{remark}
    这里的存在唯一性,和上一节$N=2$的情形说明方式相同.
\end{remark}

\begin{remark}
    证明$U_1 \otimes U_2 \otimes U_3$ 和$(U_1 \otimes U_2) \otimes U_3$同构.
    其中$U_1, U_2, U_3$是三个线性空间.
\end{remark}
\vspace{6cm}


\section{线性变换的张量积}
\begin{definition}
    对于线性空间$U,V$, 设$A,B$分别是$U,V$上的线性变换, 则存在唯一的
    $U\otimes V$上的线性变换, 记为$A\otimes B$, 使得
    \begin{equation}
    (A\otimes B)(\alpha \otimes \beta) = A\alpha \otimes B\beta, 
    \quad \forall \alpha \in V, \beta \in U.
    \label{equ:tensor_AB}
    \end{equation}
    称$A\otimes B$为$A,B$的张量积.
\end{definition}
\begin{remark}
    证明满足\eqref{equ:tensor_AB}的线性映射是存在唯一的.
\end{remark}
\vspace{3cm}

\begin{remark}
    若$A_1, A_2$是$U$上的线性变换, $B_1, B_2$是$V$上的线性变换,
    说明$(A_1 \otimes B_1)(A_2 \otimes B_2) = A_1A_2 \otimes B_1 B_2$
\end{remark}
\vspace{3cm}

\begin{remark}
    说明$\Img (A\otimes B) = \Img A \otimes \Img B$,
        $\rank (A\otimes B) = (\rank A)(\rank B)$
    %    $\Ker (A\otimes B) = \Ker$
\end{remark}
\vspace{5cm}

\begin{remark}
    证明$(A\otimes B)_{\Phi_1 \otimes \Phi_2} = A_{\Phi_1} \otimes B_{\Phi_2}$,
    其中$\Phi_1 = \{\alpha_1, \cdots \alpha_n\}$,
    $\Phi_2 = \{\beta_1, \cdots \beta_m\}$分别是
    $U,V$的一组基. $A_{\Phi_1}, B_{\Phi_2}$分别是$A,B$在基$\Psi_1,\Psi_2$
    下的矩阵, $A_{\Phi_1} \otimes B_{\Phi_2}$中的$\otimes$表示矩阵的Kronecker
    乘积.
\end{remark}
\vspace{6cm}

\begin{remark}
    设$A,B$分别是域$F$上的$n,m$级矩阵,证明$A\otimes B$和$B\otimes A$相似.
\end{remark}