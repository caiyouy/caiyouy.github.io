\paragraph{7.3}
\begin{itemize}
  \item 最大公因式: $K[x]$中任意两个多项式都有最大公因式,且可以表示为$f,g$的和式
  \item 互素
\end{itemize}
\paragraph{例题}
\begin{itemize}
%   \item[1.] 设$f(x),g(x) \in K[x]$, $a,b,c,d\in K$而且$ad-bc\ne 0$.证明,
%   $(af(x)+bg(x), cf(x)+dg(x)) = (f(x),g(x))$.
%   \item[2.] 设$A\in M_n(K)$, $f(x),g(x)\in K[x]$, 证明, 如果$d(x) = (f(x), g(x))$, 那么
%   $d(A)x=0$的解空间是$f(A)x=0$解空间和$g(A)x=0$解空间的交.
    \item[1.] $f(x)=x^3+(1+t)x^2+2x+u,\, g(x)=x^3+tx+u$的最大公因式为二次,求$t,u$的值
    \vspace{2cm}
    
    \item[2.] $f(x) = x^3-2x^2+2x-1, g(x)=x^4-x^3+2x^2-x+1$, 求$(f,g)$,并表示为$f,g$的和式
    \vspace{2cm}

    \item[3.] 在$K[x]$中, 如果$(f(x),g(x))=1$, 并且$\deg f >0, \deg g>0$, 那么在
    $K[x]$中存在唯一的多项式$u(x), v(x)$, $\deg u< \deg g, \deg v<\deg f$,
    s.t.
    $$u(x)f(x) +v(x)g(x) = 1.$$
    \vspace{2cm}

    \item[4.] 在$K[x]$中, 如果$(f(x),g(x))=d(x)$, 那么在
    $K[x]$中存在唯一的多项式$u(x)$和$v(x)$, $\deg u < \deg g - \deg d,\,
    \deg v<\deg f - \deg d$,
    s.t.
    $$u(x)f(x) +v(x)g(x) = d(x).$$
    \vspace{1cm}

%   \item[5.] 在$K[x]$中的两个非零多项式$f(x)$和$g(x)$不互素的充分必要条件是,存在两个非零多项式
%   $u(x),v(x)$, s.t. $u(x)f(x) = v(x)g(x)$且$deg\,u <deg\,g$, $deg\,v<deg\,f$.
    \item[5.] 在$K[x]$中,$f_1(x),f_2(x),\cdots,f_s(x)$两两互素,证明对于任意的
    $r_1(x),r_2(x),\cdots, r_s(x)\in K[x]$, 同余方程组
    \begin{equation}
    \nonumber
    \left\{
      \begin{aligned}
      g(x) &\equiv r_1(x) \quad \mathrm{mod}\, f_1(x)\\
      g(x) &\equiv r_2(x) \quad \mathrm{mod}\, f_2(x)\\
      \vdots&\\
      g(x) &\equiv r_s(x) \quad \mathrm{mod}\, f_s(x)\\
      \end{aligned}
    \right.
    \end{equation}
    在$K[x]$中有解,且若$c(x),d(x)$均为解,则$c(x) \equiv d(x),\,\,\mathrm{mod}\,f_1 f_2\cdots f_s$.
    \vspace{2cm}
\end{itemize}

\paragraph{7.4}
\begin{itemize}
  \item 不可约多项式
  \item 唯一因式分解定理;这里唯一性的意义是?
\end{itemize}
\paragraph{例题}
\begin{itemize}
%   \item[1.] 在$K[x]$中,设$(f,g_i)=1,i=1,2$. 证明$(fg_1,g_2)=(g_1,g_2)$.
%   \item[2.] 在$K[x]$中, 证明对于任意的正整数$m$,有$(f^m(x), g^m(x)) = (f(x),g(x))^m$.
%   \item[3.] 分别在复数域、实数域和有理数域上分解 $x^4+1$为不可约多项式的乘积 

\item[1.] 证明,数域$K$上的一个次数大于零的多项式$f$与$K[x]$中某一
不可约多项式的正整数次幂相伴的充分必要条件是,对于任意$g(x)\in K[x]$,
必有$(f(x), g(x)) = 1,$或者存在一个整数数$m$, 使得$f(x)|g^m(x)$.
\vspace{2cm}
\end{itemize}

\paragraph{7.5等}
\begin{itemize}
    \item 重因式
    \item 复数域上的不可约多项式只有一次的
    \item 实数域上的不可约多项式都是一次的,或判别式小于零的二次多项式
    \item 有理数域上的不可约多项式可以是任意次数的
\end{itemize}

\paragraph{例题}
\begin{itemize}
\item[1.] 证明, $K[x]$中一个$n$次($n\ge 1$)多项式$f(x)$, 能被它的导数整除的充分必要条件是
$f(x)$与一个一次因式的$n$次幂相伴.
\vspace{2cm}

\item[2.] 设$K$是一个数域,$R$是$K$的一个交换扩环,设$a\in R,$ 其中$J_a \ne {0},$
$$ J_a = \{f(x) \in K[x] | f(a) = 0\}$$ 
证明:
\begin{itemize}
    \item [(1)] $J_a$中存在唯一的首项系数为$1$的多项式$m(x)$, 使得$J_a$的元素都是$m(x)$
    的倍式.
    \vspace{2cm}

    \item [(2)] 如果$R$是无零因子环, 则$m(x)$在$K[x]$中不可约.
    \vspace{2cm}

    \item [(3)] 取$K=C, R=C[A]$, 其中
    \begin{equation}
        \nonumber
        A=\begin{bmatrix}
            1& -1\\
            1& 1\\
        \end{bmatrix}
    \end{equation}
    求$J_A$中的$m(x)$, 并判断$C[A]$是无零因子环么?
    \vspace{2cm}
\end{itemize}

\item[3.] 在复数域上求下述循环矩阵的全部特征值以及行列式
\begin{equation}
    \nonumber
    A=\begin{bmatrix}
        a_1& a_2& a_3& \cdots& a_n\\
        a_n& a_1& a_2& \cdots& a_{n-1}\\
        \vdots& \vdots& \vdots& &\vdots\\
        a_3& a_4& a_5& \cdots& a_{2}\\
        a_2& a_3& a_4& \cdots& a_{1}\\
    \end{bmatrix}
\end{equation}
\vspace{2cm}
\end{itemize}


% \paragraph{8.1}
% \begin{itemize}
%     \item 域
%     \item 域$F$上的线性空间的定义
%     \item 基
%     \item 维数
%     \item 过渡矩阵
% \end{itemize}

% \paragraph{例题}
% \begin{itemize}
%   \item[1.]
%   \begin{itemize}
%     \item [(a)] 把域$F$看成是$F$上的线性空间,求它的一个基和维数;
%     \item [(b)] 把复数域$C$看成是实数域$R$上的线性空间,求它的一个基和维数;
%   \end{itemize}
% \end{itemize}
% \vspace{1cm}

% \begin{itemize}
%   \item[2.]
%   \begin{itemize}
%     \item [(a)] 把实数域$R$看成是有理数域$Q$上的线性空间,证明:对于任意大于1的正整数,
%   $$1, \sqrt[n]{3}, \sqrt[n]{3^2}, \cdots, \sqrt[n]{3^{n-1}}$$
%   是线性无关的.\,(提示: 已知$g(x) = x^n-3$是Q上的不可约多项式)
%     \item [(b)] 证明: 实数域$R$作为有理数域$Q$上的线性空间是无穷维的.
%   \end{itemize}
% \end{itemize}
% \vspace{1.5cm}

% \begin{itemize}
%   \item[3.] 设$V$是域$F$上的n维线性空间,域$F$包含域$E$,\,$F$可看作域$E$
%   上的$m$维线性空间
%   \begin{itemize}
%     \item [(a)] 求证:V可以成为域$E$上的线性空间
%     \item [(b)] 证明: 求V作为域E上线性空间的维数
%   \end{itemize}
% \end{itemize}
% \vspace{1.5cm}

% \begin{itemize}
%   \item[4.] (Complexification of real vector space)
%   设$V$是数域R上的$n$维线性空间,设$V_C = \{(u,v), u,v\in V\}$,
%   定义$V_C$上的加法
%   $$(u_1,v_1) +(u_2, v_2) = (u_1+u_2, v_1+v_2)$$
%   以及C上的数乘
%   $$(a+bi)(u,v)=(au-bv, av+bu)$$
%   \begin{itemize}
%     \item [(a)] 求证:$V_C$是一个复线性空间
%     \item [(b)] 计算$V_C$的维数
%   \end{itemize}
% \end{itemize}
% \vspace{1cm}

% \begin{itemize}
%   \item[5.] (对偶空间)设$V$是数域K上的$n$维线性空间,
%   考虑复数域$C$上的线性空间$C^V$(从$V$到$R$的函数全体)中具有下述性质的函数组成的子集$W$:
%   $$f(\alpha+\beta)=f(\alpha) + f(\beta),\quad \forall \alpha, \beta \in V,$$
%   $$f(k\alpha)=kf(\alpha),\quad \forall \alpha \in V, k \in K.$$
%   \begin{itemize}
%     \item [(a)] 求证:$W$是一个复线性空间
%     \item [(b)] 求$W$的一个基和维数;设$f\in W$, 求$f$在这个基下的坐标
%   \end{itemize}
% \end{itemize}
% \vspace{1cm}

% \begin{itemize}
%   \item[6.] (零化多项式和最小多项式)
%   设A是数域K上的一个非零n阶矩阵,说明$K[A]$是$K$上的一个线性空间.
%   $K[A]$至多多少维?\footnote{Hamilton-Cayley定理告诉我们,K[A]至多n维.}
% \end{itemize}
% \vspace{1cm}

% \begin{itemize}
%   \item[7.] 设递推方程
%   $$u_n = au_{n-1} +bu_{n-2}, \quad n \ge 2,$$
%   其中$a,b$都是非零复数. 若N上的一个复值数列$u_n$满足上述递推关系,则称为上述递推方程的解.
%   一元多项式$f(x)=x^2 - ax -b$称为上述递推方程的特征多项式.
%   求证
%   \begin{itemize}
%     \item [(a)] 上述递推方程的解集$W$是一个复线性空间
%     \item [(b)] 设$\alpha$是一个非零复数,则$\alpha^n \in W$ 当且仅当$f(\alpha)=0$
%     \item [(c)] 设$\alpha$是一个非零复数,则$n\alpha^n \in W$ 当且仅当$f(\alpha)=0, f'(\alpha)=0$.
%     \item [(d)] 若$f(x)$有两不同的根$\alpha_1, \alpha_2$, 则任意$u_n \in W$, 可以表示为
%     $$ u_n = C_1 \alpha_1^n +C_2 \alpha_2^n$$
%     其中$C_1, C_2$是常数;
%     \item [(e)] 若$f(x)$有二重根$\alpha$, 则任意$u_n \in W$, 可以表示为
%     $$ u_n = C_1 \alpha^n +C_2 n\alpha^n$$
%     其中$C_1, C_2$是常数;
%   \end{itemize}
% \end{itemize}
% \vspace{6cm}


% \paragraph{8.2}
% \begin{itemize}
%     \item 子空间
%     \item 子空间的维数定理
%     $$\dim V_1 + \dim V_2 = \dim (V_1 + V_2) + \dim (V_1 \cap V_2)$$
%     \item 直和
% \end{itemize}

% \paragraph{例题}
% \begin{itemize}
% \item[1.] 设$V_1, V_2, V_3$都是域F上的有限维线性空间$V$的子空间
% \begin{itemize}
%     \item[(a)] 求证:
%     $$V_3 \cap V_1 + V_3 \cap V_2 \subset V_3 \cap (V_1 + V_2)$$ 
%     \item[(b)] 如果$V_3 \subset V_1 + V_2$,
% 试问$(V_3\cap V_1) + (V_3\cap V_2) = V_3$是否总成立?如果再加上条件$V_1 \subset V_3$呢?
%     \item[(c)] 求证,
%     \begin{equation}
%     \nonumber
%     \begin{aligned}
%     \dim V_1 + \dim V_2 + \dim V_3 \ge
%     &\dim(V_1 + V_2 + V_3)\\
%     &+\dim(V_1 \cap V_2) +\dim(V_1 \cap V_3) +\dim(V_2 \cap V_3)\\
%     &-\dim(V_1 \cap V_2 \cap V_3)
%     \end{aligned}
%     \end{equation}  
%   \end{itemize}
% \end{itemize}
% \vspace{3cm}

% \begin{itemize}
%   \item[2.] 设$V_1, \cdots, V_s$都是域F上线性空间V的子空间,
%   证明$V_1 + V_2 + \cdots + V_s$是直和
%   \begin{itemize} 
%     \item[(a)] 当且仅当,
%     $$V_i \cap \left(\sum_{j\ne i} V_j \right) = 0, \quad i=1,2,\dots,s$$
%     \item[(b)] 当且仅当,$V$中有一向量$\alpha$可以唯一表示为
%     $$\alpha = \sum_{i=1}^s \alpha_i, \quad \alpha_i \in V_i$$
%     \item[(c)] 当且仅当,
%     $$\dim(V_1 + V_2 + \cdots + V_s) = \dim V_1 + \cdots \dim V_s$$ 
%   \end{itemize}
% \end{itemize}
% \vspace{3cm}

% \begin{itemize}
%   \item[3.] 设A是数域K上的一个n阶矩阵,$\lambda_1, \lambda_2, \cdots, \lambda_s$
%   是A的全部不同的特征值,用$V_{\lambda_i}$表示A的属于$\lambda_i$的特征子空间. 证明:
%   A可以对角化的充分必要条件是
%   $$K^n = V_{\lambda_1} \oplus V_{\lambda_2} \oplus \cdots \oplus V_{\lambda_s}$$
% \end{itemize}
% \vspace{1cm}

% \begin{itemize}
%   \item[4.] 在数域K上的线性空间$K^{M_n(K)}$中,如果$f$满足,
%   对于任意的$A=(\alpha_1, \alpha_2,\cdots, \alpha_n)$,任意的n维列向量$\alpha$, 以及任意$k\in K$, $j\in \{1,2,\cdots, n\}$,
%   有 
%   $$f(\alpha_1, \cdots, \alpha_{j-1}, \alpha_{j} + \alpha, \alpha_{j+1}, \cdots, \alpha_n)
%   = f(\alpha_1, \cdots, \alpha_{j-1}, \alpha_{j}, \alpha_{j+1}, \cdots, \alpha_n) + 
%     f(\alpha_1, \cdots, \alpha_{j-1}, \alpha, \alpha_{j+1}, \cdots, \alpha_n)$$
%   $$f(\alpha_1, \cdots, \alpha_{j-1}, k\alpha_{j}, \alpha_{j+1}, \cdots, \alpha_n)
%   = kf(\alpha_1, \cdots, \alpha_{j-1}, \alpha_{j}, \alpha_{j+1}, \cdots, \alpha_n)
%   $$
%   那么称$f$是$M_n(K)$上的列线性函数. 同理如果$g(A^T)$是列线性函数,则称$g(A)$是行线性函数。
%   记所有的列/行线性函数组成的集合分别记为$V_1$和$V_2$.
%   \begin{itemize} 
%     \item[(a)] 证明: $V_1, V_2$都是$K^{M_n(K)}$的子空间
%     \item[(b)] 分别求$V_1, V_2$的一个基和维数
%     \item[(c)] 分别求$V_1 \cap V_2,\,V_1 + V_2$的一个基和维数 
%   \end{itemize}
% \end{itemize}
% \vspace{6cm}

% \paragraph{8.3}
% \begin{itemize}
%     \item 线性空间的同构
%     \item 有限维线性空间同构的充要条件
% \end{itemize}

% \paragraph{例题}
% \begin{itemize}
%   \item[1.]令
%   \begin{equation}
%   \nonumber
%   H = \left\{
%     \begin{bmatrix}
%       z_1& z_2\\
%      -z_2& z_1\\
%     \end{bmatrix},
%     z_1, z_2 \in C
%       \right\}
%   \end{equation}
%   \begin{itemize}
%     \item[(a)] H对于矩阵的加法,以及实数和矩阵的乘法构成一个实线性空间
%     \item[(b)] 给出H的一个基和维数
%     \item[(c)] 证明: $H$与$R^4$同构,并写出$H$到$R^4$的一个同构映射  
%   \end{itemize}
% \end{itemize}
% \vspace{1cm}

% \begin{itemize}
%   \item[2.]设$A \in M_n(K)$, 令$AM_n(K) = \{AB, B\in M_n(K) \}$.
%   \begin{itemize}
%     \item[(a)] 证明$AM_n(K)$是数域$K$上线性空间$M_n(K)$的子空间
%     \item[(b)] 设A的列向量组$\alpha_1, \cdots, \alpha_n$的一个极大线性无关组为
%     $\alpha_{j_1}, \cdots, \alpha_{j_r}$,证明$AM_n(K)$和$M_{r\times n}(K)$同构,并
%     写出一个同构映射. 
%     \item[(c)] 证明: $\dim [AM_n(K)] = \rank(A)n$.
%   \end{itemize}
% \end{itemize}
% \vspace{1cm}