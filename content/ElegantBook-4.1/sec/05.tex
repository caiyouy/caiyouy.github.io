\chapter{矩阵相抵和相似}

\section{矩阵相抵}
\begin{itemize}
\item 如果矩阵$A$可以通过初等行/列变换为矩阵$B$, 则称$A,B$相抵
\item 矩阵相抵是$M_{m\times n}(K)$上的一个等价关系
\item $M_{m\times n}(K)$中的两矩阵相抵当且秩相等
\end{itemize}

\subsection*{例题}
\begin{itemize}
	\item[1.] 设$A,B,C$分别是数域$K$上$s\times n, p\times m, s\times m$矩阵,
	证明矩阵方程$AX-YB=C$有解的充分必要条件是
	\begin{equation}
	\nonumber
	\mathrm{rank}
	\begin{bmatrix}
		A&0\\
		0&B
	\end{bmatrix}
	=
	\mathrm{rank}
	\begin{bmatrix}
		A&C\\
		0&B
	\end{bmatrix}
	\end{equation}
	\begin{solution}
		必要性将$C=AX-YB$代入即可; 充分性,
		假设$\mathrm{rank}(A)=a, \mathrm{rank}(B)=b$,
		则存在可逆矩阵$P_a, Q_a, P_b, Q_b$, 使得
		\begin{equation*}
			P_a A Q_a = \begin{bmatrix}
				I_a&0\\
				0&0\\
			\end{bmatrix},\quad
			P_b B Q_b = \begin{bmatrix}
				I_b&0\\
				0&0\\
			\end{bmatrix}
		\end{equation*}
		所以
		\begin{equation*}
		\begin{bmatrix}
			P_a&\\
			&P_b\\
		\end{bmatrix}
		\begin{bmatrix}
			A&C\\
			0&B\\
		\end{bmatrix}
		\begin{bmatrix}
			Q_a&\\
			&Q_b\\
		\end{bmatrix}
		=
		\begin{bmatrix}
			I_a&0& C_{11}& C_{12}\\
			0&0& C_{21}& C_{22}\\
			0&0& I_b& 0\\
			0&0& 0& 0\\
		\end{bmatrix}
		\end{equation*}
		其中
		\begin{equation*}
		P_a C Q_b = \begin{bmatrix}
			C_{11}&C_{12}\\
			C_{21}&C_{22}\\
		\end{bmatrix}
		\end{equation*}
		由已知条件, $C_{22}=0$. 所以
		\begin{equation*}
		\begin{bmatrix}
			I& &-C_{11}& \\
			&I &-C_{21}&\\
			& &I&\\
			& &&I\\
		\end{bmatrix}
		\begin{bmatrix}
			P_a&\\
			&P_b\\
		\end{bmatrix}
		\begin{bmatrix}
			A&C\\
			0&B\\
		\end{bmatrix}
		\begin{bmatrix}
			Q_a&\\
			&Q_b\\
		\end{bmatrix}
		\begin{bmatrix}
			I& & &-C_{12}\\
			&I & &\\
			& &I&\\
			& &&I\\
		\end{bmatrix}
		=
		\begin{bmatrix}
			I_a&0&0&0\\
			0&0&0&0\\
			0&0& I_b& 0\\
			0&0& 0& 0\\
		\end{bmatrix}
		\end{equation*}
		注意右上角的分块计算结果为0, 整理即可获得原矩阵方程的解.
	\end{solution}
	\vspace{5cm}
\end{itemize}