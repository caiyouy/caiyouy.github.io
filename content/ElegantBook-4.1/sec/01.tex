\chapter{线性方程组}
\begin{itemize}
    \item 矩阵的初等行变换
    \item 阶梯形矩阵,简化阶梯形矩阵
    \item Guass-Jordan算法,无解/有唯一解/有无数解
    \item 数域, $Q,R,C$
\end{itemize}

\subsection*{例题}
\begin{itemize}
    \item[1.] 证明任意一个矩阵都可以经过一系列初等行变换化为(简化)阶梯形矩阵.
	\begin{solution}
	对行数$s$进行数学归纳:$s=1$显然成立,假设$s=k$时成立, 而当$s=k+1$时,
	\begin{enumerate}
		\item 若$a_{11} \ne 0$, 使用初等行变换,将$a_{21}, \cdots, a_{k+1,1}$
	化为0,对右下的子矩阵使用归纳假设即可;
		\item 若$a_{11} = 0, \,a_{i1} \ne 0$, 交换1行和$i$行,即化为情况1;
		\item 若第一列均为零,则对右下$k$行的子矩阵使用归纳假设即可;
	\end{enumerate}
	所以对于阶梯形,结论成立。而简化阶梯形只需在阶梯形基础上再做若干次初等行变换即可。
	\end{solution}
    \vspace{2cm}

    \item[2.] 解如下线性方程组
    \begin{equation}
    \nonumber
        \left\{
        \begin{aligned}
            &x_1 + x_2 + \cdots + x_n = 1\\
            &x_2 + x_3 + \cdots + x_{n+1} = 2\\
            &x_3 + x_4 + \cdots + x_{n+2} = 3\\
            &\vdots\\
            &x_{n+1}+x_{n+2}+ \cdots + x_{2n} = n+1\\
        \end{aligned}
        \right.
    \end{equation} 
	\begin{solution}
	相邻两式相减;注意到$x_{n+2}, x_{n+3},\cdots,x_{2n}$是自由变量即可
	\end{solution}
    \vspace{2cm}

    \item[3.] 解如下线性方程组
    \begin{equation}
    \nonumber
        \left\{
        \begin{aligned}
            &x_1 +2x_2 +\cdots +(n-1)x_{n-1} +nx_{n} =b_1\\
            &nx_1 +x_2 +\cdots +(n-2)x_{n-1} +(n-1)x_{n} =b_2\\
            &(n-1)x_1 +nx_2 +\cdots +(n-3)x_{n-1} +(n-2)x_{n} =b_3\\
            &\vdots\\
            &2x_1 +3x_2 +\cdots +nx_{n-1} +x_{n} =b_n\\
        \end{aligned}
        \right.
    \end{equation} 
    其中$b_1,\cdots b_n$为给定常数
	\begin{solution}
	各式相加;相邻两项相减
	\end{solution}
    \vspace{2cm}
\end{itemize}

\chapter{行列式}
\section{n元排列}
\begin{itemize}
    \item 逆序数
    \item 奇/偶排列
\end{itemize}

\subsection*{例题}
\begin{itemize}
    \item[1.] 设$1,2,\cdots,n$的$n$元排列
    $a_1a_2\cdots a_k b_1b_2\cdots b_{n-k}$
    有$$\tau(a_1a_2\cdots a_k) = \tau(b_1b_2\cdots b_{n-k}) = 0$$
    那么$\tau(a_1a_2\cdots a_k b_1b_2\cdots b_{n-k})$是多少?
    \,$n,k,a_1,\cdots,a_k, b_1, \cdots, b_{n-k}$
    均已知,\,$\tau(a_1a_2\cdots a_k)$表示排列$a_1a_2\cdots a_k$的逆序数.
	\begin{solution}
		$\sum_{i=1}^k a_i - \frac{k(k+1)}{2}$, 答案不唯一
	\end{solution}
    \vspace{2cm}

    \item[2.] 说明n(n>1)元排列中,奇偶排列各占一半
	\begin{solution}
		任意两位置的对换,建立了奇排列和偶排列间的一一对应
	\end{solution}
    \vspace{1.5cm}

    \item[3.]
    \begin{itemize}
        \item[(1)] 若$\tau(a_1a_2\cdots a_n) = r$, 那么
        $\tau(a_na_{n-1}\cdots a_1)$ 是多少?
		\begin{solution}
			$\frac{n(n-1)}{2} - r$
		\end{solution}
        \vspace{1.5cm}
        \item[(2)] 计算所有n元排列的逆序数之和
		\begin{solution}
			在上一问的基础上,将所有排列分为$\frac{n!}{2}$组;
		\end{solution}
        \vspace{1.5cm}
    \end{itemize} 
\end{itemize}

\section{n阶行列式的定义}
\begin{equation}
\nonumber
\begin{aligned}
\det{A} &= \sum_{j_1 j_2\cdots j_n} (-1)^{\tau(j_1 j_2\cdots j_n)}a_{1j_1}a_{2j_2}\cdots a_{nj_n}\\
        &= \sum_{i_1 i_2\cdots i_n} (-1)^{\tau(i_1 i_2\cdots i_n)}a_{i_11}a_{i_22}\cdots a_{i_nn}
\end{aligned}
\end{equation}

\begin{remark}
	如何从上一行定义,导出下一行定义?见课本\cite{book1}24-25页
\end{remark}

\subsection*{例题}
\begin{itemize}
    \item[1.] 下列行列式是$x$的几次多项式,求出$x^4$项和$x^3$项的系数
    \begin{equation}
    \nonumber
    \left|
        \begin{array}{rrrr}
        5x &x &1 &x\\
        1  &x &1 &-x\\
        3  &2 &x &1\\
        3  &1 &1 &x\\
        \end{array}
    \right|
    \end{equation} 
	\begin{solution}
		按照定义讨论,或者按某行/列展开;四次多项式;$5x^4, -2x^3$
	\end{solution}
    \vspace{2cm}

    \item[2.] 计算下列行列式
    \begin{equation}
    \nonumber
    \left|
        \begin{array}{rrrrr}
        a_1 &a_2 &a_3 &\cdots &a_n\\
        b_2 &1   &0   &\cdots &0\\
        b_3 &0   &1   &\cdots &0\\
        \vdots &\vdots   &\vdots   & &\vdots\\
        b_n &0   &0   &\cdots &1\\
        \end{array}
    \right|
    \end{equation}
	\begin{solution}
		箭形行列式,利用定义;$a_1 - a_2b_2 - a_3b_3 - \cdots a_n b_n$
	\end{solution}
    \vspace{2cm}
\end{itemize}

\section{行列式的性质}
\subsection*{例题}
\begin{itemize}
    \item 计算下列行列式
    \begin{equation}
    \nonumber
    \left|
        \begin{array}{ccccc}
        x_1-a_1 &x_2     &x_3     &\cdots & x_n\\
        x_1     &x_2-a_2 &x_3     &\cdots & x_n\\
        x_1     &x_2     &x_3-a_3 &\cdots & x_n\\
        \vdots  &\vdots  &\vdots  &       & \vdots\\
        x_1     &x_2     &x_3     &\cdots & x_n-a_n\\
        \end{array}
    \right|
    \end{equation} 
    其中$a_i \ne 0, i = 1,\cdots,n.$
    \begin{solution}
		$(-1)^na_1 a_2 \cdots a_n \left(1-\frac{x_1}{a_1} - \cdots - \frac{x_n}{a_n}\right)$
		\begin{enumerate}
			\item 第一行的$-1$倍加到$2$到$n$行,化为箭形矩阵行列式;
			\item 非对角元素补上减去$0$,每列拆分为两列, $2^n$个行列式中经有$n+1$个非零;
			\item 和方法二类似,但只对第一列拆分, 找到$n$阶结果和$n-1$阶结果的关系;
		\end{enumerate}
	\end{solution}
    \vspace{2cm}
\end{itemize}

\section{行列式按一行(列)展开}
\begin{itemize}
    \item 余子式,代数余子式
    \item Vandermonde行列式(行列式的递推)
\end{itemize}

\vspace{1em}
\begin{remark}
	按行展开如何从行列式定义推导?
\end{remark}

\subsection*{例题}
\begin{itemize}
    \item[1.] 计算下列行列式
    \begin{itemize}
        \item[(1)]
        \begin{equation}
        \nonumber
        \left|
            \begin{array}{cccccccc}
            2a      &a^2     &0       &0      &\cdots & 0 &0 &0\\
            1       &2a      &a^2     &0      &\cdots & 0 &0 &0\\
            0       &1       &2a      &a^2    &\cdots & 0 &0 &0\\
            \vdots  &\vdots  &\vdots  &\vdots &       & \vdots & \vdots &\vdots\\
            0       &0       &0       &0      &\cdots & 1 &2a &a^2\\
            0       &0       &0       &0      &\cdots & 0 &1 &2a\\
            \end{array}
        \right|
        \end{equation} 
		\begin{solution}
		递推;$S_n = 2aS_{n-1} - a^2S_{n-2}, S_n = (n+1)a^n$
		\end{solution}
        \vspace{2cm}

        \item[(2)]
        \begin{equation}
        \nonumber
        \left|
            \begin{array}{cccccccc}
            a+b      &ab     &0       &0      &\cdots & 0 &0 &0\\
            1       &a+b     &ab     &0      &\cdots & 0 &0 &0\\
            0       &1       &a+b      &ab    &\cdots & 0 &0 &0\\
            \vdots  &\vdots  &\vdots  &\vdots &       & \vdots & \vdots &\vdots\\
            0       &0       &0       &0      &\cdots & 1 &a+b &ab\\
            0       &0       &0       &0      &\cdots & 0 &1 &a+b\\
            \end{array}
        \right|
        \end{equation} 
		\begin{solution}
		递推;$S_n = (a+b)S_{n-1}-abS_{n-2}$, 若$a = b$, 则化为上一小问;若$a\ne b$, $S_n = \frac{a^{n+1}-b^{n+1}}{a-b}$
		\end{solution}
        \vspace{2cm}

        \item[(3)]
        \begin{equation}
        \nonumber
        \left|
            \begin{array}{cccccccc}
            a       &b     &0       &0      &\cdots & 0 &0 &0\\
            c       &a     &b     &0      &\cdots & 0 &0 &0\\
            0       &c     &a      &b    &\cdots & 0 &0 &0\\
            \vdots  &\vdots  &\vdots  &\vdots &       & \vdots & \vdots &\vdots\\
            0       &0       &0       &0      &\cdots & c &a &b\\
            0       &0       &0       &0      &\cdots & 0 &c &a\\
            \end{array}
        \right|
        \end{equation} 
		\begin{solution}
		递推
		\begin{equation}
		\nonumber
		S_n = \left\{
			\begin{array}{ll}
			(n+1)\left(\frac{a}{2}\right)^n& a^2 = 4bc\\
			\frac{x_1^{n+1} - x_2^{n+1}}{x_1 - x_2}& a^2 \ne 4bc\\
			\end{array}
		\right.
		\end{equation}
		其中$x_1, x_2$是$x^2 - ax + bc = 0$的两个根。
		\end{solution}
        \vspace{2cm}
    \end{itemize}

    \item[2.] 计算如下行列式
    \begin{equation}
    \nonumber
    \left|
        \begin{array}{ccccc}
        1       &1       &\cdots &1 & 1\\
        x_1     &x_2     &\cdots &x_{n-1} & x_n\\
        x_1^2   &x_2^2   &\cdots &x_{n-1}^2 & x_n^2\\
        \vdots  &\vdots  &       &\vdots & \vdots\\
        x_1^{n-2} &x_2^{n-2}     &\cdots &x_{n-1}^{n-2} & x_n^{n-2}\\
        x_1^{n} &x_2^{n}     &\cdots &x_{n-1}^{n} & x_n^{n}\\
        \end{array}
    \right|
    \end{equation} 
	\begin{solution}
	考虑如下的$n+1$阶Vandermonde行列式
	\begin{equation}
    \nonumber
    \left|
        \begin{array}{cccccc}
        1       &1       &\cdots &1 & 1& 1\\
        x_1     &x_2     &\cdots &x_{n-1} & x_n & x\\
        x_1^2   &x_2^2   &\cdots &x_{n-1}^2 & x_n^2 & x^2\\
        \vdots  &\vdots  &       &\vdots & \vdots & \vdots\\
        x_1^{n-2} &x_2^{n-2}     &\cdots &x_{n-1}^{n-2} & x_n^{n-2} & x^{n-2}\\
        x_1^{n-1} &x_2^{n-1}     &\cdots &x_{n-1}^{n-1} & x_n^{n-1} & x^{n-1}\\
        x_1^{n} &x_2^{n}     &\cdots &x_{n-1}^{n} & x_n^{n} & x^n\\
        \end{array}
    \right|
    \end{equation}
	所需计算的行列式,即是该Vandermonde行列式(按最后一列展开)的$x^{n-1}$项系数的相反数。结果为
	$(x_1 + x_2 + \cdots + x_n)\Pi_{i<j}(x_j-x_i)$.
	\end{solution}
	\vspace{2cm}
\end{itemize}