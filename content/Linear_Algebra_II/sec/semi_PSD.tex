\paragraph{证明}
$n$阶实对称矩阵$A$半正定的充分必要条件是$A$的所有主子式非负

\paragraph{引理}
$n$阶实对称矩阵$A$半正定的充分必要条件是
$$A \simeq \begin{bmatrix}
	d_1& &\\
	   &\ddots&\\
	   &&d_n\\
\end{bmatrix},
\quad
d_i \ge 0.$$
其中$\simeq$表示矩阵合同.
\begin{proof}
	充分性,
	若
	$$A = P^T\begin{bmatrix}
	d_1& &\\
	   &\ddots&\\
	   &&d_n\\
	\end{bmatrix}P,
	\quad
	d_i \ge 0,$$
	故对于任意的$\mathbf{x}$, $\mathbf{x}^T A \mathbf{x} \ge 0$.
	必要性, $A$实对称,所以可以正交对角化, 
	由于半正定性, 所以特征值非负(否则取$\mathbf{x}$为负特征值对应的特征向量, $\mathbf{x}^T A\mathbf{x}< 0$), 
	故$A$合同于对角元非负的对角矩阵.
\end{proof}

\begin{remark}
	由引理易知,半正定矩阵行列式非负.
\end{remark}

\paragraph{必要性证明}
对于$A$的某一$k$阶子矩阵$A_1$, 由$A$的$i_1,i_2,\cdots,i_k$行和列组成.
取$\mathbf{x}=(x_1,x_2,\cdots,x_n)^T$, 对于$j = 1,\cdots, n,$
\begin{equation}
\nonumber
x_j = \begin{cases}
	y_s,& \text{if } j = i_s,\\
	0, & \text{otherwise.}
\end{cases}
\end{equation}
其中$\mathbf{y} = (y_1, y_2, \cdots, y_k)^T$为任意的$k$维向量.
由$A$的半正定型,易得子矩阵$A_1$的半正定性, 即
\begin{equation}
\nonumber
0 \le \mathbf{x}^T A \mathbf{x} = \mathbf{y}^T A_1 \mathbf{y}.
\end{equation}
由$A_1$的半正定性,所以其行列式非负.

\paragraph{充分性证明一}
注意到
\begin{equation}
\nonumber
|\lambda I + A| = \lambda^n + a_{n-1}\lambda^{n-1}+
\cdots + a_1 \lambda + a_0
\end{equation}
其中$a_i$恰为$A$的所有$(n-i)$阶主子式之和.
由于$a_i \ge 0$, 上述多项式的根小于等于0,
故$A$的特征值大于等于0,故$A$半正定.

\paragraph{充分性证明二}
仿照课本中关于实对称矩阵正定的充要条件是所有顺序主子式大于零的证明(P208-209),
对维数$n$归纳. $n=1$,显然成立,假设$n=k$成立,当$n=k+1$时,
根据$a_{11}$的符号,分为以下两种情形,
\begin{enumerate}
	\item [1.] 若$a_{11} = 0$, 取$A$行列为$1,j$的主子式, 其中$2\le j \le k+1$,
	由其非负, 可得$a_{1j} = a_{j1} = 0$,
	\begin{equation}
	\nonumber
	A = \begin{bmatrix}
		0& \mathbf{0}^T\\
		\mathbf{0}& A_{22}\\
	\end{bmatrix}
	\end{equation}
	对于子块$A_{22}$使用归纳假设, 可知$A_{22}$半正定, 所以$A$半正定.
	\item [2.] 若$a_{11} > 0$,
	\begin{equation}
	\nonumber
	A=\begin{bmatrix}
		a_{11}& \alpha^T\\
		\alpha & A_{22}\\
	\end{bmatrix}
	\simeq 
	\begin{bmatrix}
		1& \mathbf{0}\\
		-\frac{1}{a_{11}}\alpha& I_{k}\\
	\end{bmatrix}
	\begin{bmatrix}
		a_{11}& \alpha^T\\
		\alpha & A_{22}\\
	\end{bmatrix}
	\begin{bmatrix}
		1& -\frac{1}{a_{11}}\alpha^T\\
		\mathbf{0}& I_{k}\\
	\end{bmatrix}
	=
	\begin{bmatrix}
		a_{11}& \mathbf{0}^T\\
		\mathbf{0} & A_{22} - \frac{1}{a_{11}}\alpha \alpha^T \\
	\end{bmatrix}
	\end{equation}
	记$B = A_{22} - \frac{1}{a_{11}}\alpha \alpha^T$,
	注意到$B$
	的$i_1,i_2,\cdots, i_k$主子式$\hat{B}$和
	$A$的$1,i_1+1,i_2+1,\cdots, i_k+1$主子式$\hat{A}$
	的关系,相差一个因子$a_{11}$.具体来说,
	\begin{equation}
	\nonumber
	\hat{A}=
	\left|
	\begin{array}{cc}
		a_{11}& \beta^T\\
		\beta& \hat{A}_{22}\\
	\end{array}
	\right|
	=
	\left|
	\begin{array}{cc}
		a_{11}& \mathbf{0}^T\\
		\mathbf{0} & \hat{A}_{22}-\frac{1}{a_{11}}\beta \beta^T \\
	\end{array}
	\right| = a_{11}\hat{B}
	\end{equation}
	故由$\hat{A}\ge 0$, 可得$\hat{B}\ge 0$.
	即$B$的任意主子式非负, 所以可以对
	$B$使用归纳假设, 进而说明$A$半正定.
\end{enumerate}
