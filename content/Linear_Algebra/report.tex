\documentclass{article}
% \documentclass{book} // final aim

%%%% Chinese
% fontset = none 可以为后续自定义字体
\usepackage[UTF8, fontset=windows, scheme=plain]{ctex}
% % 设置英文字体
% \setmainfont[BoldFont={Ubuntu Bold}, ItalicFont={Ubuntu Italic}]{Ubuntu}
% \setmainfont{Microsoft YaHei}
% \setsansfont{Comic Sans MS}
% \setmonofont{Courier New}
% % 设置中文字体
% etCJKmainfont[Mapping = fullwidth-stop]{SimSun} %选项将。映射为.
% \setCJKmainfont{SimSun}
% \setCJKmonofont{Source Code Pro}
% \setCJKsansfont{YouYuan}

%%% 习题课讲义排版
\usepackage{xparse}
\newtoks\patchtoks    % helper token register
\def\longpatch#1%     % worker macro
  {\let\myoldmac#1%
   \long\def#1##1{\patchtoks={##1}\myoldmac{\the\patchtoks}}}
\longpatch{\phantom}
\NewDocumentEnvironment{solution}{ +b}{%
  \ifsolution
      \textbf{解答}\quad #1
  \else
      \phantom{\parbox{\textwidth}{#1}}
  \fi
  }{\par}
\newif\ifsolution
% \solutiontrue %添加此句将输出答案,否则输出答案所需的空白

%%%%% MATH
\usepackage{amsmath, amssymb, amsthm}
% \usepackage{physics}
% \newtheorem{theorem}{Theorem} % 定义定理环境等
% \newtheorem{lemma}{Lemma}
% \newtheorem{definition}{Definition}
% \newtheorem*{remark}{Remark}
% \numberwithin{equation}{section} % 公式分节标号

%%%%% Paper size
\usepackage{geometry}
\geometry{a4paper,left=2cm,right=2cm,bottom=2.5cm,top=2.5cm}

%%%%% Text
\usepackage{bm}
\usepackage{ulem} % underlining
\usepackage{enumitem} % customizing lists
\usepackage{titling} % 标题调整
\pretitle{           % 标题左对齐
  \begin{flushleft}
}
\posttitle{
  \end{flushleft}
}
\preauthor{
  \begin{flushleft}
}
\postauthor{
  \end{flushleft}
}
\predate{
  \begin{flushleft}
}
\postdate{
  \end{flushleft}
  \noindent\vrule height 1.0pt width \textwidth
  \vskip .75em plus .25em minus .25em% increase the vertical spacing a bit, make this particular glue stretchier
}
\usepackage{color}
\usepackage{setspace}
\renewcommand{\baselinestretch}{1.3}

%%%%% Ref
\usepackage[
  giveninits=true,       % 作者姓名首单词缩写, 并大写
  date = year,           % 日期只是年份
  % bibstyle = mystyle,  % 参考文献样式(文章最后的文献条目)
]{biblatex}
\renewbibmacro{in:}{} % 去掉文章前面的in
\DeclareFieldFormat[article,incollection,unpublished]{title}{#1} % No quotes for article titles
\DeclareFieldFormat[thesis]{title}{\mkbibemph{#1}} % Theses like book titles
\DeclareFieldFormat{journaltitle}{#1\isdot}
\DeclareFieldFormat{pages}{#1} % 去掉页码前面的pp
% 使用shortjournal代替journaltitle
\DeclareSourcemap{
  \maps[datatype=bibtex]{
    \map[overwrite]{ % Notice the overwrite: replace one field with another
      \step[fieldsource=shortjournal,fieldtarget=journaltitle]
    }
  }
}
\renewcommand*{\bibpagespunct}{\addspace} % This tells biblatex to only put a space right before the pages, no other punctuation.
\renewcommand*{\newunitpunct}{\addcomma\space}
\addbibresource{report.bib}
\usepackage[colorlinks,
            linkcolor=red,
            anchorcolor=red,
            citecolor=blue
            ]{hyperref}

%%%% Graphic
\usepackage{graphics} %color
\usepackage{tikz}
\usetikzlibrary{decorations.pathreplacing,calligraphy} % tikz 花括号

%%%%% Table
\usepackage{tabularx} %table column fix width
\newcolumntype{L}[1]{>{\raggedright\arraybackslash}p{#1}}
\newcolumntype{C}[1]{>{\centering\arraybackslash}p{#1}}
\newcolumntype{R}[1]{>{\raggedleft\arraybackslash}p{#1}}
\usepackage{booktabs} % three lines table
\usepackage{multirow} % multi row; multi column self-contained

%%%%% Figure
\usepackage{graphicx}  % minipage
\usepackage{subfig}    % subfig

%%%%% Code and other
% \usepackage[numbered,framed]{matlab-prettifier} %matlab
% \lstset{
%   style              = Matlab-editor,
%   basicstyle         = \mlttfamily,
%   escapechar         = ",
%   mlshowsectionrules = true,
% }
\usepackage{listings}
\usepackage{algorithm}
\usepackage{algorithmic}
\algsetup{
  indent=3em,
  linenosize=\small,
  linenodelimiter=.
}

%%%%%%% New command
%\newcommand{\char}[0]{\mathrm{char}\,}
\newcommand{\rank}[0]{\mathrm{rank}\,}
\newcommand{\Ker}[0]{\mathrm{Ker}\,}
\newcommand{\Img}[0]{\mathrm{Im}\,}

\title{{\huge \bfseries
线性代数A 答疑
       }}
\author{Caiyou Yuan}
\date{\today}

\begin{document}
\maketitle
% \paragraph{通向Jordan标准型的两条路}
\begin{itemize}
    \item[(1)] $\lambda$矩阵
        \begin{itemize}
            \item[] 不变因子,行列式因子,初等因子
        \end{itemize} 
    \item[(2)] 不变子空间直和分解: 
        \begin{itemize}
            \item[] 最小多项式,特征多项式
        \end{itemize}
\end{itemize}

\paragraph{两条路的一点关联}
\begin{itemize}
    \item Jordan标准型中Jordan块的最小多项式即是不变因子.
    \item n级矩阵的最小多项式等于其最后一个不变因子$d_n(\lambda)$,
          特征多项式等于最后一个行列式因子$D_n(\lambda)$.
    % \item 数域$K$上的两矩阵相似当且仅当把它们看成复矩阵后相似.
\end{itemize}

\paragraph{关于有理标准形的推广}
若最小多项式无法在$F[x]$上分解为一次因子的幂次乘积?
\paragraph{不变子空间直和分解}
$C_j = A|W_j$是$n_j$维线性空间$W_j$上的线性变换,其最小多项式为$p_j^{l_j}(\lambda)$,
其中$p_j(\lambda)$是不可约多项式.
\begin{itemize}
    \item [(1)] 若存在$\alpha \in W_j$和正整数$t$,使得
    $\alpha, C_j\alpha, \cdots, C_j^{t-1} \alpha$线性无关,
    且$C_j^t \alpha$可以由$\alpha, C_j\alpha, \cdots, C_j^{t-1} \alpha$
    线性表出, 则称$\alpha, C_j\alpha, \cdots, C_j^{t-1} \alpha$
    张成了由$\alpha$生成的$C_j$-循环子空间.

    $W_j$能分解为$\frac{1}{r} \dim \widehat{W}_j$个$C_j$-循环子空间的直和,
    其中$r=\deg p(\lambda)$, $\widehat{W}_j = \ker p(C_j)$.
    \vspace{1cm}
    \item [(2)] 证明$W_j$中存在一个基,使得$C_j$在此基下的矩阵为
    有理块组成的块对角矩阵, 每个有理块级数是$r$的倍数,且不超过$rl_j$,
    有理块的总数为$\frac{1}{r}\left(n_j - \rank\, p_j(C_j)\right)$.
    $rt$级有理块的个数$N(rt)$为,
    $$N(rt) = \frac{1}{r} \left(
        \rank\,p^{t-1}(C_j) + \rank\,p^{t+1}(C_j) 
        - 2\rank\,p^{t}(C_j).
    \right)$$
    \vspace{1cm}
\end{itemize}

\paragraph{$\lambda$-矩阵}
\begin{itemize}
    \item[(1)] 
     $A(\lambda)$的每个次数大于零的不变因子在$F[x]$上做不可约分解,所有这些不可约多项式的方幂
     (相同的按照出现的次数计算)称为$A(\lambda)$的初等因子.
    \item[(2)] 由有理块组成的块对角矩阵的初等因子由它全部有理块的初等因子组成,
    于是矩阵的有理标准型由它的初等因子唯一确定(除去有理块的排列次序).  
    \item[(3)] 设域$E$包含域$F$, 域$F$上的两矩阵相似当且仅当把他们当成$E$上的矩阵相似. 
\end{itemize}

\paragraph{例题}
$A$是域$F$上线性空间$V$上的线性变换, 且$A$的最小多项式在$F[\lambda]$中的不可约分解为
$$m(\lambda) = (\lambda-\lambda_1)^{l_1}(\lambda-\lambda_2)^{l_2}\cdots(\lambda-\lambda_s)^{l_s}$$
\begin{itemize}
\item[(1)] 令$V_i = \{\alpha \in V |\,\exists r > 0,
(\lambda_i I - A)^r \alpha = 0\}, i = 1,2,\cdots,s.$
说明$V_i$是A不变的,且$V = V_1 \oplus V_2 \oplus \cdots \oplus V_s$ 
\begin{solution}
说明$V_i = W_i$
\end{solution}
% \item[(2)] 若当$i \ne j$时,$\lambda_i \ne \lambda_j$,
% 证明$A$的Jordan标准型恰好由$s$个Jordan块组成当且仅当$\dim V_{\lambda_j} = 1, j = 1,\cdots,s.$
\item[(2)] 
\begin{itemize}
    \item[a.] 证明: A的Jordan标准型中主对角元为$\lambda_i$的Jordan块的个数为
    $N_j = \dim V - \rank (A - \lambda_j I)$.
    其中$J_t(\lambda_j)$的个数为$N_j(t) = \rank (A-\lambda_j I)^{t+1} +
    \rank (A-\lambda_j I)^{t-1} - 2\rank (A-\lambda_j I)^t$.

    \begin{solution}
        前面已说明$N_j = \dim \Ker (A|W_j - \lambda_j I)$以及
        $N_j(t) = \rank (A|W_j-\lambda_j I)^{t+1} +
        \rank (A|W_j-\lambda_j I)^{t-1} - 2\rank (A|W_j-\lambda_j I)^t$,
        说明对于$i=1,2,\cdots, l_j+1$,
        $$\Ker (A-\lambda_j I)^i = \Ker(A|W_j - \lambda_j I)^i$$即可.
    \end{solution}

    \item[b.] 证明:A可对角化,当且仅当$$ \rank (A - \lambda_iI)^2 = 
    \rank (A - \lambda_iI),\ i=1,\cdots,s. $$ 
    \begin{solution}
    \begin{itemize}
        \item [(a)]
        $\rank(\lambda_iI - A) = \rank(\lambda_i I - A)^2 =\cdots =\rank(\lambda_i - A)^{l_i}$
        即特征空间$V_{\lambda_i}$等同于$W_i$,故 
        $V = V_{\lambda_1} \oplus \cdots \oplus V_{\lambda_s}.$
        \item [(b)] 说明$N_j = N_j(1)$, 即$\dim V - \rank(A-\lambda_j I) = 
        \rank(A-\lambda_j I)^0 + \rank(A-\lambda_j I)^2 - 2\rank(A-\lambda_j I)$即可.
    \end{itemize}
    \end{solution}
\end{itemize}

\item[(3)]如果$\forall i,\ \lambda_i = 1,$ 证明$A$和$A^k\,(k\ge 1)$相似
\begin{solution}
\end{solution}
\vspace{3cm}

\item[(4)] 如果$l_1 + l_2 +\cdots l_s = \dim V$, 说明A的Jordan标准型中各个Jordan块主对角元互不相同. 
\vspace{3cm}
\end{itemize}

\paragraph{Jordan标准型的应用}
\begin{itemize}
    \item [1. ] 计算矩阵多项式
    \item [2. ] 计算矩阵幂级数,特别是指数函数
    $$e^{A} = \sum_{i=0}^{\infty} \frac{A^i}{i!}$$
    及其在求解ODE中的应用.
    \begin{itemize}
    %     \item [(1)] 设$f(t) = \sum_{k=0}^{\infty} a_k t^k$, 收敛半径为$r$,
    %     设矩阵$A$的Jordan标准型为
    %     \begin{equation}
    %     \nonumber
    %     J =
    %     \begin{pmatrix}
    %         J_{n_1}(\lambda_1)&&&\\
    %         &J_{n_2}(\lambda_2)&&\\
    %         &&\ddots&\\
    %         &&&J_{n_s}(\lambda_s)\\
    %     \end{pmatrix}
    %     \end{equation}
    %     即$A = PJP^{-1}$, 若$\forall i$, $|\lambda_i| < r$,
    %     即$\rho(A) < r$, 则矩阵幂级数$\sum_{k=0}^{\infty} a_k A^k$收敛,
    %     $$f(A) = P
    %     \begin{pmatrix}
    %         f(J_{n_1}(\lambda_1))&&&\\
    %         &f(J_{n_2}(\lambda_2))&&\\
    %         &&\ddots&\\
    %         &&&f(J_{n_s}(\lambda_s))\\
    %     \end{pmatrix}
    %     $$
    %     其中$f(J_{n_i}(\lambda_i))$该如何表达?
    %     \vspace{3cm}

        \item [(a)]齐次一阶线性常系数常微分方程组
        \begin{equation}
        \nonumber
        \left\{
        \begin{aligned}
        &\frac{\mathrm{d} x}{\mathrm{d} t} = Ax,\\
        &x(0) = x_0.
        \end{aligned}
        \right.
        \end{equation}
        的唯一解为$x(t)=e^{At}x_0$. 例如求解
        \begin{equation}
            \nonumber
            \frac{\mathrm{d} x}{\mathrm{d} t} = 
            \begin{pmatrix}
                2& 1& 4\\
                0& 2& 0\\
                0& 3& 1\\
            \end{pmatrix}
            x
        \end{equation}
        的通解.
        \vspace{3cm}
        \item [(b)]齐次高阶线性常系数常微分方程
        \begin{equation}
        \nonumber
        \left\{
        \begin{aligned}
        &x^{(n)} + a_{n-1} x^{(n-1)} + \cdots + a_1 x^{(1)} + a_0 x = 0\\
        &x(0) = x_0\\
        &x^{(1)}(0) = x_0^{(1)}\\
        &\vdots\\
        &x^{(n-1)}(0) = x_0^{(n-1)}\\
        \end{aligned}
        \right.
        \end{equation}
        其中$x^{(i)}(t) = \frac{\mathrm{d}^i x}{\mathrm{d} t^i}(t).$ 可以转化为方程组情形.
        例如求解$x^{(3)} - 3x^{(2)}-6x^{(1)} + 8x = 0$的通解.
        \vspace{3cm}
    \end{itemize}
    \item [2. ] 计算矩阵平方根
    \begin{itemize}
        \item[(1)] 设$a$是域F中的非零元, 求$J_r(a)^2$的标准型.
        \begin{solution}
            由于
            $J_r(a)^2 = (aI + J_r(0))^2
                      = a^2 I + 2aJ_r(0) + J_r(0)^2$,
            $\rank J_r(a)^2 = r-1$, 所以$\rank J_r(a)^2 \sim J_r(a^2)$.
        \end{solution}
        \vspace{3cm}
        \item[(2)] 任意的可逆复矩阵都有平方根.
        \vspace{3cm}
        \item[(3)]
        \begin{equation}
        \nonumber
            A = 
            \begin{pmatrix}
               -2& 1& 0\\
               -4& 2& 0\\
               -2& 1& 1\\
            \end{pmatrix}
        \end{equation} 
        是否有平方根,若有给出一个.
        \vspace{3cm}
        \item[(4)] 证明:不可逆的复矩阵有平方根,当且仅当其标准型中主对角元为0的
        Jordan块或是$J_1(0)$,或是$J_r(0), J_{r}(0)$成对出现,或是$J_r(0), J_{r+1}(0)$成对出现. 
        \vspace{3cm}
    \end{itemize}
\end{itemize}
% \section{具有度量的线性空间}
\subsection{双线性函数}
\begin{itemize}
    \item 双线性函数, 及其在基下的度量矩阵: 不同基下的度量矩阵是合同的,
    以及合同的矩阵可以看作是同一双线性函数在不同基下的度量矩阵;
    \item 非退化的双线性函数 $\Longleftrightarrow$ 度量矩阵非奇异
    \item 对称/反对称的双线性函数, 及其标准形式
    \begin{itemize}
        \item[1.] 特征不为2的域上,对称矩阵可以合同于对角矩阵
        \item[2.] 特征不为2的域上,反对称矩阵合同于
        \begin{equation}
        \begin{pmatrix}
        0&1\\
        -1&0\\
        &&\ddots\\
        &&&0&1\\
        &&&-1&0\\
        &&&&&0\\
        &&&&&&\ddots\\
        &&&&&&&0\\
        \end{pmatrix}
        \label{equ:matrix_asym}
        \end{equation}
        \item[3.] 特征为2的域$F$上, 因为$\forall a\in F, -a=a$, 
        所以对称矩阵和反对称矩阵相同.
        \begin{itemize}
            \item 若$\exists \alpha \in F^n$,使得$\alpha^T A \alpha \ne 0$,
                  则$A$可以合同于一个对角矩阵;
            \item 若$\forall \alpha \in F^n$,使得$\alpha^T A \alpha = 0$,
                  则$A$可以合同于矩阵\ref{equ:matrix_asym}.
        \end{itemize}
    \end{itemize}
\end{itemize}

\subsection{对称双线性函数和二次型的关系}
设$V$是域$F$上的线性空间, 映射$q: V\rightarrow F$定义为$V$上的二次函数,
如果存在$V$上的对称双线性函数$f$, 使得
$\forall \alpha \in V,q(\alpha) = f(\alpha, \alpha)$.
若$\mathrm{char}\,F \ne 2$,
则二次函数和对称双线性函数一一对应, 因为注意到,
\begin{equation}
\nonumber
f(\alpha, \beta) = 
\frac{1}{2}[q(\alpha+\beta) - q(\alpha) - q(\beta)].
\end{equation}

\subsection{双线性函数空间}
$V$上的所有双线性函数, 关于函数的加法和数乘,构成域$F$上的线性空间, 
记为$T_2(V)$.
易见$T_2(V) \cong M_n(F) \cong Hom(V,V)$.

\paragraph{例题}
取$V^*$中的对偶基为$\{f_i\}_{i=1}^n$,
证明$f_i \otimes f_j$是$T_2(V)$的一组基,
其中$(f \otimes g) \in T_2(V)$, 
定义为$(f \otimes g)(\alpha, \beta) = f(\alpha)g(\beta)$.


\subsection{例题}
设$f$是域F上n维线性空间V上的一个双线性函数,
\begin{itemize}
    \item[1.] 
    \begin{itemize}
        \item [(a)] 映射$L_f: \alpha \rightarrow \alpha_L$是$V$到$V^*$的线性映射;

        \begin{solution}
        $a_L(\beta) = f(\alpha, \beta)$ 
        \end{solution}

        \item [(b)] f是非退化的当且仅当$L_f$是线性空间$V$到$V^*$的一个同构映射.
        
        \begin{solution}
            \begin{equation}
            \nonumber
            \begin{aligned}
                \text{f非退化} &\Longleftrightarrow \text{rad}_L V = 0\\
                              &\Longleftrightarrow \Ker L_f = 0\\
                              &\Longleftrightarrow L_f \text{是双射}\\
            \end{aligned}
            \end{equation}
        \end{solution}

    \end{itemize}
    \item[2.] 若$f$非退化
    \begin{itemize}
        \item [(a)] 任给$V$上的一个双线性函数$g$,存在$V$上唯一的一个线性变换
        $G$,使得$$g(\alpha, \beta) = f(G\alpha, \beta),\, \forall \alpha,\beta \in V$$

        \begin{solution}
            即证$\alpha_L = (G\alpha)_L$,由于$L_f$是$V$到$V^*$的一个同构映射,
            则存在$\hat \alpha \in V, L_f(\hat \alpha) = \alpha_L$,
            定义$G: \alpha \rightarrow \hat \alpha$. 证明$G$线性且唯一.
        \end{solution}

        \item [(b)] 令$\sigma: g\rightarrow G$, 则$\sigma$是$T_2(V)$到$Hom(V,V)$的一个同构映射.
        
        \begin{solution}
            证明$\sigma$是线性映射,且单射.
        \end{solution}

        \item [(c)] 证明V中存在一个基使得$f,g$在此基下的度量矩阵都是对角矩阵的充分必要条件是$G$可以对角化.
        
        \begin{solution}
            充分性:设G的特征空间为$V_i$, 则$V = V_1 \oplus V_2 \oplus \cdots \oplus V_s$,
            $f|V_{i}$是$V_i$上的对称双线性函数, 则存在$V_i$上的一组基使得$f|V_i$
            的度量矩阵是对角的, 易见$g$在这组基下也是对角的. 而对于$i\ne j$, $0 \ne \alpha_i \in V_i, 0 \ne \alpha_j \in V_j$,
            则
            $$g(\alpha_i, \alpha_j) = f(G\alpha_i, \alpha_j) = \lambda_i f(\alpha_i, \alpha_j)$$
            $$g(\alpha_j, \alpha_i) = \lambda_j f(\alpha_j, \alpha_i) $$
            故$f(\alpha_i, \alpha_j) = 0, g(\alpha_i, \alpha_j) = 0$.

            必要性:设在基$\alpha_1, \cdots, \alpha_n$下$f,g$的度量矩阵都是对角的,
            即$\forall j \ne i, g(\alpha_i, \alpha_j) = 0, f(\alpha_i, \alpha_j) = 0$,
            考虑$G \alpha_i$, 由于$f(G\alpha_i, \alpha_j) = 0$, 
            将$G \alpha_i$按照基$\alpha_i$以及$\alpha_1, \alpha_2,
            \cdots \alpha_{i-1}, \alpha_{i+1}, \cdots, \alpha_n$
            的一组正交基$\hat \alpha_1,
            \cdots \hat \alpha_{i-1}, \hat \alpha_{i+1}, \cdots, 
            \hat \alpha_n$展开,
            由于$f(\hat \alpha_j, \hat \alpha_j) > 0$, 故$G\alpha_i \in <\alpha_i>.$
            故在这组基下$G$对角化.
        \end{solution}

        \item [(d)] 设$A,B$都是特征不为2的域F上的n级对称矩阵,且$A$是可逆的,
        证明$A,B$可以同时合同对角化的充分必要条件是$A^{-1}B$可以对角化.
        
        \begin{solution}
            给定空间$V$的一组基,$f,g$的度量矩阵为$A,B$, 故$f$非退化.
            设$\alpha, \beta \in V$在此基下的坐标为$x,y$,
            $f(\alpha, \beta) = x^T A y, g(\alpha, \beta) = x^T By$.
            在这组基下,线性变换$G$的矩阵为$\hat G$.
            $$x^T B y = (Gx)^T A y$$
            故$G^TA = B$, $G = A^{-1}B$, 使用上一题结论即可.
        \end{solution}
    \end{itemize} 
\end{itemize}

\section{实内积空间}
实线性空间 + 正定的对称双线性函数 = 实内积空间,
有限维的实内积空间即欧氏空间.

\subsection{实内积空间的度量}
定义$|\alpha| = \sqrt{(\alpha, \alpha)}$,
有柯西不等式
$$|(\alpha,\beta)| \le |\alpha| |\beta|, \forall \alpha, \beta.$$
定义$d(\alpha, \beta) = |\alpha - \beta|$,
证明$d$是一个距离,即满足对称性,正定性和三角不等式.

\begin{solution}
    使用柯西不等式证明$|\alpha + \beta| \le |\alpha| + |\beta|$,
    令$\alpha = a - b, \beta = b - c$即可。
\end{solution}

\subsection{实内积空间的同构}
同构映射$\sigma$, 不仅作为线性空间的同构映射,还要求保持内积, 即
$(\sigma \alpha, \sigma \beta) = (\alpha, \beta)$.
两个欧氏空间同构的充要条件是维数相同.

\subsection{例题}
在$R[x]_{n+1}$中定义内积
$$(f,g) = \int^{1}_{-1}f(x)g(x) \mathrm{d}x.$$
令
$$P_0(x) = 1, P_k(x) = \frac{1}{2^k k!} \frac{d^k}{dx^k} ((x^2-1)^k), k = 1,2,\cdots,n.$$
证明,这是$R[x]_{n+1}$的一个正交基.

\begin{solution}
    证明$P_k$和$x^i(0\le i < k)$正交即可, 不断使用分部求和.
\end{solution}

\section{正交变换}
实内积空间$V$到自身的满射$A$,满足
$$(A\alpha, A\beta) = (\alpha, \beta), \forall \alpha, \beta$$
则称$A$是$V$上的正交变换.
可以证明
\begin{enumerate}
\item $A$是线性的;

\begin{solution}
    计算$|A(\alpha + \beta) - (A\alpha + A\beta)|^2$
\end{solution}
\item $A$是单的, 从而$A$是可逆的.
\item $A$是$V$到自身的一个同构映射.
\end{enumerate}

\subsection{例题}
\begin{itemize}
    \item[1.] 证明,实内积空间V到自身的满射A是正交变换当且仅当A是保持向量长度不变的线性变换。
    \item[2.] 设$V$是n维欧氏空间,$\eta$是$V$中一单位向量,设P是$V$在$<\eta>$
    上的正交投影,令$A=I-2P$, 称$A$为关于$<\eta>^{\perp}$的镜面反射. 证明这是第二类正交变换. 
    \item[3.] 
    \begin{itemize}
        \item [(a)] 设A是2维欧氏空间V上的第二类正交变化, 证明A是关于某一条直线的镜面反射
        \item [(b)] 设A是2维欧氏空间V上的第一类正交变换, 证明A能表示为两个镜面反射的乘积
        \item [(c)] 证明$n$维欧氏空间V上的任一正交变换都可以表示成至多$n$个镜面反射的乘积
        
        \begin{solution}
            (a,b)考虑正交基$\alpha_1, \alpha_2$和$A \alpha_1, A \alpha_2$,
            先用镜面反射$B_1$将$\alpha_1$映射为$A\alpha_1$, 说明$A\alpha_2 = \pm B_1\alpha_2$.
            
            (c) 同理考虑$\alpha_1, \alpha_2,\cdots, \alpha_n$和
            $A\alpha_1, A\alpha_2,\cdots, A\alpha_n$. 假设$\alpha_1 \ne A\alpha_1$,
            取$B_1 \alpha_1 = A\alpha_1$, 考虑$B_1 \alpha_2 \cdots, B_1\alpha_n$
            和$A \alpha_2 \cdots, A\alpha_n$, 这两组都是$A \alpha_1$的正交补空间的基.
            则$C: B_1\alpha_i \rightarrow A\alpha_i, i=2,\cdots, n$
            是$n-1$维空间的正交变换, 根据归纳假设可以至多表示为$B_2,\cdots, B_n$个镜面反射的乘积.
            将$B_2,\cdots, B_n$延拓为$V$上的镜面反射,即证.
        \end{solution}
    \end{itemize}
    
    % \item[2.] 设A是n维欧氏空间V上的一个线性变换,证明$A$是镜面反射当且仅当A在V的任意一个标准正交基
    % 下的矩阵形如$I-2\alpha \alpha^T$, 其中$\alpha$是一单位向量.

    \item[4.] (正交变换的矩阵表示) A是实内积空间$V$上的正交变换.
    \begin{itemize}
        \item [(a)] 假设A有特征值,证明特征值为1或-1.
        
        \begin{solution}
            正交变换保长度
        \end{solution}
        \item [(b)] 证明A属于不同特征值的特征向量互相正交.
        
        \begin{solution}
            不妨设$\alpha_1 = 1, \alpha_2 = -1$,
            $$(\alpha_1, \alpha_2) = (A\alpha_1, AA^{-1}\alpha_2) = (\alpha_1, A^{-1}\alpha_2) = \lambda_2^{-1} (\alpha_1, \alpha_2)$$
            故$2(\alpha_1, \alpha_2) = 0$.
        \end{solution}
        \item [(c)] 若$W$是A的一个有限维不变子空间,则$W^{\perp}$也是A的不变子空间.
        
        \begin{solution}
            A可逆,$A|W$单,所以可逆,所以W也是$A^{-1}$的不变子空间.
            任意的$\alpha \in W, \beta \in W^{\perp}$,
            $$(A\beta, \alpha) = (A\beta, AA^{-1}\alpha) = (\beta, A^{-1}\alpha) = 0,$$
            所以$A\beta \in W^{\perp}$.
        \end{solution}
        \item [(d)] 若$\dim V = 2$, 若A是第一类的,那么V中存在一组正交基,使得
        A在此基下的矩阵为
        \begin{equation}
        \nonumber
        \begin{pmatrix}
            \cos \theta& -\sin \theta\\
            \sin \theta&  \cos \theta 
        \end{pmatrix},
        0 \le \theta \le \pi,
        \end{equation}
        如果A是第二类的,那么存在一组正交基,A在此基下的矩阵为
        \begin{equation}
        \nonumber
        \begin{pmatrix}
            1& 0\\
            0&-1 
        \end{pmatrix}.
        \end{equation}

        \begin{solution}
            A在标准正交基下的矩阵为正交矩阵, 
            二阶正交矩阵只有两种类型:
            \begin{equation}
            \nonumber
            \begin{pmatrix}
                \cos \theta& -\sin \theta\\
                \sin \theta&  \cos \theta 
            \end{pmatrix},\quad
            \begin{pmatrix}
                \cos \theta&  \sin \theta\\
                \sin \theta& -\cos \theta 
            \end{pmatrix},
            0 \le \theta < 2\pi,
            \end{equation}
        \end{solution}
        \item [(e)] 证明V中存在一个标准正交基,使得此基下A的矩阵为分块对角:
        \begin{equation}
        \nonumber
        \mathrm{diag}
        \left\{
            \lambda_1, \cdots, \lambda_r,
            \begin{pmatrix}
            \cos \theta_1& -\sin \theta_1\\
            \sin \theta_1&  \cos \theta_1
            \end{pmatrix},
            \cdots,
            \begin{pmatrix}
            \cos \theta_m& -\sin \theta_m\\
            \sin \theta_m&  \cos \theta_m
            \end{pmatrix}
        \right\}
        \end{equation}
        其中$\lambda_i = 1$或$-1$, $0 < \theta_i < \pi$.
        
        \begin{solution}
            对维数进行数学归纳
        \end{solution}
    \end{itemize}
\end{itemize}
\paragraph{命题7.6.2(矩阵理论苏育才等)} 设n阶方阵$A$的最小多项式为
$$m(\lambda) = (\lambda-\lambda_1)^{k_1}(\lambda-\lambda_2)^{k_2}\cdots(\lambda-\lambda_s)^{k_s},$$
则
$$g(\lambda)=\sum_{i=1}^s m_i(\lambda)
\left(a_{i0} + a_{i1}(\lambda-\lambda_i) 
+a_{i2}(\lambda-\lambda_i)^2 +\cdots a_{ik_i}(\lambda-\lambda_i)^{k_i-1}\right)$$
满足如下Hermite插值条件
$$g^{(j)}(\lambda_i) = f^{(j)}(\lambda_i),\quad i=1,\cdots,s,\,j=0,\cdots,k_i-1,$$
其中
$$m_i(\lambda) = m(\lambda)/(\lambda-\lambda_i)^{k_i},$$
$$a_{ij} = \frac{1}{j!}\left(\frac{f(\lambda)}{m_i(\lambda)} \right)^{(j)}
\bigg|_{\lambda=\lambda_i}.$$

\paragraph{证明}
只需验证插值条件即可. 注意到, 当$j \le k_i-1$时,
$$g^{(j)}(\lambda)\bigg|_{\lambda=\lambda_i} = 
\left[
m_i(\lambda)
\left(a_{i0} + a_{i1}(\lambda-\lambda_i) 
+a_{i2}(\lambda-\lambda_i)^2 +\cdots a_{ik_i}(\lambda-\lambda_i)^{k_i-1}\right)
\right]^{(j)}\bigg|_{\lambda=\lambda_i}
$$
只需验证上式等于$f^{(j)}(\lambda_i)$即可.

对$j$进行数学归纳.当$j=0$时,易见成立.
当$j=1$时,
\begin{equation}
\nonumber
\begin{aligned}
g^{(1)}(\lambda_i) &=
m_i(\lambda_i)^{(1)}a_{i0}+
m_i(\lambda_i)a_{i1}\\
&=
\left(m_i(\lambda)^{(1)}\frac{f(\lambda)}{m_i(\lambda)}+
m_i(\lambda)
\left(
	\frac{f(\lambda)}{m_i(\lambda)}\right)^{(1)}
\right) \Bigg|_{\lambda = \lambda_i}\\
&=f(\lambda_i)^{(1)}.
\end{aligned}
\end{equation}

假设$j<k$时成立, 其中$k\le k_i-1$.当$j=k$时
\begin{equation}
\nonumber
\begin{aligned}
g^{(j)}(\lambda_i) &=
\sum_{p=0}^{j}C_{j}^p
m_i(\lambda_i)^{(p)}(j-p)!a_{i(j-p)}\\
&=
\left(
\sum_{p=0}^{j}C_{j}^p
m_i(\lambda)^{(p)}
\left(\frac{f(\lambda)}{m_i(\lambda)} \right)^{(j-p)}
\right) \Bigg|_{\lambda = \lambda_i}\\
&=f(\lambda_i)^{(j)}.
\end{aligned}
\end{equation}
证毕.



\end{document}