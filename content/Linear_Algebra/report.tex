\documentclass{article}
% \documentclass{book} // final aim

%%%% Chinese
% fontset = none 可以为后续自定义字体
\usepackage[UTF8, fontset=macnew, scheme=plain]{ctex}
% % 设置英文字体
% \setmainfont[BoldFont={Ubuntu Bold}, ItalicFont={Ubuntu Italic}]{Ubuntu}
% \setmainfont{Microsoft YaHei}
% \setsansfont{Comic Sans MS}
% \setmonofont{Courier New}
% % 设置中文字体
% etCJKmainfont[Mapping = fullwidth-stop]{SimSun} %选项将。映射为.
% \setCJKmainfont{SimSun}
% \setCJKmonofont{Source Code Pro}
% \setCJKsansfont{YouYuan}

%%% 习题课讲义排版
\usepackage{xparse}
\newtoks\patchtoks    % helper token register
\def\longpatch#1%     % worker macro
  {\let\myoldmac#1%
   \long\def#1##1{\patchtoks={##1}\myoldmac{\the\patchtoks}}}
\longpatch{\phantom}
\NewDocumentEnvironment{solution}{ +b}{%
  \ifsolution
      \textbf{解答}\quad #1
  \else
      \phantom{\parbox{\textwidth}{#1}}
  \fi
  }{\par}
\newif\ifsolution
% \solutiontrue %添加此句将输出答案,否则输出答案所需的空白

%%%%% MATH
\usepackage{amsmath, amssymb, amsthm}
% \usepackage{physics}
% \newtheorem{theorem}{Theorem} % 定义定理环境等
% \newtheorem{lemma}{Lemma}
% \newtheorem{definition}{Definition}
% \newtheorem*{remark}{Remark}
% \numberwithin{equation}{section} % 公式分节标号

%%%%% Paper size
\usepackage{geometry}
\geometry{a4paper,left=2cm,right=2cm,bottom=2.5cm,top=2.5cm}

%%%%% Text
\usepackage{bm}
\usepackage{ulem} % underlining
\usepackage{enumitem} % customizing lists
\usepackage{titling} % 标题调整
\pretitle{           % 标题左对齐
  \begin{flushleft}
}
\posttitle{
  \end{flushleft}
}
\preauthor{
  \begin{flushleft}
}
\postauthor{
  \end{flushleft}
}
\predate{
  \begin{flushleft}
}
\postdate{
  \end{flushleft}
  \noindent\vrule height 1.0pt width \textwidth
  \vskip .75em plus .25em minus .25em% increase the vertical spacing a bit, make this particular glue stretchier
}
\usepackage{color}
\usepackage{setspace}
\renewcommand{\baselinestretch}{1.3}

%%%%% Ref
\usepackage[
  giveninits=true,       % 作者姓名首单词缩写, 并大写
  date = year,           % 日期只是年份
  % bibstyle = mystyle,  % 参考文献样式(文章最后的文献条目)
]{biblatex}
\renewbibmacro{in:}{} % 去掉文章前面的in
\DeclareFieldFormat[article,incollection,unpublished]{title}{#1} % No quotes for article titles
\DeclareFieldFormat[thesis]{title}{\mkbibemph{#1}} % Theses like book titles
\DeclareFieldFormat{journaltitle}{#1\isdot}
\DeclareFieldFormat{pages}{#1} % 去掉页码前面的pp
% 使用shortjournal代替journaltitle
\DeclareSourcemap{
  \maps[datatype=bibtex]{
    \map[overwrite]{ % Notice the overwrite: replace one field with another
      \step[fieldsource=shortjournal,fieldtarget=journaltitle]
    }
  }
}
\renewcommand*{\bibpagespunct}{\addspace} % This tells biblatex to only put a space right before the pages, no other punctuation.
\renewcommand*{\newunitpunct}{\addcomma\space}
\addbibresource{report.bib}
\usepackage[colorlinks,
            linkcolor=red,
            anchorcolor=red,
            citecolor=blue
            ]{hyperref}

%%%% Graphic
\usepackage{graphics} %color
\usepackage{tikz}
\usetikzlibrary{decorations.pathreplacing,calligraphy} % tikz 花括号

%%%%% Table
\usepackage{tabularx} %table column fix width
\newcolumntype{L}[1]{>{\raggedright\arraybackslash}p{#1}}
\newcolumntype{C}[1]{>{\centering\arraybackslash}p{#1}}
\newcolumntype{R}[1]{>{\raggedleft\arraybackslash}p{#1}}
\usepackage{booktabs} % three lines table
\usepackage{multirow} % multi row; multi column self-contained

%%%%% Figure
\usepackage{graphicx}  % minipage
\usepackage{subfig}    % subfig

%%%%% Code and other
% \usepackage[numbered,framed]{matlab-prettifier} %matlab
% \lstset{
%   style              = Matlab-editor,
%   basicstyle         = \mlttfamily,
%   escapechar         = ",
%   mlshowsectionrules = true,
% }
\usepackage{listings}
\usepackage{algorithm}
\usepackage{algorithmic}
\algsetup{
  indent=3em,
  linenosize=\small,
  linenodelimiter=.
}

%%%%%%% New command
%\newcommand{\char}[0]{\mathrm{char}\,}
\newcommand{\rank}[0]{\mathrm{rank}\,}
\newcommand{\Ker}[0]{\mathrm{Ker}\,}
\newcommand{\Img}[0]{\mathrm{Im}\,}

\title{{\huge \bfseries
线性代数A 讲义02
       }}
\author{Caiyou Yuan}
\date{\today}

\begin{document}
\maketitle
% include, new page
% \section{线性方程组}
\begin{itemize}
    \item 矩阵的初等行变换
    \item 阶梯形矩阵,简化阶梯形矩阵
    \item Guass-Jordan算法, 无解/有唯一解/有无数解
    \item 数域, $Q,R,C$
\end{itemize}

\subsection*{例题}
\begin{itemize}
    \item[1.] 证明任意一个矩阵都可以经过一系列初等行变换化为(简化)阶梯形矩阵
    \vspace{2.5cm}

    \item[2.] 解如下线性方程组
    \begin{equation}
    \nonumber
        \left\{
        \begin{aligned}
            &x_1 + x_2 + \cdots + x_n = 1\\
            &x_2 + x_3 + \cdots + x_{n+1} = 2\\
            &x_3 + x_4 + \cdots + x_{n+2} = 3\\
            &\vdots\\
            &x_{n+1}+x_{n+2}+ \cdots + x_{2n} = n+1\\
        \end{aligned}
        \right.
    \end{equation} 
    \vspace{2.5cm}

    \item[3.] 解如下线性方程组
    \begin{equation}
    \nonumber
        \left\{
        \begin{aligned}
            &x_1 +2x_2 +\cdots +(n-1)x_{n-1} +nx_{n} =b_1\\
            &nx_1 +x_2 +\cdots +(n-2)x_{n-1} +(n-1)x_{n} =b_2\\
            &(n-1)x_1 +nx_2 +\cdots +(n-3)x_{n-1} +(n-2)x_{n} =b_3\\
            &\vdots\\
            &2x_1 +3x_2 +\cdots +nx_{n-1} +x_{n} =b_n\\
        \end{aligned}
        \right.
    \end{equation} 
    其中$b_1,\cdots b_n$为给定常数
    \vspace{2.5cm}
\end{itemize}

\section{行列式}
\subsection{n元排列}
\begin{itemize}
    \item 逆序数
    \item 奇/偶排列
\end{itemize}

\subsection*{例题}
这里使用$\tau(a_1a_2\cdots a_k)$表示排列$a_1a_2\cdots a_k$的逆序数.
\begin{itemize}
    \item[1.] 设$1,2,\cdots,n$的n元排列
    $a_1a_2\cdots a_k b_1b_2\cdots b_{n-k}$
    有$$\tau(a_1a_2\cdots a_k) = \tau(b_1b_2\cdots b_{n-k}) = 0$$
    那么$\tau(a_1a_2\cdots a_k b_1b_2\cdots b_{n-k})$是多少?
    $n,k,a_1,\cdots,a_k, b_1, \cdots, b_{n-k}$
    均已知.
    \vspace{2.5cm}

    \item[2.] 说明n(n>1)元排列中,奇偶排列各占一半
    \vspace{1.5cm}

    \item[3.]
    \begin{itemize}
        \item[(1)] 若$\tau(a_1a_2\cdots a_n) = r$, 那么
        $\tau(a_na_{n-1}\cdots a_1)$ 是多少?
        \vspace{1.5cm}
        \item[(2)] 计算所有n元排列的逆序数之和
        \vspace{1.5cm}
    \end{itemize} 
\end{itemize}

\subsection{n阶行列式的定义}
\begin{equation}
\nonumber
\begin{aligned}
\det{A} &= \sum_{j_1 j_2\cdots j_n} (-1)^{\tau(j_1 j_2\cdots j_n)}a_{1j_1}a_{2j_2}\cdots a_{nj_n}\\
        &= \sum_{i_1 i_2\cdots i_n} (-1)^{\tau(i_1 i_2\cdots i_n)}a_{i_11}a_{i_22}\cdots a_{i_nn}
\end{aligned}
\end{equation}

\subsection*{例题}
\begin{itemize}
    \item[1.] 下列行列式是$x$的几次多项式,求出$x^4$项和$x^3$项的系数
    \begin{equation}
    \nonumber
    \left|
        \begin{array}{rrrr}
        5x &x &1 &x\\
        1  &x &1 &-x\\
        3  &2 &x &1\\
        3  &1 &1 &x\\
        \end{array}
    \right|
    \end{equation} 
    \vspace{2.5cm}

    \item[2.] 计算下列行列式
    \begin{equation}
    \nonumber
    \left|
        \begin{array}{rrrrr}
        a_1 &a_2 &a_3 &\cdots &a_n\\
        b_2 &1   &0   &\cdots &0\\
        b_3 &0   &1   &\cdots &0\\
        \vdots &\vdots   &\vdots   & &\vdots\\
        b_n &0   &0   &\cdots &1\\
        \end{array}
    \right|
    \end{equation}
    \vspace{2.5cm}
\end{itemize}

\subsection{行列式的性质}
\subsection*{例题}
\begin{itemize}
    \item 计算下列行列式
    \begin{equation}
    \nonumber
    \left|
        \begin{array}{ccccc}
        x_1-a_1 &x_2     &x_3     &\cdots & x_n\\
        x_1     &x_2-a_2 &x_3     &\cdots & x_n\\
        x_1     &x_2     &x_3-a_3 &\cdots & x_n\\
        \vdots  &\vdots  &\vdots  &       & \vdots\\
        x_1     &x_2     &x_3     &\cdots & x_n-a_n\\
        \end{array}
    \right|
    \end{equation} 
    其中$a_i \ne 0, i = 1,\cdots,n.$
    
    \begin{solution}
    (方法一)第一行的-1倍加到2到n行,化为箭形矩阵行列式;
    (方法二)非对角元素补上减去0,每列拆分为两列, $2^n$个行列式中经有n个非零;
    (方法三)和方法二类似,但只对第一列拆分, 找到n阶结果和n-1阶结果的关系
    \end{solution}
    \vspace{2cm}
\end{itemize}

\subsection{行列式按一行(列)展开}
\begin{itemize}
    \item 代数余子式, 行列式按行展开
    \item Vandermonde行列式 
\end{itemize}

\subsection*{例题}
\begin{itemize}
    \item[1.] 计算下列行列式
    \begin{itemize}
        \item[(1)]
        \begin{equation}
        \nonumber
        \left|
            \begin{array}{cccccccc}
            2a      &a^2     &0       &0      &\cdots & 0 &0 &0\\
            1       &2a      &a^2     &0      &\cdots & 0 &0 &0\\
            0       &1       &2a      &a^2    &\cdots & 0 &0 &0\\
            \vdots  &\vdots  &\vdots  &\vdots &       & \vdots & \vdots &\vdots\\
            0       &0       &0       &0      &\cdots & 1 &2a &a^2\\
            0       &0       &0       &0      &\cdots & 0 &1 &2a\\
            \end{array}
        \right|
        \end{equation} 
        \vspace{2cm}

        \item[(2)]
        \begin{equation}
        \nonumber
        \left|
            \begin{array}{cccccccc}
            a+b      &ab     &0       &0      &\cdots & 0 &0 &0\\
            1       &a+b     &ab     &0      &\cdots & 0 &0 &0\\
            0       &1       &a+b      &ab    &\cdots & 0 &0 &0\\
            \vdots  &\vdots  &\vdots  &\vdots &       & \vdots & \vdots &\vdots\\
            0       &0       &0       &0      &\cdots & 1 &a+b &ab\\
            0       &0       &0       &0      &\cdots & 0 &1 &a+b\\
            \end{array}
        \right|
        \end{equation} 
        \vspace{2cm}

        \item[(3)]
        \begin{equation}
        \nonumber
        \left|
            \begin{array}{cccccccc}
            a       &b     &0       &0      &\cdots & 0 &0 &0\\
            c       &a     &b     &0      &\cdots & 0 &0 &0\\
            0       &c     &a      &b    &\cdots & 0 &0 &0\\
            \vdots  &\vdots  &\vdots  &\vdots &       & \vdots & \vdots &\vdots\\
            0       &0       &0       &0      &\cdots & c &a &b\\
            0       &0       &0       &0      &\cdots & 0 &c &a\\
            \end{array}
        \right|
        \end{equation} 
        \vspace{2cm}
    \end{itemize}

    \item[2.] 计算如下行列式
    \begin{equation}
    \nonumber
    \left|
        \begin{array}{ccccc}
        1       &1       &\cdots &1 & 1\\
        x_1     &x_2     &\cdots &x_{n-1} & x_n\\
        x_1^2   &x_2^2   &\cdots &x_{n-1}^2 & x_n^2\\
        \vdots  &\vdots  &       &\vdots & \vdots\\
        x_1^{n-2} &x_2^{n-2}     &\cdots &x_{n-1}^{n-2} & x_n^{n-2}\\
        x_1^{n} &x_2^{n}     &\cdots &x_{n-1}^{n} & x_n^{n}\\
        \end{array}
    \right|
    \end{equation} 
\end{itemize}
\paragraph{7.5等}
\begin{itemize}
    \item 重因式
    \item 复数域上的不可约多项式只有一次的
    \item 实数域上的不可约多项式都是一次的,或判别式小于零的二次多项式
    \item 有理数域上的不可约多项式可以是任意次数的
\end{itemize}

\paragraph{多项式理论的应用: $\lambda$-矩阵}
即矩阵的每个元素都是多项式环$K[\lambda]$中元素;矩阵乘法,加法以及行列式等概念,和数字矩阵类似

\paragraph{$\lambda$-矩阵的初等变换}
\begin{itemize}
  \item [(a)] 矩阵的两行/列互换位置
  \item [(b)] 矩阵的某一行/列乘以非零常数c
  \item [(c)] 矩阵的某一行/列加上另一行/列的$p(\lambda)$倍,其中$p(\lambda) \in K[\lambda]$.
\end{itemize}
如果$A(\lambda)$可以经过一系列行和列的初等变换化为$B(\lambda)$,
则称$A(\lambda)$和$B(\lambda)$等价.

\paragraph{例题}
\begin{itemize}
  \item[1.] 证明: 设$A(\lambda)$的左上角元素$a_{11}(\lambda) \ne 0$, 并且$A(\lambda)$中
至少有一个元素不能被它除尽,那么一定可以找到和$A(\lambda)$等价的$B(\lambda)$,
它的左上角元素也不为零,但是次数小于$a_{11}(\lambda)$的次数
  \vspace{3cm}
  \item[2.] 证明:任意一个非零的$s\times n$的$\lambda$-矩阵$A(\lambda)$都等价于如下形式的矩阵(被称为标准形式)
\begin{equation}
\nonumber
\begin{pmatrix}
  d_1(\lambda)&&&&&&\\
  &d_2(\lambda)&&&&&\\
  &&\ddots&&&&\\
  &&&d_r(\lambda)&&&\\
  &&&&0&&\\
  &&&&&\ddots&\\
  &&&&&&0\\
\end{pmatrix}
\end{equation}
其中$r\ge 1$, $d_i(\lambda)$是首一的多项式(被称为不变因子),且
$$d_i(\lambda)\,|\,d_{i+1}(\lambda)\quad i=1,2,\cdots, r-1.$$
\vspace{4cm}
\item[3.]用初等变换化$\lambda$-矩阵
\begin{equation}
\nonumber
A(\lambda) =
\begin{bmatrix}
1-\lambda& 2\lambda-1& \lambda\\
\lambda&   \lambda^2&  -\lambda\\
1+\lambda^2& \lambda^3+\lambda-1& -\lambda^2\\
\end{bmatrix}
\end{equation}
为标准型.
\vspace{3cm}
\end{itemize}

\paragraph{$\lambda$-矩阵标准形式的唯一性}
下面来借助行列式因子的概念来说明$\lambda$-矩阵标准形式的唯一性.
\begin{itemize}
  \item[1.] ($\lambda$-矩阵的秩)如果$A(\lambda)$中有一个$r\ge 1$级子式不为零,而所有的$r+1$级子式(如果有的话)全为零,
则称$A(\lambda)$的秩为$r$. 零矩阵的秩记为0.
  \item[2.] ($\lambda$-矩阵的行列式因子) 设$A(\lambda)$的秩为r,对于$1 \le k \le r$, $A(\lambda)$中全部$k$级子式的首一最大公因式
$D_{k}(\lambda)$称为$A(\lambda)$的$k$阶行列式因子.
\end{itemize}

\paragraph{例题}
\begin{itemize}
  \item[1.] 等价的$\lambda$-矩阵具有相同的秩和相同的各阶行列式因子.
  \vspace{2cm}
  \item[2.] 证明$\lambda$-矩阵的不变因子和行列式因子有如下关系
  $$
  d_1(\lambda) = D_1(\lambda),\quad
  d_2(\lambda) = \frac{D_2(\lambda)}{D_1(\lambda)},\quad
  \dots, \quad d_r = \frac{D_r(\lambda)}{D_{r-1}(\lambda)}
  $$ 
  \vspace{3cm}
  \item[3.] 说明$\lambda$-矩阵的标准形式是唯一的.
  \vspace{2cm} 
\end{itemize}

\paragraph{8.1}
\begin{itemize}
    \item 域
    \item 域$F$上的线性空间的定义
    \item 基
    \item 维数
    \item 过渡矩阵
\end{itemize}

\paragraph{例题}
\begin{itemize}
  \item[1.]
  \begin{itemize}
    \item [(a)] 把域$F$看成是$F$上的线性空间,求它的一个基和维数;
    \item [(b)] 把复数域$C$看成是实数域$R$上的线性空间,求它的一个基和维数;
  \end{itemize}
\end{itemize}
\vspace{1cm}

\begin{itemize}
  \item[2.]
  \begin{itemize}
    \item [(a)] 把实数域$R$看成是有理数域$Q$上的线性空间,证明:对于任意大于1的正整数,
  $$1, \sqrt[n]{3}, \sqrt[n]{3^2}, \cdots, \sqrt[n]{3^{n-1}}$$
  是线性无关的.\,(提示: 已知$g(x) = x^n-3$是Q上的不可约多项式)
    \item [(b)] 证明: 实数域$R$作为有理数域$Q$上的线性空间是无穷维的.
  \end{itemize}
\end{itemize}
\vspace{1.5cm}

\begin{itemize}
  \item[3.] 设$V$是域$F$上的n维线性空间,域$F$包含域$E$,\,$F$可看作域$E$
  上的$m$维线性空间
  \begin{itemize}
    \item [(a)] 求证:V可以成为域$E$上的线性空间
    \item [(b)] 证明: 求V作为域E上线性空间的维数
  \end{itemize}
\end{itemize}
\vspace{1.5cm}

\begin{itemize}
  \item[4.] (Complexification of real vector space)
  设$V$是数域R上的$n$维线性空间,设$V_C = \{(u,v), u,v\in V\}$,
  定义$V_C$上的加法
  $$(u_1,v_1) +(u_2, v_2) = (u_1+u_2, v_1+v_2)$$
  以及C上的数乘
  $$(a+bi)(u,v)=(au-bv, av+bu)$$
  \begin{itemize}
    \item [(a)] 求证:$V_C$是一个复线性空间
    \item [(b)] 计算$V_C$的维数
  \end{itemize}
\end{itemize}
\vspace{1cm}

\begin{itemize}
  \item[5.] (对偶空间)设$V$是数域K上的$n$维线性空间,
  考虑复数域$C$上的线性空间$C^V$(从$V$到$R$的函数全体)中具有下述性质的函数组成的子集$W$:
  $$f(\alpha+\beta)=f(\alpha) + f(\beta),\quad \forall \alpha, \beta \in V,$$
  $$f(k\alpha)=kf(\alpha),\quad \forall \alpha \in V, k \in K.$$
  \begin{itemize}
    \item [(a)] 求证:$W$是一个复线性空间
    \item [(b)] 求$W$的一个基和维数;设$f\in W$, 求$f$在这个基下的坐标
  \end{itemize}
\end{itemize}
\vspace{1cm}

\begin{itemize}
  \item[6.] (零化多项式和最小多项式)
  设A是数域K上的一个非零n阶矩阵,说明$K[A]$是$K$上的一个线性空间.
  $K[A]$至多多少维?\footnote{Hamilton-Cayley定理告诉我们,K[A]至多n维.}
\end{itemize}
\vspace{1cm}

\begin{itemize}
  \item[7.] 设递推方程
  $$u_n = au_{n-1} +bu_{n-2}, \quad n \ge 2,$$
  其中$a,b$都是非零复数. 若N上的一个复值数列$u_n$满足上述递推关系,则称为上述递推方程的解.
  一元多项式$f(x)=x^2 - ax -b$称为上述递推方程的特征多项式.
  求证
  \begin{itemize}
    \item [(a)] 上述递推方程的解集$W$是一个复线性空间
    \item [(b)] 设$\alpha$是一个非零复数,则$\alpha^n \in W$ 当且仅当$f(\alpha)=0$
    \item [(c)] 设$\alpha$是一个非零复数,则$n\alpha^n \in W$ 当且仅当$f(\alpha)=0, f'(\alpha)=0$.
    \item [(d)] 若$f(x)$有两不同的根$\alpha_1, \alpha_2$, 则任意$u_n \in W$, 可以表示为
    $$ u_n = C_1 \alpha_1^n +C_2 \alpha_2^n$$
    其中$C_1, C_2$是常数;
    \item [(e)] 若$f(x)$有二重根$\alpha$, 则任意$u_n \in W$, 可以表示为
    $$ u_n = C_1 \alpha^n +C_2 n\alpha^n$$
    其中$C_1, C_2$是常数;
  \end{itemize}
\end{itemize}
\vspace{6cm}


\paragraph{8.2}
\begin{itemize}
    \item 子空间
    \item 子空间的维数定理
    $$\dim V_1 + \dim V_2 = \dim (V_1 + V_2) + \dim (V_1 \cap V_2)$$
    \item 直和
\end{itemize}

\paragraph{例题}
\begin{itemize}
\item[1.] 设$V_1, V_2, V_3$都是域F上的有限维线性空间$V$的子空间
\begin{itemize}
    \item[(a)] 求证:
    $$V_3 \cap V_1 + V_3 \cap V_2 \subset V_3 \cap (V_1 + V_2)$$ 
    \item[(b)] 如果$V_3 \subset V_1 + V_2$,
试问$(V_3\cap V_1) + (V_3\cap V_2) = V_3$是否总成立?如果再加上条件$V_1 \subset V_3$呢?
    \item[(c)] 求证,
    \begin{equation}
    \nonumber
    \begin{aligned}
    \dim V_1 + \dim V_2 + \dim V_3 \ge
    &\dim(V_1 + V_2 + V_3)\\
    &+\dim(V_1 \cap V_2) +\dim(V_1 \cap V_3) +\dim(V_2 \cap V_3)\\
    &-\dim(V_1 \cap V_2 \cap V_3)
    \end{aligned}
    \end{equation}  
  \end{itemize}
\end{itemize}
\vspace{3cm}

\begin{itemize}
  \item[2.] 设$V_1, \cdots, V_s$都是域F上线性空间V的子空间,
  证明$V_1 + V_2 + \cdots + V_s$是直和
  \begin{itemize} 
    \item[(a)] 当且仅当,
    $$V_i \cap \left(\sum_{j\ne i} V_j \right) = 0, \quad i=1,2,\dots,s$$
    \item[(b)] 当且仅当,$V$中有一向量$\alpha$可以唯一表示为
    $$\alpha = \sum_{i=1}^s \alpha_i, \quad \alpha_i \in V_i$$
    \item[(c)] 当且仅当,
    $$\dim(V_1 + V_2 + \cdots + V_s) = \dim V_1 + \cdots \dim V_s$$ 
  \end{itemize}
\end{itemize}
\vspace{3cm}

\begin{itemize}
  \item[3.] 设A是数域K上的一个n阶矩阵,$\lambda_1, \lambda_2, \cdots, \lambda_s$
  是A的全部不同的特征值,用$V_{\lambda_i}$表示A的属于$\lambda_i$的特征子空间. 证明:
  A可以对角化的充分必要条件是
  $$K^n = V_{\lambda_1} \oplus V_{\lambda_2} \oplus \cdots \oplus V_{\lambda_s}$$
\end{itemize}
\vspace{1cm}

\begin{itemize}
  \item[4.] 在数域K上的线性空间$K^{M_n(K)}$中,如果$f$满足,
  对于任意的$A=(\alpha_1, \alpha_2,\cdots, \alpha_n)$,任意的n维列向量$\alpha$, 以及任意$k\in K$, $j\in \{1,2,\cdots, n\}$,
  有 
  $$f(\alpha_1, \cdots, \alpha_{j-1}, \alpha_{j} + \alpha, \alpha_{j+1}, \cdots, \alpha_n)
  = f(\alpha_1, \cdots, \alpha_{j-1}, \alpha_{j}, \alpha_{j+1}, \cdots, \alpha_n) + 
    f(\alpha_1, \cdots, \alpha_{j-1}, \alpha, \alpha_{j+1}, \cdots, \alpha_n)$$
  $$f(\alpha_1, \cdots, \alpha_{j-1}, k\alpha_{j}, \alpha_{j+1}, \cdots, \alpha_n)
  = kf(\alpha_1, \cdots, \alpha_{j-1}, \alpha_{j}, \alpha_{j+1}, \cdots, \alpha_n)
  $$
  那么称$f$是$M_n(K)$上的列线性函数. 同理如果$g(A^T)$是列线性函数,则称$g(A)$是行线性函数。
  记所有的列/行线性函数组成的集合分别记为$V_1$和$V_2$.
  \begin{itemize} 
    \item[(a)] 证明: $V_1, V_2$都是$K^{M_n(K)}$的子空间
    \item[(b)] 分别求$V_1, V_2$的一个基和维数
    \item[(c)] 分别求$V_1 \cap V_2,\,V_1 + V_2$的一个基和维数 
  \end{itemize}
\end{itemize}
\vspace{6cm}

\paragraph{8.3}
\begin{itemize}
    \item 线性空间的同构
    \item 有限维线性空间同构的充要条件
\end{itemize}

\paragraph{例题}
\begin{itemize}
  \item[1.]令
  \begin{equation}
  \nonumber
  H = \left\{
    \begin{bmatrix}
      z_1& z_2\\
     -z_2& z_1\\
    \end{bmatrix},
    z_1, z_2 \in C
      \right\}
  \end{equation}
  \begin{itemize}
    \item[(a)] H对于矩阵的加法,以及实数和矩阵的乘法构成一个实线性空间
    \item[(b)] 给出H的一个基和维数
    \item[(c)] 证明: $H$与$R^4$同构,并写出$H$到$R^4$的一个同构映射  
  \end{itemize}
\end{itemize}
\vspace{1cm}

\begin{itemize}
  \item[2.]设$A \in M_n(K)$, 令$AM_n(K) = \{AB, B\in M_n(K) \}$.
  \begin{itemize}
    \item[(a)] 证明$AM_n(K)$是数域$K$上线性空间$M_n(K)$的子空间
    \item[(b)] 设A的列向量组$\alpha_1, \cdots, \alpha_n$的一个极大线性无关组为
    $\alpha_{j_1}, \cdots, \alpha_{j_r}$,证明$AM_n(K)$和$M_{r\times n}(K)$同构,并
    写出一个同构映射. 
    \item[(c)] 证明: $\dim [AM_n(K)] = \rank(A)n$.
  \end{itemize}
\end{itemize}
\vspace{1cm}

\end{document}
