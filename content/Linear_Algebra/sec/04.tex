\paragraph{8.5}
\begin{itemize}
    \item 不变子空间
    \begin{itemize}
        \item[1.] A有非平凡子空间 $\Longleftrightarrow$ V中存在一个基,A在此基下的矩阵为分块上三角
        \item[2.] A能分解成一些非平凡不变子空间的直和 $\Longleftrightarrow$ V在某个基下的矩阵为分块对角
    \end{itemize}
    \item 若$f(x)=f_1(x)f_2(x), (f_1(x),f_2(x))=1$, 则
    $$\Ker f(A) = \Ker f_1(A) \oplus \Ker f_2(A)$$
\end{itemize}

\paragraph{例题}
\begin{itemize}
    \item [1.] 设$\mathbf{A}$是域$F$上$n$维线性空间$V$上的线性变换,$W$是$\mathbf{A}$的一个非平凡不变子空间,
    在$W$中取一个基$\alpha_1, \cdots, \alpha_r$, 
    把它扩充成$V$的一个基$\alpha_1, \cdots, \alpha_n$,
    则$\mathbf{A}$在$V$的矩阵为
    \begin{equation}
    \nonumber
    \begin{pmatrix}
        A_1& A_3\\
        0  & A_2\\
    \end{pmatrix}
    \end{equation}
    \begin{itemize}
        \item [(a) ] 证明$A_2$是商变换
        \begin{equation}
        \nonumber
        \begin{aligned}
        \mathbf{\hat A}: \quad &V/W \longrightarrow V/W\\
                         &\alpha + W \longrightarrow \mathbf{A}\alpha + W.
        \end{aligned}
        \end{equation}
        在基
        $\alpha_{r+1} + W, \alpha_{r+2}+W, \cdots, \alpha_n+W$下的矩阵.
        \vspace{2cm}
        \item [(b) ] 设$\mathbf{A}, \mathbf{A}|_W, \mathbf{\hat A}$的特征多项式是
        $f(\lambda),f_1(\lambda), f_2(\lambda)$, 证明
        $$f(\lambda) = f_1(\lambda)f_2(\lambda)$$
        \vspace{2cm}
    \end{itemize}

    \item [2.] (同时对角化的充分必要条件) 设$A_1, \cdots, A_m$是域$F$上线性空间$V$上的线性变换,每一个都可以对角化,
    证明: 他们可以在某一组基下同时对角化的充分必要条件是他们的乘积可交换,即$A_i A_j = A_j A_i$.
    \vspace{4cm}

    \item [3.] 设$A$是域$F$上线性空间$V$的一个线性变换,设$f(x),g(x) \in F[x]$,
    $(f(x),g(x))=d(x), \left[f(x),g(x)\right]=m(x)$.
    \begin{itemize}
        \item [(a) ] 证明,
        $$\Ker d(A) = \Ker f(A) \cap \Ker g(A)$$
        $$\Ker m(A) = \Ker f(A) + \Ker g(A)$$
        \item [(b) ] 证明,
        $$\rank(f(A)) +\rank(g(A)) = \rank(d(A)) + \rank(m(A))$$
    \end{itemize}
    \vspace{3cm}
\end{itemize}


\paragraph{8.6}
\begin{itemize}
    \item Hamilton-Cayley定理: 对于线性变换$A$的特征多项式$f(x)$,有$f(A)=0$
    \item 线性变换的最小多项式
    \begin{itemize}
        \item [1. ] 如果$V$能分解成一些非平凡不变子空间的直和,
        $$V = W_1 \oplus W_2 \cdots \oplus W_s$$
        则$A$的最小多项式$m(\lambda) = [m_1(\lambda), m_2(\lambda), \cdots, m_s(\lambda)]$,其中
        $m_j(\lambda)$是$W_j$上变换$A|_{W_{j}}$的最小多项式.
        \item [2. ] $A$可以对角化的充分必要条件是,$A$的最小多项式$m(\lambda)$可以在$F[\lambda]$
        中分解成不同的一次因式的乘积.
    \end{itemize}
\end{itemize}

\paragraph{例题}
\begin{itemize}
    \item [1.] 设$A,B$分别是$n,m$阶复矩阵. 证明:矩阵方程$AX-XB=0$只有零解的充分必要条件是$A$和$B$没有公共特征值.
    \vspace{3cm}
    \item [2.] Hamilton-Cayley定理对于实数域上的线性变换成立么?
    \vspace{3cm}
    \item [3.] 设$A$是域F上的$n$维线性空间$V$上的线性变换,A有$s$个不同的特征值,可以对角化,
    属于特征值$\lambda_i$的特征子空间的维数是$n_i$.
    \begin{itemize}
        \item [(a) ] 证明,
        $$\dim C(A) = \sum_{i} n_i^2$$
        \item [(b) ] 说明若$s<n$,
        $$F[A] \subsetneq C(A)$$
        若$s=n$,
        $$F[A] = C(A)$$
    \end{itemize}
\end{itemize}


