\paragraph{7.5等}
\begin{itemize}
    \item 重因式
    \item 复数域上的不可约多项式只有一次的
    \item 实数域上的不可约多项式都是一次的,或判别式小于零的二次多项式
    \item 有理数域上的不可约多项式可以是任意次数的
\end{itemize}

\paragraph{多项式理论的应用: $\lambda$-矩阵}
即矩阵的每个元素都是多项式环$K[\lambda]$中元素;矩阵乘法,加法以及行列式等概念,和数字矩阵类似

\paragraph{$\lambda$-矩阵的初等变换}
\begin{itemize}
  \item [(a)] 矩阵的两行/列互换位置
  \item [(b)] 矩阵的某一行/列乘以非零常数c
  \item [(c)] 矩阵的某一行/列加上另一行/列的$p(\lambda)$倍,其中$p(\lambda) \in K[\lambda]$.
\end{itemize}
如果$A(\lambda)$可以经过一系列行和列的初等变换化为$B(\lambda)$,
则称$A(\lambda)$和$B(\lambda)$等价.

\paragraph{例题}
\begin{itemize}
  \item[1.] 证明: 设$A(\lambda)$的左上角元素$a_{11}(\lambda) \ne 0$, 并且$A(\lambda)$中
至少有一个元素不能被它除尽,那么一定可以找到和$A(\lambda)$等价的$B(\lambda)$,
它的左上角元素也不为零,但是次数小于$a_{11}(\lambda)$的次数
  \vspace{3cm}
  \item[2.] 证明:任意一个非零的$s\times n$的$\lambda$-矩阵$A(\lambda)$都等价于如下形式的矩阵(被称为标准形式)
\begin{equation}
\nonumber
\begin{pmatrix}
  d_1(\lambda)&&&&&&\\
  &d_2(\lambda)&&&&&\\
  &&\ddots&&&&\\
  &&&d_r(\lambda)&&&\\
  &&&&0&&\\
  &&&&&\ddots&\\
  &&&&&&0\\
\end{pmatrix}
\end{equation}
其中$r\ge 1$, $d_i(\lambda)$是首一的多项式(被称为不变因子),且
$$d_i(\lambda)\,|\,d_{i+1}(\lambda)\quad i=1,2,\cdots, r-1.$$
\vspace{4cm}
\item[3.]用初等变换化$\lambda$-矩阵
\begin{equation}
\nonumber
A(\lambda) =
\begin{bmatrix}
1-\lambda& 2\lambda-1& \lambda\\
\lambda&   \lambda^2&  -\lambda\\
1+\lambda^2& \lambda^3+\lambda-1& -\lambda^2\\
\end{bmatrix}
\end{equation}
为标准型.
\vspace{3cm}
\end{itemize}

\paragraph{$\lambda$-矩阵标准形式的唯一性}
下面来借助行列式因子的概念来说明$\lambda$-矩阵标准形式的唯一性.
\begin{itemize}
  \item[1.] ($\lambda$-矩阵的秩)如果$A(\lambda)$中有一个$r\ge 1$级子式不为零,而所有的$r+1$级子式(如果有的话)全为零,
则称$A(\lambda)$的秩为$r$. 零矩阵的秩记为0.
  \item[2.] ($\lambda$-矩阵的行列式因子) 设$A(\lambda)$的秩为r,对于$1 \le k \le r$, $A(\lambda)$中全部$k$级子式的首一最大公因式
$D_{k}(\lambda)$称为$A(\lambda)$的$k$阶行列式因子.
\end{itemize}

\paragraph{例题}
\begin{itemize}
  \item[1.] 等价的$\lambda$-矩阵具有相同的秩和相同的各阶行列式因子.
  \vspace{2cm}
  \item[2.] 证明$\lambda$-矩阵的不变因子和行列式因子有如下关系
  $$
  d_1(\lambda) = D_1(\lambda),\quad
  d_2(\lambda) = \frac{D_2(\lambda)}{D_1(\lambda)},\quad
  \dots, \quad d_r = \frac{D_r(\lambda)}{D_{r-1}(\lambda)}
  $$ 
  \vspace{3cm}
  \item[3.] 说明$\lambda$-矩阵的标准形式是唯一的.
  \vspace{2cm} 
\end{itemize}

\paragraph{8.1}
\begin{itemize}
    \item 域
    \item 域$F$上的线性空间的定义
    \item 基
    \item 维数
    \item 过渡矩阵
\end{itemize}

\paragraph{例题}
\begin{itemize}
  \item[1.]
  \begin{itemize}
    \item [(a)] 把域$F$看成是$F$上的线性空间,求它的一个基和维数;
    \item [(b)] 把复数域$C$看成是实数域$R$上的线性空间,求它的一个基和维数;
  \end{itemize}
\end{itemize}
\vspace{1cm}

\begin{itemize}
  \item[2.]
  \begin{itemize}
    \item [(a)] 把实数域$R$看成是有理数域$Q$上的线性空间,证明:对于任意大于1的正整数,
  $$1, \sqrt[n]{3}, \sqrt[n]{3^2}, \cdots, \sqrt[n]{3^{n-1}}$$
  是线性无关的.\,(提示: 已知$g(x) = x^n-3$是Q上的不可约多项式)
    \item [(b)] 证明: 实数域$R$作为有理数域$Q$上的线性空间是无穷维的.
  \end{itemize}
\end{itemize}
\vspace{1.5cm}

\begin{itemize}
  \item[3.] 设$V$是域$F$上的n维线性空间,域$F$包含域$E$,\,$F$可看作域$E$
  上的$m$维线性空间
  \begin{itemize}
    \item [(a)] 求证:V可以成为域$E$上的线性空间
    \item [(b)] 证明: 求V作为域E上线性空间的维数
  \end{itemize}
\end{itemize}
\vspace{1.5cm}

\begin{itemize}
  \item[4.] (Complexification of real vector space)
  设$V$是数域R上的$n$维线性空间,设$V_C = \{(u,v), u,v\in V\}$,
  定义$V_C$上的加法
  $$(u_1,v_1) +(u_2, v_2) = (u_1+u_2, v_1+v_2)$$
  以及C上的数乘
  $$(a+bi)(u,v)=(au-bv, av+bu)$$
  \begin{itemize}
    \item [(a)] 求证:$V_C$是一个复线性空间
    \item [(b)] 计算$V_C$的维数
  \end{itemize}
\end{itemize}
\vspace{1cm}

\begin{itemize}
  \item[5.] (对偶空间)设$V$是数域K上的$n$维线性空间,
  考虑复数域$C$上的线性空间$C^V$(从$V$到$R$的函数全体)中具有下述性质的函数组成的子集$W$:
  $$f(\alpha+\beta)=f(\alpha) + f(\beta),\quad \forall \alpha, \beta \in V,$$
  $$f(k\alpha)=kf(\alpha),\quad \forall \alpha \in V, k \in K.$$
  \begin{itemize}
    \item [(a)] 求证:$W$是一个复线性空间
    \item [(b)] 求$W$的一个基和维数;设$f\in W$, 求$f$在这个基下的坐标
  \end{itemize}
\end{itemize}
\vspace{1cm}

\begin{itemize}
  \item[6.] (零化多项式和最小多项式)
  设A是数域K上的一个非零n阶矩阵,说明$K[A]$是$K$上的一个线性空间.
  $K[A]$至多多少维?\footnote{Hamilton-Cayley定理告诉我们,K[A]至多n维.}
\end{itemize}
\vspace{1cm}

\begin{itemize}
  \item[7.] 设递推方程
  $$u_n = au_{n-1} +bu_{n-2}, \quad n \ge 2,$$
  其中$a,b$都是非零复数. 若N上的一个复值数列$u_n$满足上述递推关系,则称为上述递推方程的解.
  一元多项式$f(x)=x^2 - ax -b$称为上述递推方程的特征多项式.
  求证
  \begin{itemize}
    \item [(a)] 上述递推方程的解集$W$是一个复线性空间
    \item [(b)] 设$\alpha$是一个非零复数,则$\alpha^n \in W$ 当且仅当$f(\alpha)=0$
    \item [(c)] 设$\alpha$是一个非零复数,则$n\alpha^n \in W$ 当且仅当$f(\alpha)=0, f'(\alpha)=0$.
    \item [(d)] 若$f(x)$有两不同的根$\alpha_1, \alpha_2$, 则任意$u_n \in W$, 可以表示为
    $$ u_n = C_1 \alpha_1^n +C_2 \alpha_2^n$$
    其中$C_1, C_2$是常数;
    \item [(e)] 若$f(x)$有二重根$\alpha$, 则任意$u_n \in W$, 可以表示为
    $$ u_n = C_1 \alpha^n +C_2 n\alpha^n$$
    其中$C_1, C_2$是常数;
  \end{itemize}
\end{itemize}
\vspace{6cm}


\paragraph{8.2}
\begin{itemize}
    \item 子空间
    \item 子空间的维数定理
    $$\dim V_1 + \dim V_2 = \dim (V_1 + V_2) + \dim (V_1 \cap V_2)$$
    \item 直和
\end{itemize}

\paragraph{例题}
\begin{itemize}
\item[1.] 设$V_1, V_2, V_3$都是域F上的有限维线性空间$V$的子空间
\begin{itemize}
    \item[(a)] 求证:
    $$V_3 \cap V_1 + V_3 \cap V_2 \subset V_3 \cap (V_1 + V_2)$$ 
    \item[(b)] 如果$V_3 \subset V_1 + V_2$,
试问$(V_3\cap V_1) + (V_3\cap V_2) = V_3$是否总成立?如果再加上条件$V_1 \subset V_3$呢?
    \item[(c)] 求证,
    \begin{equation}
    \nonumber
    \begin{aligned}
    \dim V_1 + \dim V_2 + \dim V_3 \ge
    &\dim(V_1 + V_2 + V_3)\\
    &+\dim(V_1 \cap V_2) +\dim(V_1 \cap V_3) +\dim(V_2 \cap V_3)\\
    &-\dim(V_1 \cap V_2 \cap V_3)
    \end{aligned}
    \end{equation}  
  \end{itemize}
\end{itemize}
\vspace{3cm}

\begin{itemize}
  \item[2.] 设$V_1, \cdots, V_s$都是域F上线性空间V的子空间,
  证明$V_1 + V_2 + \cdots + V_s$是直和
  \begin{itemize} 
    \item[(a)] 当且仅当,
    $$V_i \cap \left(\sum_{j\ne i} V_j \right) = 0, \quad i=1,2,\dots,s$$
    \item[(b)] 当且仅当,$V$中有一向量$\alpha$可以唯一表示为
    $$\alpha = \sum_{i=1}^s \alpha_i, \quad \alpha_i \in V_i$$
    \item[(c)] 当且仅当,
    $$\dim(V_1 + V_2 + \cdots + V_s) = \dim V_1 + \cdots \dim V_s$$ 
  \end{itemize}
\end{itemize}
\vspace{3cm}

\begin{itemize}
  \item[3.] 设A是数域K上的一个n阶矩阵,$\lambda_1, \lambda_2, \cdots, \lambda_s$
  是A的全部不同的特征值,用$V_{\lambda_i}$表示A的属于$\lambda_i$的特征子空间. 证明:
  A可以对角化的充分必要条件是
  $$K^n = V_{\lambda_1} \oplus V_{\lambda_2} \oplus \cdots \oplus V_{\lambda_s}$$
\end{itemize}
\vspace{1cm}

\begin{itemize}
  \item[4.] 在数域K上的线性空间$K^{M_n(K)}$中,如果$f$满足,
  对于任意的$A=(\alpha_1, \alpha_2,\cdots, \alpha_n)$,任意的n维列向量$\alpha$, 以及任意$k\in K$, $j\in \{1,2,\cdots, n\}$,
  有 
  $$f(\alpha_1, \cdots, \alpha_{j-1}, \alpha_{j} + \alpha, \alpha_{j+1}, \cdots, \alpha_n)
  = f(\alpha_1, \cdots, \alpha_{j-1}, \alpha_{j}, \alpha_{j+1}, \cdots, \alpha_n) + 
    f(\alpha_1, \cdots, \alpha_{j-1}, \alpha, \alpha_{j+1}, \cdots, \alpha_n)$$
  $$f(\alpha_1, \cdots, \alpha_{j-1}, k\alpha_{j}, \alpha_{j+1}, \cdots, \alpha_n)
  = kf(\alpha_1, \cdots, \alpha_{j-1}, \alpha_{j}, \alpha_{j+1}, \cdots, \alpha_n)
  $$
  那么称$f$是$M_n(K)$上的列线性函数. 同理如果$g(A^T)$是列线性函数,则称$g(A)$是行线性函数。
  记所有的列/行线性函数组成的集合分别记为$V_1$和$V_2$.
  \begin{itemize} 
    \item[(a)] 证明: $V_1, V_2$都是$K^{M_n(K)}$的子空间
    \item[(b)] 分别求$V_1, V_2$的一个基和维数
    \item[(c)] 分别求$V_1 \cap V_2,\,V_1 + V_2$的一个基和维数 
  \end{itemize}
\end{itemize}
\vspace{6cm}

\paragraph{8.3}
\begin{itemize}
    \item 线性空间的同构
    \item 有限维线性空间同构的充要条件
\end{itemize}

\paragraph{例题}
\begin{itemize}
  \item[1.]令
  \begin{equation}
  \nonumber
  H = \left\{
    \begin{bmatrix}
      z_1& z_2\\
     -z_2& z_1\\
    \end{bmatrix},
    z_1, z_2 \in C
      \right\}
  \end{equation}
  \begin{itemize}
    \item[(a)] H对于矩阵的加法,以及实数和矩阵的乘法构成一个实线性空间
    \item[(b)] 给出H的一个基和维数
    \item[(c)] 证明: $H$与$R^4$同构,并写出$H$到$R^4$的一个同构映射  
  \end{itemize}
\end{itemize}
\vspace{1cm}

\begin{itemize}
  \item[2.]设$A \in M_n(K)$, 令$AM_n(K) = \{AB, B\in M_n(K) \}$.
  \begin{itemize}
    \item[(a)] 证明$AM_n(K)$是数域$K$上线性空间$M_n(K)$的子空间
    \item[(b)] 设A的列向量组$\alpha_1, \cdots, \alpha_n$的一个极大线性无关组为
    $\alpha_{j_1}, \cdots, \alpha_{j_r}$,证明$AM_n(K)$和$M_{r\times n}(K)$同构,并
    写出一个同构映射. 
    \item[(c)] 证明: $\dim [AM_n(K)] = \rank(A)n$.
  \end{itemize}
\end{itemize}
\vspace{1cm}