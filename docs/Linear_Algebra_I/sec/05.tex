\chapter{矩阵相抵和相似}

\section{矩阵相抵}
\begin{itemize}
\item 如果矩阵$A$可以通过初等行/列变换为矩阵$B$, 则称$A,B$相抵
\item 矩阵相抵是$M_{m\times n}(K)$上的一个等价关系
\item $M_{m\times n}(K)$中的两矩阵相抵当且秩相等
\end{itemize}

\subsection*{例题}
\begin{itemize}
	\item[1.] 设$A,B,C$分别是数域$K$上$s\times n, p\times m, s\times m$矩阵,
	证明矩阵方程$AX-YB=C$有解的充分必要条件是
	\begin{equation}
	\nonumber
	\mathrm{rank}
	\begin{bmatrix}
		A&0\\
		0&B
	\end{bmatrix}
	=
	\mathrm{rank}
	\begin{bmatrix}
		A&C\\
		0&B
	\end{bmatrix}
	\end{equation}
	\begin{solution}
		必要性将$C=AX-YB$代入即可; 充分性,
		假设$\mathrm{rank}(A)=a, \mathrm{rank}(B)=b$,
		则存在可逆矩阵$P_a, Q_a, P_b, Q_b$, 使得
		\begin{equation*}
			P_a A Q_a = \begin{bmatrix}
				I_a&0\\
				0&0\\
			\end{bmatrix},\quad
			P_b B Q_b = \begin{bmatrix}
				I_b&0\\
				0&0\\
			\end{bmatrix}
		\end{equation*}
		所以
		\begin{equation*}
		\begin{bmatrix}
			P_a&\\
			&P_b\\
		\end{bmatrix}
		\begin{bmatrix}
			A&C\\
			0&B\\
		\end{bmatrix}
		\begin{bmatrix}
			Q_a&\\
			&Q_b\\
		\end{bmatrix}
		=
		\begin{bmatrix}
			I_a&0& C_{11}& C_{12}\\
			0&0& C_{21}& C_{22}\\
			0&0& I_b& 0\\
			0&0& 0& 0\\
		\end{bmatrix}
		\end{equation*}
		其中
		\begin{equation*}
		P_a C Q_b = \begin{bmatrix}
			C_{11}&C_{12}\\
			C_{21}&C_{22}\\
		\end{bmatrix}
		\end{equation*}
		由已知条件, $C_{22}=0$. 所以
		\begin{equation*}
		\begin{bmatrix}
			I& &-C_{11}& \\
			&I &-C_{21}&\\
			& &I&\\
			& &&I\\
		\end{bmatrix}
		\begin{bmatrix}
			P_a&\\
			&P_b\\
		\end{bmatrix}
		\begin{bmatrix}
			A&C\\
			0&B\\
		\end{bmatrix}
		\begin{bmatrix}
			Q_a&\\
			&Q_b\\
		\end{bmatrix}
		\begin{bmatrix}
			I& & &-C_{12}\\
			&I & &\\
			& &I&\\
			& &&I\\
		\end{bmatrix}
		=
		\begin{bmatrix}
			I_a&0&0&0\\
			0&0&0&0\\
			0&0& I_b& 0\\
			0&0& 0& 0\\
		\end{bmatrix}
		\end{equation*}
		注意右上角的分块计算结果为0, 整理即可获得原矩阵方程的解.
	\end{solution}
	\vspace{5cm}
\end{itemize}

\section{广义逆矩阵}
\begin{itemize}
\item (矩阵相抵标准型的应用)如果$A$的相抵标准型为
\begin{equation*}
	A = P\begin{bmatrix}
		I_r& 0\\
		0& 0\\
	\end{bmatrix}
	Q
\end{equation*}
其中$P,Q$分别是$s,n$级可逆矩阵,
那么矩阵方程$AXA=A$通解为
\begin{equation*}
	X = Q^{-1}\begin{bmatrix}
		I_r& B\\
		C& D\\
	\end{bmatrix}
	P^{-1}
\end{equation*}
其中$B,C,D$分别是任意的$r\times (s-r), (n-r)\times r, (n-r)\times (s-r)$
矩阵. 证明?

\item 非齐次线性方程组$AX=\beta$的通解为$X = A^{-}\beta$, 其中
$A^{-}$是$A$的任意一个广义逆. 证明?
\end{itemize}

\subsection*{例题}
\begin{itemize}
	\item[1.] (广义逆的一个充分必要条件)
	设$A,B$分别是数域$K$上的$s\times n, n\times s$的矩阵
	\begin{itemize}
		\item [(a)] 证明$\rank{A-ABA} = \rank{A} + \rank{I-BA} - n$ 
		\begin{solution}
			Sylvester不等式,考虑等式成立的条件
		\end{solution}
		\vspace{2cm}
		\item [(b)] 证明$B$是$A$的一个广义逆的充分必要条件是$\rank{A} + \rank{I-BA} = n$.
		\vspace{1.5cm}
	\end{itemize}
	\item[2.] (两两正交的幂等矩阵的充分必要条件)
	设$A_1, A_2, \cdots, A_s$都是数域$K$上的$n$级矩阵
	\begin{itemize}
		\item [(a)]令$D = \mathrm{diag}\{A_1,A_2,\cdots, A_s\}$,
		$E = (\underbrace{I_n, I_n, \cdots, I_n}_{s\text{个}})$. 证明:$A_1, A_2, \cdots, A_s$都是幂等矩阵,且$A_iA_j=0$(当$i \ne j$)的
		充分必要条件是$E^T E$是$D$的一个广义逆.
		\begin{solution}
			验证$DE^TED=D$
		\end{solution}
		\vspace{1.5cm}
		\item [(b)]令$A=\sum_{i=1}^s A_i$, 证明:$A_1, A_2, \cdots, A_s$都是幂等矩阵,
		且$A_iA_j=0$(当$i \ne j$)的
		充分必要条件是$A$是幂等矩阵,且$\rank{A} = \sum_{i=1}^s \rank{A_i}$.
		\begin{solution}
		使用第一题的结论,验证$E^T E$是$D$的一个广义逆. 其中$\rank{I-E^TED} = n(s-1)+\rank{I_n-A}$(使用初等变换化为块对角).
		\end{solution}
		\vspace{2cm}
	\end{itemize}
\end{itemize}

\section{矩阵的相似和对角化}
\begin{itemize}
\item 相似矩阵有相同的行列式,秩,迹,特征多项式,特征值
\item $n$级矩阵可对角化的充分必要条件是有$n$个线性无关的特征向量
\item 如何求矩阵的所有线性无关的特征向量?
\end{itemize}

\subsection*{例题}
\begin{itemize}
	\item[1.] (Frobenius矩阵)复数域上的矩阵
	\begin{equation}
	\nonumber
	A = \begin{bmatrix}
		0& 1& 0& \cdots& 0& 0\\
		0& 0& 1& \cdots& 0& 0\\
		\vdots& \vdots& \vdots& & \vdots& \vdots\\
		0& 0& 0& \cdots& 0& 1\\
	   -a_0& -a_1& -a_1& \cdots& -a_{n-2}& -a_{n-1}\\
	\end{bmatrix}
	\end{equation}
	$A$是否可以对角化?若是求它的所有特征值和特征向量, 若否说明原因.
	\begin{solution}
		特征多项式为$\lambda^{n} + a_{n-1}\lambda^{n-1} +\cdots + a_1 \lambda +a_0$, 设其$n$个根为$\lambda_i$
		注意到$\rank{\lambda_i I - A} = n-1$. 故$A$可对角化当且仅当特征多项式无重根.
	\end{solution}
	\vspace{2cm}
\end{itemize}

\section{实对称矩阵的对角化}
\begin{itemize}
\item 实对称矩阵在实数域上有特征值
\item 实对称矩阵一定正交相似于对角矩阵
\end{itemize}

\subsection*{例题}
\begin{itemize}
	\item[1.] (复Schur分解) 定义$U^H$表示矩阵的共轭转置, 若$U^H U = I$, 
	则称方阵$U$为酉矩阵. 
	证明对任意的复数方阵$A$, 存在酉矩阵$U$,使得$U^H A U$为上三角矩阵.
	\vspace{2cm}

	\item[2.] (实Schur分解) 证明,对任意的实数方阵, 
	存在实正交矩阵$U$,使得$U^T A U$为分块上三角矩阵, 形如
	\begin{equation}\nonumber
	\begin{bmatrix}
		T_{11}& T_{12}& \cdots& T_{1r}\\
		& T_{22}& \cdots& T_{2r}\\
		& &  \ddots& \vdots\\
		& &  &T_{rr}
	\end{bmatrix}
	\end{equation}
	其中$T_{ii}$为1阶或2阶矩阵, 且当是2阶矩阵,其特征值是一对共轭复根.
	\begin{solution}
		若$\lambda$为$A$的实特征值,取其实特征向量即可, 做法相同;
		若$\lambda = \omega + i\mu$为复特征值($\mu \ne 0$), 取其复特征向量
		$x = u +iv$, 则
		\begin{equation}
		\nonumber
		A[u,v] = [u, v]\begin{bmatrix}
			\omega& \mu\\
			-\mu   & \omega\\
		\end{bmatrix}
		\end{equation}
		且$u$和$v$线性无关
		(首先$v \ne 0$, 否则$i\mu u=0$;
		其次假设$u = kv$,$k$为实数,则可推出$(1+k^2)\mu = 0$矛盾),
		取QR分解,
		\begin{equation}
		\nonumber
		Q[u,v] = \begin{bmatrix}
			R\\
			0
		\end{bmatrix}
		\end{equation}
		\begin{equation}
		\nonumber
		QAQ^T\begin{bmatrix}
			R\\
			0
		\end{bmatrix}
		 = 
		\begin{bmatrix}
			R\\
			0
		\end{bmatrix}
		\begin{bmatrix}
			\omega& \mu\\
			-\mu   & \omega\\
		\end{bmatrix}
		\end{equation}
		由于$u,v$的线性无关,所以$R$非奇异,
		\begin{equation}\nonumber
			QAQ^T = 
			\begin{bmatrix}
				T& *\\
				0& *\\
			\end{bmatrix}
		\end{equation}
		其中$T = R
		\begin{bmatrix}
			\omega& \mu\\
			-\mu   & \omega\\
		\end{bmatrix}
		R^{-1}
		$.
	\end{solution}
	\vspace{2cm}
\end{itemize}

\section{二次型和矩阵合同}
\begin{itemize}
\item 数域$K$上的任一对称矩阵都合同于一个对角矩阵(合同标准形)
\item 合同标准形并不唯一; 实对称矩阵的规范形是唯一的(惯性定律); 复对称矩阵呢?
\item $n$阶实对称矩阵是正定的,当且仅当正惯性指数为$n$/特征值全大于0/所有顺序主子式大于0 
\end{itemize}

\subsection*{例题}
\begin{itemize}
	\item[1.] 设实二次型$$f(x_1, \cdots, x_n) = l_1^2 + \cdots + l_s^2 - l_{s+1}^2 - \cdots - l_{s+u}^2$$
	其中$l_i$是$x_1,\cdots,x_n$的一次齐次多项式. 证明$f$的正惯性指数$p \le s$, 负惯性指数$q \le u$.
	\begin{solution}
		仿照惯性定律的证明方式
	\end{solution}
	\vspace{2cm}

	\item[2.] (复对称矩阵的合同) 证明, 复对称矩阵$A,B$合同的充分必要条件是$\rank{A} = \rank{B}$.
	\begin{solution}
		由于$A,B$相抵, 必要性得证;充分性,只需证明$A$合同于$\mathrm{diag}\{I_r, 0\}$,
		其中$\rank{A} = r$. 首先, $A$可以通过可逆矩阵$C_1$, 合同对角化于对角矩阵$\mathrm{diag}\{d_1,\cdots, d_r,0,\cdots,0\}$,
		然后令$C_2 = \mathrm{diag}\{\sqrt{d_1}, \cdots, \sqrt{d_r}, 1, \cdots, 1\}$, 
		故$A$可以通过$C_2^{-1}C_1$, 合同于$\mathrm{diag}\{I_r, 0\}$.
	\end{solution}
	\vspace{2cm}

	\item[3.] 证明, 如果$A,B$都是$n$级正定矩阵,那么$AB$是正定矩阵的充要条件是$AB=BA$.
	\begin{solution}
		必要性,由AB的对称性可知; 充分性,$AB=BA$则意味着这两个正定矩阵可以同时对角化,
		具体的,存在正交矩阵$T$,使得$A = T^T \Lambda T$, 
		其中$\Lambda = 
		\mathrm{diag}\{
			\underbrace{\lambda_{1},\cdots,\lambda_{1}}_{n_1 \text{个}},
			\underbrace{\lambda_{2},\cdots,\lambda_{2}}_{n_2 \text{个}},
			\cdots
			\underbrace{\lambda_{k},\cdots,\lambda_{k}}_{n_k \text{个}}
		\}$,
		$n_1 + n_2 + \cdots + n_k = n$,
		故$T^T \Lambda T B = B T^T \Lambda T$,
		即$\Lambda T B T^T = T B T^T \Lambda $, 故$TB T^T$
		是块对角矩阵, 形如$\mathrm{diag}\{B_1, \cdots, B_k\}$, 其中$B_i$是$n_i$级的对称方阵.
		所以存在$n_i \times n_i$的正交矩阵$T_i$, 使得
		\begin{equation}\nonumber
		T_i^T B_i T_i = \mathrm{diag}\{\ \underbrace{\lambda_i, \cdots, \lambda_i}_{n_i \text{个}}\}
		\end{equation}
		所以$S = \mathrm{diag}\{T_1, \cdots, T_k\} T$使得$A,B$同时对角化.
	\end{solution}
	\vspace{2cm}

	\item[4.](Hadamard不等式) 
	\begin{itemize}
		\item [(a)] 证明: 如果$A$是$n$级正定矩阵, $B \ne 0$是$n$级半正定矩阵, 那么
		\begin{equation}
		\nonumber
		|A+B| > \mathrm{max}\{|A|, |B|\}
		\end{equation}
		\begin{solution}
			存在可逆矩阵$C$,使得$C^TAC=I, C^TBC = D = 
			\mathrm{diag}\{\lambda_1,\cdots, \lambda_n\}, \lambda_i \ge 0$.
			$|A+B| = |C^{-T}||C^{-1}|(1+\lambda_1)\cdots(1+\lambda_n)$,
			$|A| = |C^{-T}||C^{-1}|$,
			$|B| = |C^{-T}||C^{-1}|\lambda_1 \cdots \lambda_n$,
		\end{solution}
		\vspace{1cm}

		\item[(b)] 证明: 若
		\begin{equation}
		\nonumber
		M = \begin{bmatrix}
			A& B\\
			B^T& D\\
		\end{bmatrix}
		\end{equation}
		是正定矩阵. 证明$|M| \le |A||D|$, 等号成立当且仅当$B = 0$.
		\begin{solution}
			$|M| = |A||D-B^TA^{-1}B|$ 并利用(a)的结论.
		\end{solution}
		\vspace{2cm}

		\item[(c)] 
		证明:如果$C=(c_{ij})$是$n$级实矩阵, 则
		$$|C| \le \prod_{j=1}^n \sqrt{\sum_{i=1}^n c_{ij}^2}$$
		这说明,欧氏空间中,$n$个向量张成的超多面体的体积不超过这$n$个向量的长度乘积.
		\begin{solution}
			将(b)的结论推广到任意多个对角块;考虑$C$非奇异,则$C^TC$正定, 对其使用推广的
			(b)中结论.
		\end{solution}
		\vspace{2cm}
	\end{itemize}

	\item[4.] (正定矩阵的平方根分解) 证明: 对于正定矩阵$A$, 存在唯一的正定矩阵$C$, 使得$A = C^2$.
	\begin{solution}
		存在性通过对$A$正交对角化易得.
		唯一性, 假设
		$$C_1 = T_1^T \mathrm{diag}\{\lambda_1, \cdots, \lambda_n\} T_1$$
		$$C_2 = T_2^T \mathrm{diag}\{\mu_1, \cdots, \mu_n\} T_2$$
		由于$C_1^2 = C_2^2$, 这两个对角阵相似, 所以$\mu_i = \lambda_i$.
		({\color{red}可能有$T_1 \ne T_2$}),
		由于
		$
		T_1^T \mathrm{diag}\{\lambda_1^2, \cdots, \lambda_n^2\} T_1 = 
		T_2^T \mathrm{diag}\{\lambda_1^2, \cdots, \lambda_n^2\} T_2
		$,
		所以
		$
		T_2 T_1^T \mathrm{diag}\{\lambda_1^2, \cdots, \lambda_n^2\} T_1 = 
		\mathrm{diag}\{\lambda_1^2, \cdots, \lambda_n^2\} T_2 T_1^T
		$,
		$T_2 T_1^T$块对角, 故有
		$$
		T_2 T_1^T \mathrm{diag}\{\lambda_1, \cdots, \lambda_n\} T_1 = 
		\mathrm{diag}\{\lambda_1, \cdots, \lambda_n\} T_2 T_1^T
		$$
		证毕.


	\end{solution}
	\vspace{2cm}

	\item[5.] (极坐标分解) 证明任意实可逆矩阵$A$, 可以被唯一分解为$$A = TS_1 = S_2 T,$$
	其中$T$为正交矩阵, $S_1, S_2$为正定矩阵.
	\begin{solution}
		存在性:$A = A(A^TA)^{-\frac{1}{2}}(A^TA)^{\frac{1}{2}}$, 其中$A(A^TA)^{-\frac{1}{2}}$为正交阵,
		唯一性:由$A^TA$平方根的唯一性
	\end{solution}
	\vspace{2cm}

	\item[6.] (SVD分解) 
	\begin{itemize}
		\item [(a)] 证明任意的实可逆矩阵$A$, 都存在正交矩阵$T_1,T_2$, 使得
		$$A = T_1 \mathrm{diag}\{\lambda_1, \cdots, \lambda_n\} T_2,$$
		其中$\lambda_i$是$A^TA$特征值的平方根. 这个分解具有唯一性么?
		\begin{solution}
		$A = A(A^TA)^{-\frac{1}{2}}(A^TA)^{\frac{1}{2}}$, 再对$(A^TA)^{\frac{1}{2}}$做谱分解即可.
		考虑$A$正定时, $A$的正交对角化是SVD分解,正交对角化的特征向量的选择不唯一,故SVD也不唯一.
		\end{solution}
		\vspace{2cm}

		\item[(b)] $A$不可逆的时候呢?
		\begin{solution}
		同理考虑$A^TA$的谱分解,并做矩阵分块
		\begin{equation}\nonumber
			\begin{bmatrix}
				V_1^T \\
				V_2^T \\
			\end{bmatrix}
			A^TA
			\begin{bmatrix}
				V_1& V_2 \\
			\end{bmatrix}
			=
			\begin{bmatrix}
				D& 0\\
				0& 0\\
			\end{bmatrix}
		\end{equation}
		所以$V_2^T A^TA V_2 = 0$, 注意到$\mathrm{tr}(V_2^TA^TAV_2) = ||AV_2||_F 
		= 0$, 所以$AV_2 = 0$.
		取$U_1 = MV_1D^{-\frac{1}{2}}$,
		$U_1^T U = D^{-\frac{1}{2}}V_1^TM^T M V_1 D^{-\frac{1}{2}} 
		= D^{-\frac{1}{2}} D D^{-\frac{1}{2}} = I$,
		且$U_1D^{\frac{1}{2}}V_1^T = MV_1V_1^T = M(I-V_2V_2^T) = M$.
		所以将$U_1$扩充为一组基, 得到对应的$U_2$, 有
		\begin{equation}\nonumber
			M = U_1D^{\frac{1}{2}}V_1^T = 
			\begin{bmatrix}
				U_1& U_2
			\end{bmatrix}
			\begin{bmatrix}
				D^{\frac{1}{2}} & 0\\
				              0 & 0\\
			\end{bmatrix}
			\begin{bmatrix}
				V_1^T\\
				V_2^T\\
			\end{bmatrix}
		\end{equation}
		\end{solution}
		\vspace{2cm}
	\end{itemize}
\end{itemize}