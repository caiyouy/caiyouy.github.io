\chapter{矩阵}
\section{矩阵的运算和特殊矩阵}
\begin{itemize}
\item 相同行列数目的矩阵全体在加法和数乘下构成线性空间
\item 矩阵的乘法(线性映射的复合,有限维线性映射的典型)
\end{itemize}

\subsection*{例题}
\begin{itemize}
	\item[1.] (矩阵的幂次)
	\begin{itemize}
		\item [(a)]
		\begin{equation}
		\nonumber
		A = \begin{bmatrix}
			0& 1& 0& \cdots& 0\\
			0& 0& 1& \cdots& 0\\
			\vdots& \vdots& \vdots& & \vdots\\
			0& 0& 0& \cdots& 1\\
			0& 0& 0& \cdots& 0\\
		\end{bmatrix}_{n\times n}
		\end{equation}
		计算$A^m$, 其中$m$为正整数.
		\vspace{2cm}

		\item [(b)]
		\begin{equation}
		\nonumber
		J = \begin{bmatrix}
			x& 1& 0& \cdots& 0\\
			0& x& 1& \cdots& 0\\
			\vdots& \vdots& \vdots& & \vdots\\
			0& 0& 0& \cdots& 1\\
			0& 0& 0& \cdots& x\\
		\end{bmatrix}_{n\times n}
		\end{equation}
		计算$J^m$, 其中$m$为正整数.
		\vspace{2cm}
	\end{itemize}
	
	\item[2.] 证明:反对称矩阵的秩为偶数
	\begin{solution}
	取行向量组的极大线性无关组,记为$i_1, i_2, \cdots, i_r$行, 考虑这些行列组成的子矩阵,
	行列式不为零, 且反对称,故$r$为偶数.
	\end{solution}
	\vspace{1cm}
\end{itemize}
	

\section{矩阵相乘的秩与行列式}
\begin{itemize}
\item $\rank{AB} \le \min\{\rank{A}, \rank{B}\}$
\item 实数域上,$\rank{A^TA} = \rank{A}$
\item Binet-Cauchy公式
\end{itemize}

\subsection*{例题}
\begin{itemize}
	\item[1.] 举例说明,对于复矩阵$A$, 有可能$\rank{A^TA} \ne \rank A$
	\vspace{2cm}

	\item[2.] 证明,对于实数域上的任意$s\times n$矩阵$A$, 都有$\rank{AA^TA} = \rank A$
	\vspace{1.5cm}

	\item[3.] 设$A,B$分别是数域$K$上的$s\times n, n\times m$矩阵,
	证明:如果$\rank{AB} = \rank B$, 那么对于任意的数域$K$上的$m\times r$的矩阵
	$C$, 都有 
	$$\rank{ABC} = \rank{BC}.$$
	\begin{solution}
	注意到$ABx=0$和$Bx=0$同解, 证明$ABCx=0$和$BCx=0$同解即可.
	\end{solution}
	\vspace{2cm}

	\item[4.] 计算如下矩阵的行列式
	\begin{equation}
	\nonumber
	A = \begin{bmatrix}
		\cos(\alpha_1 - \beta_1)& \cos(\alpha_1 - \beta_2)& \cdots& \cos(\alpha_1-\beta_n)\\
		\cos(\alpha_2 - \beta_1)& \cos(\alpha_2 - \beta_2)& \cdots& \cos(\alpha_2-\beta_n)\\
		\vdots& \vdots& & \vdots\\
		\cos(\alpha_n - \beta_1)& \cos(\alpha_n - \beta_2)& \cdots& \cos(\alpha_n-\beta_n)\\
	\end{bmatrix}
	\end{equation}
	\begin{solution}
	$\cos(\alpha_i - \beta_j) = \cos(\alpha_i)\cos(\beta_j) + \sin(\alpha_i)\sin(\beta_j)$,
	然后使用Binet-Cauchy公式; $n\ge 2$时为0, 其他时候计算可得.
	\end{solution}
	\vspace{2cm}

\end{itemize}

\section{可逆矩阵}
\begin{itemize}
\item $\det A \ne 0$ 是方阵$A$可逆的充分必要条件
\item 可逆矩阵可以表示为若干初等矩阵的乘积
\end{itemize}

\subsection*{例题}
\begin{itemize}
	\item[1.] (不可约对角占有矩阵)求如下矩阵的逆矩阵
	\begin{equation} \nonumber
	A = \begin{bmatrix}
		2& -1& & & & \\
		-1& 2&-1&& & \\
		& -1& 2& -1& &\\
		&& \ddots& \ddots& \ddots\\
		&&& -1& 2& -1\\
		&&&& -1& 2\\
	\end{bmatrix}_{n \times n}
	\end{equation}
	\begin{solution}
		计算伴随矩阵即可;注意到,伴随矩阵对称,而且当$i\ge j$时,
		$A_{ij}= S_{n-i}S_{j-1} = (n-i+1)j$, 其中$S_{k} = k+1$表示$k$阶
		和$A$同类型矩阵的行列式.
	\end{solution}
	\vspace{2cm}

	\item[2.] ($AB$和$BA$的非零谱相同) $A$,$B$分别为 $n\times m$和$m\times n$
	的矩阵
	\begin{itemize}
		\item [(a).] 证明, 若$I_n - AB$可逆,则$I_m - BA$也可逆.
		\begin{solution}
			验证$(I_m - BA)^{-1} = I_m + B(I_n-AB)^{-1}A$
		\end{solution}
		\vspace{2cm}

		\item [(b).] 证明  $\lambda^{m}\det(\lambda I_n - AB) = 
		                   \lambda^{n}\det(\lambda I_m - BA)$
		\begin{solution}
			考虑分块矩阵
			\begin{equation} \nonumber
			\begin{bmatrix}
				\lambda I_n & A\\
				B& I_m\\
			\end{bmatrix}
			\end{equation}
			使用初等行/列变换,计算其行列式
		\end{solution}
		\vspace{2cm}
	\end{itemize}
\end{itemize}

\section{分块矩阵}
\subsection*{例题}
\begin{itemize}
	\item[1.] (矩阵的LU分解)证明:如果$A$的所有顺序主子式不等于零,则存在可逆的下三角矩阵$L$, 使得
	$LA$是上三角矩阵
	\begin{solution}
		考虑Gauss消去,使用初等变换将矩阵化为阶梯型的过程.
		使用数学归纳法, 和分块矩阵的记号说明.
	\end{solution}
	\vspace{2cm}
\end{itemize}

\section{正交矩阵和欧氏空间}
\begin{itemize}
\item 正交矩阵的定义和性质(实数域)
\item 欧氏空间的内积(对称正定的双线性函数)
\item Schmidt正交化
\end{itemize}

\subsection*{例题}
\begin{itemize}
	\item[1.] (矩阵的QR分解)
	证明:可逆矩阵$A$,可以唯一分解为正交矩阵$Q$与主对角元都是正数的上三角矩阵$R$的乘积
	\begin{solution}
		存在性可以通过Householder变换,Givens变换或者Schimidt正交化来说明;
		唯一性,注意到上三角的正交矩阵为主对角元为$\pm 1$的对角矩阵;
	\end{solution}
	\vspace{2cm}

	\item[2.] $A$是$s\times n$的实矩阵, 说明$A^T$的像空间和$A$的核空间正交, 
	即在两空间中各自任意取一个向量,内积均为零
	\vspace{2cm}

	\item[3.] (最小二乘) $A$是$m\times n$的实矩阵, $m > n$, $b \in \mathbb{R}^n$,
	如果存在$x_0$使得, 对于任意的$x \in \mathbb{R}^n$,
	$$|Ax_0-b|^2 \le |Ax-b|^2$$
	则称$x_0$是$Ax=b$的最小二乘解. 证明$x_0$是最小二乘解,当且仅当$x_0$是如下线性方程组的解
	$$A^TAx=A^Tb$$
	\begin{solution}
	转化为无约束二次优化问题,求梯度可得必要性,充分性可以求Hessian,利用函数的凸性;
	或者利用扰动说明充分性, 
	$$|Ax_0-b|^2 \le |A(x_0+ty)-b|^2$$
	即 $t^2(y^TA^TAy) + 2ty^T(A^TAx_0 - A^Tb) \ge 0.$
	\end{solution}
	\vspace{2cm}
\end{itemize}

\section{线性映射}
\begin{itemize}
\item $\dim(\ker A) + \dim(\mathrm{Im} A) = \dim(K^n)$
\end{itemize}

\subsection*{例题}
\begin{itemize}
	\item[1.] 设$S$是一个有限集合, 映射$f:S \rightarrow S$,
	证明: $f$为单射和$f$为满射互相等价.
	\begin{solution}
		$|f(S)| = |S|$
	\end{solution}
	\vspace{1cm}

	\item[2.] 上一题结论在$S$不是有限集合的时候成立么?若成立,请证明;不成立给出反例
	\vspace{1cm}

	\item[3.] 设$V$是一个有限维线性空间, 线性映射$L:V \rightarrow V$,
	证明: $L$为单射和$L$为满射互相等价.
	\begin{solution}
		映射的线性,保证了$L(V)$也是一个线性空间.
		考虑$V$的一组基$\alpha_1,\dots,\alpha_n$, 以及$L\alpha_1, \dots, L\alpha_n$,
		\begin{enumerate}
		\item [1.] 若$L$为单射, 则$L\alpha_i$也有$n$个, 其线性无关. 所以$\dim L(V) = n, L(V) = V$, 所以是满射. 
		\item [2.] 若$L$为满射, $\dim L(V) = n$, 注意到$L\alpha_1, \dots, L\alpha_n$张成$L(V)$,
		$n \le \rank{\{ L\alpha_1,\dots,L\alpha_n\}} \le n$, 
		故$L\alpha_i$线性无关,故$L$为单射
		\end{enumerate}
	\end{solution}
	\vspace{2cm}

\end{itemize}