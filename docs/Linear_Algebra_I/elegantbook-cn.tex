\documentclass[cn,10pt,
               result=noanswer,
               math=cm,
            %    chinesefont=founder,
               citestyle=gb7714-2015,bibstyle=gb7714-2015]{elegantbook}

\title{线性代数A 习题课讲义集}
\author{Caiyou Yuan}
% \institute{Elegant\LaTeX{} Program}
\date{October 5, 2021}
% \version{0.1}
% \bioinfo{自定义}{信息}
% \extrainfo{各人自扫门前雪,休管他人瓦上霜。—— 陈元靓}
\setcounter{tocdepth}{3}

% \logo{logo-blue.png}
\cover{The_four_subspaces.jpg}

% 本文档命令
\usepackage{array}
\newcommand{\ccr}[1]{\makecell{{\color{#1}\rule{1cm}{1cm}}}}
\definecolor{customcolor}{RGB}{32,178,170}
\colorlet{coverlinecolor}{customcolor}

\newcommand{\rank}[1]{\mathrm{rank}(#1)}

\begin{document}

\maketitle
\frontmatter

\chapter*{特别声明}
\markboth{Introduction}{前言}
这里主要整理了作者在担任北京大学线性代数A这门课程的助教期间,
在习题课上讲解的一些题目,
主要参考了,
\begin{enumerate}
  \item 高等代数(第三版)丘维声,高等教育出版社
  \item 高等代数学习指导书(第二版)丘维生,清华大学出版社
\end{enumerate}
以及一些其他书籍。

作者能力有限,难免有考虑不周和疏漏之处,欢迎批评指正。

\begin{flushright}
Caiyou Yuan\\
October 5, 2021
\end{flushright}

\tableofcontents

\mainmatter
\paragraph{7.1}
\begin{itemize}
  \item 一元多项式的概念和运算
  \item 环的基本概念; 关于加法构成交换群,乘法具有结合律,加法乘法具有左右分配律
  \item 一元多项式环$K[x]$的通用性质
\end{itemize}

\paragraph{例题}
\begin{itemize}
  \item[1.] $R$是有单位元1($\ne 0$)的环,若对于$a\in R$, $\exists b \in R$, s.t.
$ab = ba = 1$, 则称$b$为$a$的逆. 证明$a$的逆是唯一的.
  \item[2.] 设$R$是一个环,证明: $0a = a0 = 0, \forall a\in R$;
  $\forall a, b \in R, a(-b)=-ab$.  
  \item[3.] 设$A\in M_n(C)$, 设
  $$|\lambda I - A| = (\lambda - \lambda_1)^{l_1}(\lambda - \lambda_2)^{l_2}\cdots(\lambda - \lambda_s)^{l_s},$$
  其中$\lambda_1,\cdots,\lambda_s$是两两不同的复数,$l_1 + l_2 + \cdots + l_s = n$.
  证明,对于$k \in C$, $k \ne 0$, 矩阵$kA$的特征多项式为
  $$|\lambda I - kA| = (\lambda - k\lambda_1)^{l_1}(\lambda - k\lambda_2)^{l_2}\cdots(\lambda - k\lambda_s)^{l_s},$$
  $A^3$的特征多项式为
  $$|\lambda I - A^3| = (\lambda - \lambda_1^3)^{l_1}(\lambda - \lambda_2^3)^{l_2}\cdots(\lambda - \lambda_s^3)^{l_s}.$$
\end{itemize}

\vspace{0.5cm}
\paragraph{7.2}
\begin{itemize}
  \item 整除关系
  \item 带余除法; 主要思路:对被除式的次数用数学归纳法
\end{itemize}
\paragraph{例题}
\begin{itemize}
  \item[1.] 设$d,n\in N^*$, 则$K[x]$中,$x^d-1|x^n-1 \Leftrightarrow d|n$.
\end{itemize}

\vspace{0.5cm}
\paragraph{7.3}
\begin{itemize}
  \item 最大公因式: $K[x]$中任意两个多项式都有最大公因式,且可以表示为$f,g$的和式;
  \item 互素
\end{itemize}
\paragraph{例题}
\begin{itemize}
  \item[1.] 设$f(x),g(x) \in K[x]$, $a,b,c,d\in K$而且$ad-bc\ne 0$.证明,
  $(af(x)+bg(x), cf(x)+dg(x)) = (f(x),g(x))$.
  \item[2.] 设$A\in M_n(K)$, $f(x),g(x)\in K[x]$, 证明, 如果$d(x) = (f(x), g(x))$, 那么
  $d(A)x=0$的解空间是$f(A)x=0$解空间和$g(A)x=0$解空间的交.
  \item[3.] 在$K[x]$中, 如果$(f(x),g(x))=1$, 并且$deg\,f >0, deg\,g>0$, 那么在
  $K[x]$中存在唯一的多项式$u(x), v(x)$, $deg\, u< deg\, g, \,deg\, v<deg\, f$,
  s.t.
  $$u(x)f(x) +v(x)g(x) = 1.$$
  \item[4.] 在$K[x]$中, 如果$(f(x),g(x))=d(x)$, 那么在
  $K[x]$中存在唯一的多项式$u(x)$和$v(x)$, $deg\,u < deg\,g - deg\,d,\,
  deg\,v<deg\,f - deg\,d$,
  s.t.
  $$u(x)f(x) +v(x)g(x) = d(x).$$
  \item[5.] 在$K[x]$中的两个非零多项式$f(x)$和$g(x)$不互素的充分必要条件是,存在两个非零多项式
  $u(x),v(x)$, s.t. $u(x)f(x) = v(x)g(x)$且$deg\,u <deg\,g$, $deg\,v<deg\,f$.
  \item[6.] 在$K[x]$中,$f_1(x),f_2(x),\cdots,f_s(x)$两两互素,证明对于任意的
  $r_1(x),r_2(x),\cdots, r_s(x)\in K[x]$, 同余方程组
  \begin{equation}
  \nonumber
  \left\{
    \begin{aligned}
    g(x) &\equiv r_1(x) \quad \mathrm{mod}\, f_1(x)\\
    g(x) &\equiv r_2(x) \quad \mathrm{mod}\, f_2(x)\\
    \vdots&\\
    g(x) &\equiv r_s(x) \quad \mathrm{mod}\, f_s(x)\\
    \end{aligned}
  \right.
  \end{equation}
  在$K[x]$中有解,且若$c(x),d(x)$均为解,则$c(x) \equiv d(x),\,\,\mathrm{mod}\,f_1 f_2\cdots f_s$.
\end{itemize}

\vspace{0.5cm}
\paragraph{7.4}
\begin{itemize}
  \item 不可约多项式
  \item 唯一因式分解定理;这里唯一性的意义是?
\end{itemize}
\paragraph{例题}
\begin{itemize}
  \item[1.] 在$K[x]$中,设$(f,g_i)=1,i=1,2$. 证明$(fg_1,g_2)=(g_1,g_2)$.
  \item[2.] 在$K[x]$中, 证明对于任意的正整数$m$,有$(f^m(x), g^m(x)) = (f(x),g(x))^m$.
  \item[3.] 分别在复数域、实数域和有理数域上分解 $x^4+1$为不可约多项式的乘积 
\end{itemize}
\paragraph{7.5等}
\begin{itemize}
    \item 重因式
    \item 复数域上的不可约多项式只有一次的
    \item 实数域上的不可约多项式都是一次的,或判别式小于零的二次多项式
    \item 有理数域上的不可约多项式可以是任意次数的
\end{itemize}

\paragraph{多项式理论的应用: $\lambda$-矩阵}
即矩阵的每个元素都是多项式环$K[\lambda]$中元素;矩阵乘法,加法以及行列式等概念,和数字矩阵类似

\paragraph{$\lambda$-矩阵的初等变换}
\begin{itemize}
  \item [(a)] 矩阵的两行/列互换位置
  \item [(b)] 矩阵的某一行/列乘以非零常数c
  \item [(c)] 矩阵的某一行/列加上另一行/列的$p(\lambda)$倍,其中$p(\lambda) \in K[\lambda]$.
\end{itemize}
如果$A(\lambda)$可以经过一系列行和列的初等变换化为$B(\lambda)$,
则称$A(\lambda)$和$B(\lambda)$等价.

\paragraph{例题}
\begin{itemize}
  \item[1.] 证明: 设$A(\lambda)$的左上角元素$a_{11}(\lambda) \ne 0$, 并且$A(\lambda)$中
至少有一个元素不能被它除尽,那么一定可以找到和$A(\lambda)$等价的$B(\lambda)$,
它的左上角元素也不为零,但是次数小于$a_{11}(\lambda)$的次数
  \vspace{3cm}
  \item[2.] 证明:任意一个非零的$s\times n$的$\lambda$-矩阵$A(\lambda)$都等价于如下形式的矩阵(被称为标准形式)
\begin{equation}
\nonumber
\begin{pmatrix}
  d_1(\lambda)&&&&&&\\
  &d_2(\lambda)&&&&&\\
  &&\ddots&&&&\\
  &&&d_r(\lambda)&&&\\
  &&&&0&&\\
  &&&&&\ddots&\\
  &&&&&&0\\
\end{pmatrix}
\end{equation}
其中$r\ge 1$, $d_i(\lambda)$是首一的多项式(被称为不变因子),且
$$d_i(\lambda)\,|\,d_{i+1}(\lambda)\quad i=1,2,\cdots, r-1.$$
\vspace{4cm}
\item[3.]用初等变换化$\lambda$-矩阵
\begin{equation}
\nonumber
A(\lambda) =
\begin{bmatrix}
1-\lambda& 2\lambda-1& \lambda\\
\lambda&   \lambda^2&  -\lambda\\
1+\lambda^2& \lambda^3+\lambda-1& -\lambda^2\\
\end{bmatrix}
\end{equation}
为标准型.
\vspace{3cm}
\end{itemize}

\paragraph{$\lambda$-矩阵标准形式的唯一性}
下面来借助行列式因子的概念来说明$\lambda$-矩阵标准形式的唯一性.
\begin{itemize}
  \item[1.] ($\lambda$-矩阵的秩)如果$A(\lambda)$中有一个$r\ge 1$级子式不为零,而所有的$r+1$级子式(如果有的话)全为零,
则称$A(\lambda)$的秩为$r$. 零矩阵的秩记为0.
  \item[2.] ($\lambda$-矩阵的行列式因子) 设$A(\lambda)$的秩为r,对于$1 \le k \le r$, $A(\lambda)$中全部$k$级子式的首一最大公因式
$D_{k}(\lambda)$称为$A(\lambda)$的$k$阶行列式因子.
\end{itemize}

\paragraph{例题}
\begin{itemize}
  \item[1.] 等价的$\lambda$-矩阵具有相同的秩和相同的各阶行列式因子.
  \vspace{2cm}
  \item[2.] 证明$\lambda$-矩阵的不变因子和行列式因子有如下关系
  $$
  d_1(\lambda) = D_1(\lambda),\quad
  d_2(\lambda) = \frac{D_2(\lambda)}{D_1(\lambda)},\quad
  \dots, \quad d_r = \frac{D_r(\lambda)}{D_{r-1}(\lambda)}
  $$ 
  \vspace{3cm}
  \item[3.] 说明$\lambda$-矩阵的标准形式是唯一的.
  \vspace{2cm} 
\end{itemize}

\paragraph{8.1}
\begin{itemize}
    \item 域
    \item 域$F$上的线性空间的定义
    \item 基
    \item 维数
    \item 过渡矩阵
\end{itemize}

\paragraph{例题}
\begin{itemize}
  \item[1.]
  \begin{itemize}
    \item [(a)] 把域$F$看成是$F$上的线性空间,求它的一个基和维数;
    \item [(b)] 把复数域$C$看成是实数域$R$上的线性空间,求它的一个基和维数;
  \end{itemize}
\end{itemize}
\vspace{1cm}

\begin{itemize}
  \item[2.]
  \begin{itemize}
    \item [(a)] 把实数域$R$看成是有理数域$Q$上的线性空间,证明:对于任意大于1的正整数,
  $$1, \sqrt[n]{3}, \sqrt[n]{3^2}, \cdots, \sqrt[n]{3^{n-1}}$$
  是线性无关的.\,(提示: 已知$g(x) = x^n-3$是Q上的不可约多项式)
    \item [(b)] 证明: 实数域$R$作为有理数域$Q$上的线性空间是无穷维的.
  \end{itemize}
\end{itemize}
\vspace{1.5cm}

\begin{itemize}
  \item[3.] 设$V$是域$F$上的n维线性空间,域$F$包含域$E$,\,$F$可看作域$E$
  上的$m$维线性空间
  \begin{itemize}
    \item [(a)] 求证:V可以成为域$E$上的线性空间
    \item [(b)] 证明: 求V作为域E上线性空间的维数
  \end{itemize}
\end{itemize}
\vspace{1.5cm}

\begin{itemize}
  \item[4.] (Complexification of real vector space)
  设$V$是数域R上的$n$维线性空间,设$V_C = \{(u,v), u,v\in V\}$,
  定义$V_C$上的加法
  $$(u_1,v_1) +(u_2, v_2) = (u_1+u_2, v_1+v_2)$$
  以及C上的数乘
  $$(a+bi)(u,v)=(au-bv, av+bu)$$
  \begin{itemize}
    \item [(a)] 求证:$V_C$是一个复线性空间
    \item [(b)] 计算$V_C$的维数
  \end{itemize}
\end{itemize}
\vspace{1cm}

\begin{itemize}
  \item[5.] (对偶空间)设$V$是数域K上的$n$维线性空间,
  考虑复数域$C$上的线性空间$C^V$(从$V$到$R$的函数全体)中具有下述性质的函数组成的子集$W$:
  $$f(\alpha+\beta)=f(\alpha) + f(\beta),\quad \forall \alpha, \beta \in V,$$
  $$f(k\alpha)=kf(\alpha),\quad \forall \alpha \in V, k \in K.$$
  \begin{itemize}
    \item [(a)] 求证:$W$是一个复线性空间
    \item [(b)] 求$W$的一个基和维数;设$f\in W$, 求$f$在这个基下的坐标
  \end{itemize}
\end{itemize}
\vspace{1cm}

\begin{itemize}
  \item[6.] (零化多项式和最小多项式)
  设A是数域K上的一个非零n阶矩阵,说明$K[A]$是$K$上的一个线性空间.
  $K[A]$至多多少维?\footnote{Hamilton-Cayley定理告诉我们,K[A]至多n维.}
\end{itemize}
\vspace{1cm}

\begin{itemize}
  \item[7.] 设递推方程
  $$u_n = au_{n-1} +bu_{n-2}, \quad n \ge 2,$$
  其中$a,b$都是非零复数. 若N上的一个复值数列$u_n$满足上述递推关系,则称为上述递推方程的解.
  一元多项式$f(x)=x^2 - ax -b$称为上述递推方程的特征多项式.
  求证
  \begin{itemize}
    \item [(a)] 上述递推方程的解集$W$是一个复线性空间
    \item [(b)] 设$\alpha$是一个非零复数,则$\alpha^n \in W$ 当且仅当$f(\alpha)=0$
    \item [(c)] 设$\alpha$是一个非零复数,则$n\alpha^n \in W$ 当且仅当$f(\alpha)=0, f'(\alpha)=0$.
    \item [(d)] 若$f(x)$有两不同的根$\alpha_1, \alpha_2$, 则任意$u_n \in W$, 可以表示为
    $$ u_n = C_1 \alpha_1^n +C_2 \alpha_2^n$$
    其中$C_1, C_2$是常数;
    \item [(e)] 若$f(x)$有二重根$\alpha$, 则任意$u_n \in W$, 可以表示为
    $$ u_n = C_1 \alpha^n +C_2 n\alpha^n$$
    其中$C_1, C_2$是常数;
  \end{itemize}
\end{itemize}
\vspace{6cm}


\paragraph{8.2}
\begin{itemize}
    \item 子空间
    \item 子空间的维数定理
    $$\dim V_1 + \dim V_2 = \dim (V_1 + V_2) + \dim (V_1 \cap V_2)$$
    \item 直和
\end{itemize}

\paragraph{例题}
\begin{itemize}
\item[1.] 设$V_1, V_2, V_3$都是域F上的有限维线性空间$V$的子空间
\begin{itemize}
    \item[(a)] 求证:
    $$V_3 \cap V_1 + V_3 \cap V_2 \subset V_3 \cap (V_1 + V_2)$$ 
    \item[(b)] 如果$V_3 \subset V_1 + V_2$,
试问$(V_3\cap V_1) + (V_3\cap V_2) = V_3$是否总成立?如果再加上条件$V_1 \subset V_3$呢?
    \item[(c)] 求证,
    \begin{equation}
    \nonumber
    \begin{aligned}
    \dim V_1 + \dim V_2 + \dim V_3 \ge
    &\dim(V_1 + V_2 + V_3)\\
    &+\dim(V_1 \cap V_2) +\dim(V_1 \cap V_3) +\dim(V_2 \cap V_3)\\
    &-\dim(V_1 \cap V_2 \cap V_3)
    \end{aligned}
    \end{equation}  
  \end{itemize}
\end{itemize}
\vspace{3cm}

\begin{itemize}
  \item[2.] 设$V_1, \cdots, V_s$都是域F上线性空间V的子空间,
  证明$V_1 + V_2 + \cdots + V_s$是直和
  \begin{itemize} 
    \item[(a)] 当且仅当,
    $$V_i \cap \left(\sum_{j\ne i} V_j \right) = 0, \quad i=1,2,\dots,s$$
    \item[(b)] 当且仅当,$V$中有一向量$\alpha$可以唯一表示为
    $$\alpha = \sum_{i=1}^s \alpha_i, \quad \alpha_i \in V_i$$
    \item[(c)] 当且仅当,
    $$\dim(V_1 + V_2 + \cdots + V_s) = \dim V_1 + \cdots \dim V_s$$ 
  \end{itemize}
\end{itemize}
\vspace{3cm}

\begin{itemize}
  \item[3.] 设A是数域K上的一个n阶矩阵,$\lambda_1, \lambda_2, \cdots, \lambda_s$
  是A的全部不同的特征值,用$V_{\lambda_i}$表示A的属于$\lambda_i$的特征子空间. 证明:
  A可以对角化的充分必要条件是
  $$K^n = V_{\lambda_1} \oplus V_{\lambda_2} \oplus \cdots \oplus V_{\lambda_s}$$
\end{itemize}
\vspace{1cm}

\begin{itemize}
  \item[4.] 在数域K上的线性空间$K^{M_n(K)}$中,如果$f$满足,
  对于任意的$A=(\alpha_1, \alpha_2,\cdots, \alpha_n)$,任意的n维列向量$\alpha$, 以及任意$k\in K$, $j\in \{1,2,\cdots, n\}$,
  有 
  $$f(\alpha_1, \cdots, \alpha_{j-1}, \alpha_{j} + \alpha, \alpha_{j+1}, \cdots, \alpha_n)
  = f(\alpha_1, \cdots, \alpha_{j-1}, \alpha_{j}, \alpha_{j+1}, \cdots, \alpha_n) + 
    f(\alpha_1, \cdots, \alpha_{j-1}, \alpha, \alpha_{j+1}, \cdots, \alpha_n)$$
  $$f(\alpha_1, \cdots, \alpha_{j-1}, k\alpha_{j}, \alpha_{j+1}, \cdots, \alpha_n)
  = kf(\alpha_1, \cdots, \alpha_{j-1}, \alpha_{j}, \alpha_{j+1}, \cdots, \alpha_n)
  $$
  那么称$f$是$M_n(K)$上的列线性函数. 同理如果$g(A^T)$是列线性函数,则称$g(A)$是行线性函数。
  记所有的列/行线性函数组成的集合分别记为$V_1$和$V_2$.
  \begin{itemize} 
    \item[(a)] 证明: $V_1, V_2$都是$K^{M_n(K)}$的子空间
    \item[(b)] 分别求$V_1, V_2$的一个基和维数
    \item[(c)] 分别求$V_1 \cap V_2,\,V_1 + V_2$的一个基和维数 
  \end{itemize}
\end{itemize}
\vspace{6cm}

\paragraph{8.3}
\begin{itemize}
    \item 线性空间的同构
    \item 有限维线性空间同构的充要条件
\end{itemize}

\paragraph{例题}
\begin{itemize}
  \item[1.]令
  \begin{equation}
  \nonumber
  H = \left\{
    \begin{bmatrix}
      z_1& z_2\\
     -z_2& z_1\\
    \end{bmatrix},
    z_1, z_2 \in C
      \right\}
  \end{equation}
  \begin{itemize}
    \item[(a)] H对于矩阵的加法,以及实数和矩阵的乘法构成一个实线性空间
    \item[(b)] 给出H的一个基和维数
    \item[(c)] 证明: $H$与$R^4$同构,并写出$H$到$R^4$的一个同构映射  
  \end{itemize}
\end{itemize}
\vspace{1cm}

\begin{itemize}
  \item[2.]设$A \in M_n(K)$, 令$AM_n(K) = \{AB, B\in M_n(K) \}$.
  \begin{itemize}
    \item[(a)] 证明$AM_n(K)$是数域$K$上线性空间$M_n(K)$的子空间
    \item[(b)] 设A的列向量组$\alpha_1, \cdots, \alpha_n$的一个极大线性无关组为
    $\alpha_{j_1}, \cdots, \alpha_{j_r}$,证明$AM_n(K)$和$M_{r\times n}(K)$同构,并
    写出一个同构映射. 
    \item[(c)] 证明: $\dim [AM_n(K)] = \rank(A)n$.
  \end{itemize}
\end{itemize}
\vspace{1cm}
\paragraph{8.1}
\begin{itemize}
    \item 域, 以及域$F$上的线性空间
    \item 基和维数
    \begin{itemize}
		\item[a.] 所有非零线性空间均有基
		\item[b.] 线性空间中的任意一组线性无关的向量可以扩充为基
	\end{itemize}
    \item 过渡矩阵
\end{itemize}

\paragraph{例题}
\begin{itemize}
  \item[1.]
  \begin{itemize}
    \item [(a)] 把域$F$看成是$F$上的线性空间,求它的一个基和维数;
    \item [(b)] 把复数域$C$看成是实数域$R$上的线性空间,求它的一个基和维数;
    \item [(c)] 把实数域$R$看成是有理数域$Q$上的线性空间,证明:对于任意大于1的正整数$n$,
  $$1, \sqrt[n]{3}, \sqrt[n]{3^2}, \cdots, \sqrt[n]{3^{n-1}}$$
  是线性无关的.\,(提示: 已知$g(x) = x^n-3$是Q上的不可约多项式)
    \item [(d)] 证明: 实数域$R$作为有理数域$Q$上的线性空间是无穷维的.
\end{itemize}
\end{itemize}
\vspace{2cm}

\begin{itemize}
  \item[2.] 设$V$是域$F$上的n维线性空间,域$F$包含域$E$,\,$F$可看作域$E$
  上的$m$维线性空间
  \begin{itemize}
    \item [(a)] 求证:V可以成为域$E$上的线性空间
    \item [(b)] 证明: 求V作为域E上线性空间的维数
  \end{itemize}
\end{itemize}
\vspace{1.5cm}

\begin{itemize}
  \item[3.] (Complexification of real vector space)
  设$V$是数域R上的$n$维线性空间,设$V_C = \{(u,v), u,v\in V\}$,
  定义$V_C$上的加法
  $$(u_1,v_1) +(u_2, v_2) = (u_1+u_2, v_1+v_2)$$
  以及C上的数乘
  $$(a+bi)(u,v)=(au-bv, av+bu)$$
  \begin{itemize}
    \item [(a)] 求证:$V_C$是一个复线性空间
    \item [(b)] 计算$V_C$的维数
  \end{itemize}
\end{itemize}
\vspace{1cm}

\begin{itemize}
  \item[4.] (对偶空间)设$V$是数域K上的$n$维线性空间,
  考虑复数域$C$上的线性空间$C^V$(从$V$到$R$的函数全体)中具有下述性质的函数组成的子集$W$:
  $$f(\alpha+\beta)=f(\alpha) + f(\beta),\quad \forall \alpha, \beta \in V,$$
  $$f(k\alpha)=kf(\alpha),\quad \forall \alpha \in V, k \in K.$$
  \begin{itemize}
    \item [(a)] 求证:$W$是一个复线性空间
    \item [(b)] 求$W$的一个基和维数;设$f\in W$, 求$f$在这个基下的坐标
  \end{itemize}
\end{itemize}
\vspace{2cm}

\begin{itemize}
  \item[5.] (零化多项式和最小多项式)
  设A是数域K上的一个非零n阶矩阵,说明$K[A]$是$K$上的一个线性空间.
  $K[A]$至多多少维?\footnote{Hamilton-Cayley定理告诉我们,K[A]至多n维.}
\end{itemize}
\vspace{1cm}

\begin{itemize}
  \item[6.] 设递推方程
  $$u_n = au_{n-1} +bu_{n-2}, \quad n \ge 2,$$
  其中$a,b$都是非零复数. 若N上的一个复值数列$u_n$满足上述递推关系,则称为上述递推方程的解.
  一元多项式$f(x)=x^2 - ax -b$称为上述递推方程的特征多项式.
  求证
  \begin{itemize}
    \item [(a)] 上述递推方程的解集$W$是一个复线性空间
    \item [(b)] 设$\alpha$是一个非零复数,则$\alpha^n \in W$ 当且仅当$f(\alpha)=0$
    \item [(c)] 设$\alpha$是一个非零复数,则$n\alpha^n \in W$ 当且仅当$f(\alpha)=0, f'(\alpha)=0$.
    \item [(d)] 若$f(x)$有两不同的根$\alpha_1, \alpha_2$, 则任意$u_n \in W$, 可以表示为
    $$ u_n = C_1 \alpha_1^n +C_2 \alpha_2^n$$
    其中$C_1, C_2$是常数;
    \item [(e)] 若$f(x)$有二重根$\alpha$, 则任意$u_n \in W$, 可以表示为
    $$ u_n = C_1 \alpha^n +C_2 n\alpha^n$$
    其中$C_1, C_2$是常数;
  \end{itemize}
\end{itemize}
\vspace{6cm}


\paragraph{8.2}
\begin{itemize}
    \item 子空间
    \item 子空间的维数定理
    $$\dim V_1 + \dim V_2 = \dim (V_1 + V_2) + \dim (V_1 \cap V_2)$$
    \item 直和
\end{itemize}

\paragraph{例题}
% \begin{itemize}
% \item[1.] 设$V_1, V_2, V_3$都是域F上的有限维线性空间$V$的子空间
% \begin{itemize}
%     \item[(a)] 求证:
%     $$V_3 \cap V_1 + V_3 \cap V_2 \subset V_3 \cap (V_1 + V_2)$$ 
%     \item[(b)] 如果$V_3 \subset V_1 + V_2$,
% 试问$(V_3\cap V_1) + (V_3\cap V_2) = V_3$是否总成立?如果再加上条件$V_1 \subset V_3$呢?
%     \item[(c)] 求证,
%     \begin{equation}
%     \nonumber
%     \begin{aligned}
%     \dim V_1 + \dim V_2 + \dim V_3 \ge
%     &\dim(V_1 + V_2 + V_3)\\
%     &+\dim(V_1 \cap V_2) +\dim(V_1 \cap V_3) +\dim(V_2 \cap V_3)\\
%     &-\dim(V_1 \cap V_2 \cap V_3)
%     \end{aligned}
%     \end{equation}  
%   \end{itemize}
% \end{itemize}
% \vspace{3cm}

% \begin{itemize}
%   \item[2.] 设$V_1, \cdots, V_s$都是域F上线性空间V的子空间,
%   证明$V_1 + V_2 + \cdots + V_s$是直和
%   \begin{itemize} 
%     \item[(a)] 当且仅当,
%     $$V_i \cap \left(\sum_{j\ne i} V_j \right) = 0, \quad i=1,2,\dots,s$$
%     \item[(b)] 当且仅当,$V$中有一向量$\alpha$可以唯一表示为
%     $$\alpha = \sum_{i=1}^s \alpha_i, \quad \alpha_i \in V_i$$
%     \item[(c)] 当且仅当,
%     $$\dim(V_1 + V_2 + \cdots + V_s) = \dim V_1 + \cdots \dim V_s$$ 
%   \end{itemize}
% \end{itemize}
% \vspace{3cm}

% \begin{itemize}
%   \item[3.] 设A是数域K上的一个n阶矩阵,$\lambda_1, \lambda_2, \cdots, \lambda_s$
%   是A的全部不同的特征值,用$V_{\lambda_i}$表示A的属于$\lambda_i$的特征子空间. 证明:
%   A可以对角化的充分必要条件是
%   $$K^n = V_{\lambda_1} \oplus V_{\lambda_2} \oplus \cdots \oplus V_{\lambda_s}$$
% \end{itemize}
% \vspace{1cm}

\begin{itemize}
  \item[1.] 设$V_1, V_2, \cdots, V_s$是域$F$上线性空间V的$s$个真子空间,证明:如果$char F \ne 0$,
  $V_1\cup V_2\cup \cdots \cup V_s \ne V$.
  \vspace{2cm}

  \item[2.] 在数域K上的线性空间$K^{M_n(K)}$中,如果$f$满足,
  对于任意的$A=(\alpha_1, \alpha_2,\cdots, \alpha_n)$,任意的n维列向量$\alpha$, 以及任意$k\in K$, $j\in \{1,2,\cdots, n\}$,
  有 
  $$f(\alpha_1, \cdots, \alpha_{j-1}, \alpha_{j} + \alpha, \alpha_{j+1}, \cdots, \alpha_n)
  = f(\alpha_1, \cdots, \alpha_{j-1}, \alpha_{j}, \alpha_{j+1}, \cdots, \alpha_n) + 
    f(\alpha_1, \cdots, \alpha_{j-1}, \alpha, \alpha_{j+1}, \cdots, \alpha_n)$$
  $$f(\alpha_1, \cdots, \alpha_{j-1}, k\alpha_{j}, \alpha_{j+1}, \cdots, \alpha_n)
  = kf(\alpha_1, \cdots, \alpha_{j-1}, \alpha_{j}, \alpha_{j+1}, \cdots, \alpha_n)
  $$
  那么称$f$是$M_n(K)$上的列线性函数. 同理如果$g(A^T)$是列线性函数,则称$g(A)$是行线性函数。
  记所有的列/行线性函数组成的集合分别记为$V_1$和$V_2$.
  \begin{itemize} 
    \item[(a)] 证明: $V_1, V_2$都是$K^{M_n(K)}$的子空间
    \item[(b)] 分别求$V_1, V_2$的一个基和维数
    \item[(c)] 分别求$V_1 \cap V_2,\,V_1 + V_2$的一个基和维数 
  \end{itemize}
\end{itemize}
\vspace{6cm}

\paragraph{8.3}
\begin{itemize}
    \item 线性空间的同构
    \item 有限维线性空间同构的充要条件
\end{itemize}

\paragraph{例题}
\begin{itemize}
  \item[1.]令
  \begin{equation}
  \nonumber
  H = \left\{
    \begin{bmatrix}
      z_1& z_2\\
     -\bar{z}_2& \bar{z}_1\\
    \end{bmatrix},
    z_1, z_2 \in C
      \right\}
  \end{equation}
  \begin{itemize}
    \item[(a)] H对于矩阵的加法,以及实数和矩阵的乘法构成一个实线性空间
    \item[(b)] 给出H的一个基和维数
    \item[(c)] 证明: $H$与$R^4$同构,并写出$H$到$R^4$的一个同构映射  
  \end{itemize}
\end{itemize}
\vspace{2cm}

\begin{itemize}
  \item[2.]设$A \in M_n(K)$, 令$AM_n(K) = \{AB, B\in M_n(K) \}$.
  \begin{itemize}
    \item[(a)] 证明$AM_n(K)$是数域$K$上线性空间$M_n(K)$的子空间
    \item[(b)] 设A的列向量组$\alpha_1, \cdots, \alpha_n$的一个极大线性无关组为
    $\alpha_{j_1}, \cdots, \alpha_{j_r}$,证明$AM_n(K)$和$M_{r\times n}(K)$同构,并
    写出一个同构映射. 
    \item[(c)] 证明: $\dim [AM_n(K)] = \rank(A)n$.
  \end{itemize}
\end{itemize}
\vspace{4cm}

\paragraph{8.4}
\begin{itemize}
    \item 商空间
    \item $\dim{(V/W)} = \dim V - \dim W$
\end{itemize}

\paragraph{例题}
\begin{itemize}
    \item [1.] 设$U,W$都是域F上线性空间$V$的子空间, 证明$(U+W)/W \cong U/(U\cap W)$
    \vspace{2cm}
    % \item [2.] 设$V$是域F上的一个$n$维线性空间, ($n\ge 3$). $U$是$V$的一个2维子空间,用
    % $\Omega_1$表示V中包含U的所有$n-1$维子空间组成的集合,用$\Omega_2$表示商空间$V/U$的所有
    % $n-3$维子空间组成的集合,令
    % \begin{equation}
    %     \nonumber
    %     \begin{aligned}
    %     \sigma:\quad &\Omega_1 \longrightarrow \Omega_2\\
    %                  & W \longrightarrow W/U.
    %     \end{aligned}
    % \end{equation}
    % 证明: $\sigma$是双射.
    % \vspace{3cm}
\end{itemize}

\paragraph{9.1-4}
\begin{itemize}
    \item 线性映射,线性变换,线性函数
    \item 线性映射的核与像:
    \begin{itemize}
        \item[1.] $\dim{(\Ker A)} + \dim{(\Img A)} = \dim{V}$
        \item[2.] 有限维线性变换, 单等价于满
    \end{itemize}
    \item 线性映射的矩阵表示:
    \begin{itemize}
        \item[1.] $Hom(V,V') \cong M_{s \times n}(F)$
        \item[2.] 相似矩阵 $\Longleftrightarrow$ 线性变换在不同基下的表示 
    \end{itemize}
    \item 线性映射的行列式,秩,迹,特征值,特征向量等
\end{itemize}

\paragraph{例题}
\begin{itemize}
	\item [1.] 设$A \in Hom(V,V)$, 证明对于任意的$k$, 
	$$\rank{A^k} - \rank{A^{k+1}} \ge \rank{A^{k+1}} - \rank{A^{k+2}}$$
	\vspace{3cm}

	\item [2.] 设$f$是$M_n(K)$上的线性函数,且对于任意的$A,B \in M_n(K),
	f(AB)=f(BA)$, 求证$f=c \mathrm{tr}$, 其中$c$是某一常数,
	$\mathrm{tr}$是迹算子.
	\vspace{2cm}

	\item [3.] (Frobenius秩不等式) 设$V,U,W,M$都是域F上的线性空间,
    并且$V,U$都是有限维的,设$A \in Hom(V,U), B \in Hom(U,W), C \in Hom(W,M)$.
    证明,$$\rank(CBA) \ge \rank(CB) + \rank(BA) - \rank(B).$$
    \vspace{2cm}

    \item [4.] (幂零矩阵的矩阵表示)设V是域$F$上的n维线性空间,A是V上的一个线性变换. 如果
    $A^{n-1} \ne 0, A^n = 0$, 那么在V中存在一个基,使得A在此基下的矩阵为
    \begin{equation}
    \nonumber
    \begin{bmatrix}
        0& 1& 0& \cdots &0\\
        0& 0& 1& \cdots &0\\
        \vdots& \vdots& \vdots& \cdots & 1\\
        0& 0& 0& \cdots &0\\
    \end{bmatrix}
    \end{equation}
    \vspace{3cm}

    \item [5.] 设$V$和$V'$分别是域$F$上n维,s维线性空间,A是$V$到$V'$
    的一个线性映射,证明存在$V$的一个基和$V'$的一个基,使得A在这对基下的矩阵为,
    \begin{equation}
    \nonumber
    \begin{bmatrix}
        I_r& 0\\
        0&   0\\
    \end{bmatrix}
    \end{equation}
    其中$r = \rank(A)$.
    \vspace{3cm}

    \item [6.](两两正交的幂等变换的充要条件)
    设$A_i \in M_n(K)$, $i=1,2,\cdots,s$,其中$K$是数域.
    令$A=A_1 + A_2 +\cdots + A_s$.
    \begin{itemize}
        \item[(1) ] 证明: 如果
              $$ \rank(A) = \rank(A_1) + \rank(A_2) \cdots + \rank(A_s),$$
              那么
              $$AM_n(K) = A_1M_n(K) \oplus A_2M_n(K) \oplus \cdots \oplus A_s M_n(K).$$
        \item [(2) ] 证明: $A_1, A_2, \cdots, A_n$是两两正交的幂等矩阵,
              当且仅当$A$是幂等矩阵,并且
              $$ \rank(A) = \rank(A_1) + \rank(A_2) \cdots + \rank(A_s).$$
    \end{itemize}
    \vspace{5cm}

    % \item [7. ] (对角化和数域相关) 设$f(x)=x^n +a_{n-1}x^{n-1} +\cdots +a_1 x +a_0$ 是有理数域Q上的一个不可约
    % 多项式,$n>1$,$\omega$是$f(x)$的一个复根. 把$C$看成是$Q$上的线性空间,令
    % \begin{equation}
    %     \nonumber
    %     \begin{aligned}
    %     \mathbf{B}:\quad &Q[x] \longrightarrow C\\
    %                      &g(x) \longrightarrow g(\omega).
    %     \end{aligned}
    % \end{equation}
    % \begin{itemize}
    %     \item[(1) ] $\mathbf{B}$是不是一线性映射?若是,给出$\Img \mathbf{B}$的一个基和维数;
    %     \item[(2) ] 令$\mathbf{A}(z) = \omega z, \forall z \in \Img \mathbf{B}$, 
    %                 $\mathbf{A}$是不是$\Img \mathbf{B}$上的线性变换?如果是,求$\mathbf{A}$在(1)中基下的矩阵A;
    %     \item[(3) ] $\mathbf{A}$是否可以对角化?
    %     \item[(4) ] 把矩阵$A$看成是复数域上的矩阵,该矩阵可以对角化么?
    % \end{itemize}
    % \vspace{5cm}
\end{itemize}

\paragraph{8.5}
\begin{itemize}
    \item 不变子空间
    \begin{itemize}
        \item[1.] A有非平凡子空间 $\Longleftrightarrow$ V中存在一个基,A在此基下的矩阵为分块上三角
        \item[2.] A能分解成一些非平凡不变子空间的直和 $\Longleftrightarrow$ V在某个基下的矩阵为分块对角
    \end{itemize}
    \item 若$f(x)=f_1(x)f_2(x), (f_1(x),f_2(x))=1$, 则
    $$\Ker f(A) = \Ker f_1(A) \oplus \Ker f_2(A)$$
\end{itemize}

\paragraph{例题}
\begin{itemize}
    \item [1.] 设$\mathbf{A}$是域$F$上$n$维线性空间$V$上的线性变换,$W$是$\mathbf{A}$的一个非平凡不变子空间,
    在$W$中取一个基$\alpha_1, \cdots, \alpha_r$, 
    把它扩充成$V$的一个基$\alpha_1, \cdots, \alpha_n$,
    则$\mathbf{A}$在$V$的矩阵为
    \begin{equation}
    \nonumber
    \begin{pmatrix}
        A_1& A_3\\
        0  & A_2\\
    \end{pmatrix}
    \end{equation}
    \begin{itemize}
        \item [(a) ] 证明$A_2$是商变换
        \begin{equation}
        \nonumber
        \begin{aligned}
        \mathbf{\hat A}: \quad &V/W \longrightarrow V/W\\
                         &\alpha + W \longrightarrow \mathbf{A}\alpha + W.
        \end{aligned}
        \end{equation}
        在基
        $\alpha_{r+1} + W, \alpha_{r+2}+W, \cdots, \alpha_n+W$下的矩阵.
        \vspace{2cm}
        \item [(b) ] 设$\mathbf{A}, \mathbf{A}|_W, \mathbf{\hat A}$的特征多项式是
        $f(\lambda),f_1(\lambda), f_2(\lambda)$, 证明
        $$f(\lambda) = f_1(\lambda)f_2(\lambda)$$
        \vspace{2cm}
    \end{itemize}

    \item [2.] (同时对角化的充分必要条件) 设$A_1, \cdots, A_m$是域$F$上线性空间$V$上的线性变换,每一个都可以对角化,
    证明: 他们可以在某一组基下同时对角化的充分必要条件是他们的乘积可交换,即$A_i A_j = A_j A_i$.
    \vspace{4cm}

    \item [3.] 设$A$是域$F$上线性空间$V$的一个线性变换,设$f(x),g(x) \in F[x]$,
    $(f(x),g(x))=d(x), \left[f(x),g(x)\right]=m(x)$.
    \begin{itemize}
        \item [(a) ] 证明,
        $$\Ker d(A) = \Ker f(A) \cap \Ker g(A)$$
        $$\Ker m(A) = \Ker f(A) + \Ker g(A)$$
        \item [(b) ] 证明,
        $$\rank(f(A)) +\rank(g(A)) = \rank(d(A)) + \rank(m(A))$$
    \end{itemize}
    \vspace{3cm}
\end{itemize}


\paragraph{8.6}
\begin{itemize}
    \item Hamilton-Cayley定理: 对于线性变换$A$的特征多项式$f(x)$,有$f(A)=0$
    \item 线性变换的最小多项式
    \begin{itemize}
        \item [1. ] 如果$V$能分解成一些非平凡不变子空间的直和,
        $$V = W_1 \oplus W_2 \cdots \oplus W_s$$
        则$A$的最小多项式$m(\lambda) = [m_1(\lambda), m_2(\lambda), \cdots, m_s(\lambda)]$,其中
        $m_j(\lambda)$是$W_j$上变换$A|_{W_{j}}$的最小多项式.
        \item [2. ] $A$可以对角化的充分必要条件是,$A$的最小多项式$m(\lambda)$可以在$F[\lambda]$
        中分解成不同的一次因式的乘积.
    \end{itemize}
\end{itemize}

\paragraph{例题}
\begin{itemize}
    \item [1.] 设$A,B$分别是$n,m$阶复矩阵. 证明:矩阵方程$AX-XB=0$只有零解的充分必要条件是$A$和$B$没有公共特征值.
    \vspace{3cm}
    \item [2.] Hamilton-Cayley定理对于实数域上的线性变换成立么?
    \vspace{3cm}
    \item [3.] 设$A$是域F上的$n$维线性空间$V$上的线性变换,A有$s$个不同的特征值,可以对角化,
    属于特征值$\lambda_i$的特征子空间的维数是$n_i$.
    \begin{itemize}
        \item [(a) ] 证明,
        $$\dim C(A) = \sum_{i} n_i^2$$
        \item [(b) ] 说明若$s<n$,
        $$F[A] \subsetneq C(A)$$
        若$s=n$,
        $$F[A] = C(A)$$
    \end{itemize}
\end{itemize}



\chapter{矩阵相抵和相似}

\section{矩阵相抵}
\begin{itemize}
\item 如果矩阵$A$可以通过初等行/列变换为矩阵$B$, 则称$A,B$相抵
\item 矩阵相抵是$M_{m\times n}(K)$上的一个等价关系
\item $M_{m\times n}(K)$中的两矩阵相抵当且秩相等
\end{itemize}

\subsection*{例题}
\begin{itemize}
	\item[1.] 设$A,B,C$分别是数域$K$上$s\times n, p\times m, s\times m$矩阵,
	证明矩阵方程$AX-YB=C$有解的充分必要条件是
	\begin{equation}
	\nonumber
	\mathrm{rank}
	\begin{bmatrix}
		A&0\\
		0&B
	\end{bmatrix}
	=
	\mathrm{rank}
	\begin{bmatrix}
		A&C\\
		0&B
	\end{bmatrix}
	\end{equation}
	\begin{solution}
		必要性将$C=AX-YB$代入即可; 充分性,
		假设$\mathrm{rank}(A)=a, \mathrm{rank}(B)=b$,
		则存在可逆矩阵$P_a, Q_a, P_b, Q_b$, 使得
		\begin{equation*}
			P_a A Q_a = \begin{bmatrix}
				I_a&0\\
				0&0\\
			\end{bmatrix},\quad
			P_b B Q_b = \begin{bmatrix}
				I_b&0\\
				0&0\\
			\end{bmatrix}
		\end{equation*}
		所以
		\begin{equation*}
		\begin{bmatrix}
			P_a&\\
			&P_b\\
		\end{bmatrix}
		\begin{bmatrix}
			A&C\\
			0&B\\
		\end{bmatrix}
		\begin{bmatrix}
			Q_a&\\
			&Q_b\\
		\end{bmatrix}
		=
		\begin{bmatrix}
			I_a&0& C_{11}& C_{12}\\
			0&0& C_{21}& C_{22}\\
			0&0& I_b& 0\\
			0&0& 0& 0\\
		\end{bmatrix}
		\end{equation*}
		其中
		\begin{equation*}
		P_a C Q_b = \begin{bmatrix}
			C_{11}&C_{12}\\
			C_{21}&C_{22}\\
		\end{bmatrix}
		\end{equation*}
		由已知条件, $C_{22}=0$. 所以
		\begin{equation*}
		\begin{bmatrix}
			I& &-C_{11}& \\
			&I &-C_{21}&\\
			& &I&\\
			& &&I\\
		\end{bmatrix}
		\begin{bmatrix}
			P_a&\\
			&P_b\\
		\end{bmatrix}
		\begin{bmatrix}
			A&C\\
			0&B\\
		\end{bmatrix}
		\begin{bmatrix}
			Q_a&\\
			&Q_b\\
		\end{bmatrix}
		\begin{bmatrix}
			I& & &-C_{12}\\
			&I & &\\
			& &I&\\
			& &&I\\
		\end{bmatrix}
		=
		\begin{bmatrix}
			I_a&0&0&0\\
			0&0&0&0\\
			0&0& I_b& 0\\
			0&0& 0& 0\\
		\end{bmatrix}
		\end{equation*}
		注意右上角的分块计算结果为0, 整理即可获得原矩阵方程的解.
	\end{solution}
	\vspace{5cm}
\end{itemize}


\nocite{*} 
\printbibliography

\appendix

\end{document}
