\section{具有度量的线性空间}
\subsection{双线性函数}
\begin{itemize}
    \item 双线性函数, 及其在基下的度量矩阵: 不同基下的度量矩阵是合同的,
    以及合同的矩阵可以看作是同一双线性函数在不同基下的度量矩阵;
    \item 非退化的双线性函数 $\Longleftrightarrow$ 度量矩阵非奇异
    \item 对称/反对称的双线性函数, 及其标准形式
    \begin{itemize}
        \item[1.] 特征不为2的域上,对称矩阵可以合同于对角矩阵
        \item[2.] 特征不为2的域上,反对称矩阵合同于
        \begin{equation}
        \begin{pmatrix}
        0&1\\
        -1&0\\
        &&\ddots\\
        &&&0&1\\
        &&&-1&0\\
        &&&&&0\\
        &&&&&&\ddots\\
        &&&&&&&0\\
        \end{pmatrix}
        \label{equ:matrix_asym}
        \end{equation}
        \item[3.] 特征为2的域$F$上, 因为$\forall a\in F, -a=a$, 
        所以对称矩阵和反对称矩阵相同.
        \begin{itemize}
            \item 若$\exists \alpha \in F^n$,使得$\alpha^T A \alpha \ne 0$,
                  则$A$可以合同于一个对角矩阵;
            \item 若$\forall \alpha \in F^n$,使得$\alpha^T A \alpha = 0$,
                  则$A$可以合同于矩阵\ref{equ:matrix_asym}.
        \end{itemize}
    \end{itemize}
\end{itemize}

\subsection{对称双线性函数和二次型的关系}
设$V$是域$F$上的线性空间, 映射$q: V\rightarrow F$定义为$V$上的二次函数,
如果存在$V$上的对称双线性函数$f$, 使得
$\forall \alpha \in V,q(\alpha) = f(\alpha, \alpha)$.
若$\mathrm{char}\,F \ne 2$,
则二次函数和对称双线性函数一一对应, 因为注意到,
\begin{equation}
\nonumber
f(\alpha, \beta) = 
\frac{1}{2}[q(\alpha+\beta) - q(\alpha) - q(\beta)].
\end{equation}

\subsection{双线性函数空间}
$V$上的所有双线性函数, 关于函数的加法和数乘,构成域$F$上的线性空间, 
记为$T_2(V)$.
易见$T_2(V) \cong M_n(F) \cong Hom(V,V)$.

\paragraph{例题}
取$V^*$中的对偶基为$\{f_i\}_{i=1}^n$,
证明$f_i \otimes f_j$是$T_2(V)$的一组基,
其中$(f \otimes g) \in T_2(V)$, 
定义为$(f \otimes g)(\alpha, \beta) = f(\alpha)g(\beta)$.


\subsection{例题}
设$f$是域F上n维线性空间V上的一个双线性函数,
\begin{itemize}
    \item[1.] 
    \begin{itemize}
        \item [(a)] 映射$L_f: \alpha \rightarrow \alpha_L$是$V$到$V^*$的线性映射;

        \begin{solution}
        $a_L(\beta) = f(\alpha, \beta)$ 
        \end{solution}

        \item [(b)] f是非退化的当且仅当$L_f$是线性空间$V$到$V^*$的一个同构映射.
        
        \begin{solution}
            \begin{equation}
            \nonumber
            \begin{aligned}
                \text{f非退化} &\Longleftrightarrow \text{rad}_L V = 0\\
                              &\Longleftrightarrow \Ker L_f = 0\\
                              &\Longleftrightarrow L_f \text{是双射}\\
            \end{aligned}
            \end{equation}
        \end{solution}

    \end{itemize}
    \item[2.] 若$f$非退化
    \begin{itemize}
        \item [(a)] 任给$V$上的一个双线性函数$g$,存在$V$上唯一的一个线性变换
        $G$,使得$$g(\alpha, \beta) = f(G\alpha, \beta),\, \forall \alpha,\beta \in V$$

        \begin{solution}
            即证$\alpha_L = (G\alpha)_L$,由于$L_f$是$V$到$V^*$的一个同构映射,
            则存在$\hat \alpha \in V, L_f(\hat \alpha) = \alpha_L$,
            定义$G: \alpha \rightarrow \hat \alpha$. 证明$G$线性且唯一.
        \end{solution}

        \item [(b)] 令$\sigma: g\rightarrow G$, 则$\sigma$是$T_2(V)$到$Hom(V,V)$的一个同构映射.
        
        \begin{solution}
            证明$\sigma$是线性映射,且单射.
        \end{solution}

        \item [(c)] 证明V中存在一个基使得$f,g$在此基下的度量矩阵都是对角矩阵的充分必要条件是$G$可以对角化.
        
        \begin{solution}
            充分性:设G的特征空间为$V_i$, 则$V = V_1 \oplus V_2 \oplus \cdots \oplus V_s$,
            $f|V_{i}$是$V_i$上的对称双线性函数, 则存在$V_i$上的一组基使得$f|V_i$
            的度量矩阵是对角的, 易见$g$在这组基下也是对角的. 而对于$i\ne j$, $0 \ne \alpha_i \in V_i, 0 \ne \alpha_j \in V_j$,
            则
            $$g(\alpha_i, \alpha_j) = f(G\alpha_i, \alpha_j) = \lambda_i f(\alpha_i, \alpha_j)$$
            $$g(\alpha_j, \alpha_i) = \lambda_j f(\alpha_j, \alpha_i) $$
            故$f(\alpha_i, \alpha_j) = 0, g(\alpha_i, \alpha_j) = 0$.

            必要性:设在基$\alpha_1, \cdots, \alpha_n$下$f,g$的度量矩阵都是对角的,
            即$\forall j \ne i, g(\alpha_i, \alpha_j) = 0, f(\alpha_i, \alpha_j) = 0$,
            考虑$G \alpha_i$, 由于$f(G\alpha_i, \alpha_j) = 0$, 
            将$G \alpha_i$按照基$\alpha_i$以及$\alpha_1, \alpha_2,
            \cdots \alpha_{i-1}, \alpha_{i+1}, \cdots, \alpha_n$
            的一组正交基$\hat \alpha_1,
            \cdots \hat \alpha_{i-1}, \hat \alpha_{i+1}, \cdots, 
            \hat \alpha_n$展开,
            由于$f(\hat \alpha_j, \hat \alpha_j) > 0$, 故$G\alpha_i \in <\alpha_i>.$
            故在这组基下$G$对角化.
        \end{solution}

        \item [(d)] 设$A,B$都是特征不为2的域F上的n级对称矩阵,且$A$是可逆的,
        证明$A,B$可以同时合同对角化的充分必要条件是$A^{-1}B$可以对角化.
        
        \begin{solution}
            给定空间$V$的一组基,$f,g$的度量矩阵为$A,B$, 故$f$非退化.
            设$\alpha, \beta \in V$在此基下的坐标为$x,y$,
            $f(\alpha, \beta) = x^T A y, g(\alpha, \beta) = x^T By$.
            在这组基下,线性变换$G$的矩阵为$\hat G$.
            $$x^T B y = (Gx)^T A y$$
            故$G^TA = B$, $G = A^{-1}B$, 使用上一题结论即可.
        \end{solution}
    \end{itemize} 
\end{itemize}

\section{实内积空间}
实线性空间 + 正定的对称双线性函数 = 实内积空间,
有限维的实内积空间即欧氏空间.

\subsection{实内积空间的度量}
定义$|\alpha| = \sqrt{(\alpha, \alpha)}$,
有柯西不等式
$$|(\alpha,\beta)| \le |\alpha| |\beta|, \forall \alpha, \beta.$$
定义$d(\alpha, \beta) = |\alpha - \beta|$,
证明$d$是一个距离,即满足对称性,正定性和三角不等式.

\begin{solution}
    使用柯西不等式证明$|\alpha + \beta| \le |\alpha| + |\beta|$,
    令$\alpha = a - b, \beta = b - c$即可。
\end{solution}

\subsection{实内积空间的同构}
同构映射$\sigma$, 不仅作为线性空间的同构映射,还要求保持内积, 即
$(\sigma \alpha, \sigma \beta) = (\alpha, \beta)$.
两个欧氏空间同构的充要条件是维数相同.

\subsection{例题}
在$R[x]_{n+1}$中定义内积
$$(f,g) = \int^{1}_{-1}f(x)g(x) \mathrm{d}x.$$
令
$$P_0(x) = 1, P_k(x) = \frac{1}{2^k k!} \frac{d^k}{dx^k} ((x^2-1)^k), k = 1,2,\cdots,n.$$
证明,这是$R[x]_{n+1}$的一个正交基.

\begin{solution}
    证明$P_k$和$x^i(0\le i < k)$正交即可, 不断使用分部求和.
\end{solution}

\section{正交变换}
实内积空间$V$到自身的满射$A$,满足
$$(A\alpha, A\beta) = (\alpha, \beta), \forall \alpha, \beta$$
则称$A$是$V$上的正交变换.
可以证明
\begin{enumerate}
\item $A$是线性的;

\begin{solution}
    计算$|A(\alpha + \beta) - (A\alpha + A\beta)|^2$
\end{solution}
\item $A$是单的, 从而$A$是可逆的.
\item $A$是$V$到自身的一个同构映射.
\end{enumerate}

\subsection{例题}
\begin{itemize}
    \item[1.] 证明,实内积空间V到自身的满射A是正交变换当且仅当A是保持向量长度不变的线性变换。
    \item[2.] 设$V$是n维欧氏空间,$\eta$是$V$中一单位向量,设P是$V$在$<\eta>$
    上的正交投影,令$A=I-2P$, 称$A$为关于$<\eta>^{\perp}$的镜面反射. 证明这是第二类正交变换. 
    \item[3.] 
    \begin{itemize}
        \item [(a)] 设A是2维欧氏空间V上的第二类正交变化, 证明A是关于某一条直线的镜面反射
        \item [(b)] 设A是2维欧氏空间V上的第一类正交变换, 证明A能表示为两个镜面反射的乘积
        \item [(c)] 证明$n$维欧氏空间V上的任一正交变换都可以表示成至多$n$个镜面反射的乘积
        
        \begin{solution}
            (a,b)考虑正交基$\alpha_1, \alpha_2$和$A \alpha_1, A \alpha_2$,
            先用镜面反射$B_1$将$\alpha_1$映射为$A\alpha_1$, 说明$A\alpha_2 = \pm B_1\alpha_2$.
            
            (c) 同理考虑$\alpha_1, \alpha_2,\cdots, \alpha_n$和
            $A\alpha_1, A\alpha_2,\cdots, A\alpha_n$. 假设$\alpha_1 \ne A\alpha_1$,
            取$B_1 \alpha_1 = A\alpha_1$, 考虑$B_1 \alpha_2 \cdots, B_1\alpha_n$
            和$A \alpha_2 \cdots, A\alpha_n$, 这两组都是$A \alpha_1$的正交补空间的基.
            则$C: B_1\alpha_i \rightarrow A\alpha_i, i=2,\cdots, n$
            是$n-1$维空间的正交变换, 根据归纳假设可以至多表示为$B_2,\cdots, B_n$个镜面反射的乘积.
            将$B_2,\cdots, B_n$延拓为$V$上的镜面反射,即证.
        \end{solution}
    \end{itemize}
    
    % \item[2.] 设A是n维欧氏空间V上的一个线性变换,证明$A$是镜面反射当且仅当A在V的任意一个标准正交基
    % 下的矩阵形如$I-2\alpha \alpha^T$, 其中$\alpha$是一单位向量.

    \item[4.] (正交变换的矩阵表示) A是实内积空间$V$上的正交变换.
    \begin{itemize}
        \item [(a)] 假设A有特征值,证明特征值为1或-1.
        
        \begin{solution}
            正交变换保长度
        \end{solution}
        \item [(b)] 证明A属于不同特征值的特征向量互相正交.
        
        \begin{solution}
            不妨设$\alpha_1 = 1, \alpha_2 = -1$,
            $$(\alpha_1, \alpha_2) = (A\alpha_1, AA^{-1}\alpha_2) = (\alpha_1, A^{-1}\alpha_2) = \lambda_2^{-1} (\alpha_1, \alpha_2)$$
            故$2(\alpha_1, \alpha_2) = 0$.
        \end{solution}
        \item [(c)] 若$W$是A的一个有限维不变子空间,则$W^{\perp}$也是A的不变子空间.
        
        \begin{solution}
            A可逆,$A|W$单,所以可逆,所以W也是$A^{-1}$的不变子空间.
            任意的$\alpha \in W, \beta \in W^{\perp}$,
            $$(A\beta, \alpha) = (A\beta, AA^{-1}\alpha) = (\beta, A^{-1}\alpha) = 0,$$
            所以$A\beta \in W^{\perp}$.
        \end{solution}
        \item [(d)] 若$\dim V = 2$, 若A是第一类的,那么V中存在一组正交基,使得
        A在此基下的矩阵为
        \begin{equation}
        \nonumber
        \begin{pmatrix}
            \cos \theta& -\sin \theta\\
            \sin \theta&  \cos \theta 
        \end{pmatrix},
        0 \le \theta \le \pi,
        \end{equation}
        如果A是第二类的,那么存在一组正交基,A在此基下的矩阵为
        \begin{equation}
        \nonumber
        \begin{pmatrix}
            1& 0\\
            0&-1 
        \end{pmatrix}.
        \end{equation}

        \begin{solution}
            A在标准正交基下的矩阵为正交矩阵, 
            二阶正交矩阵只有两种类型:
            \begin{equation}
            \nonumber
            \begin{pmatrix}
                \cos \theta& -\sin \theta\\
                \sin \theta&  \cos \theta 
            \end{pmatrix},\quad
            \begin{pmatrix}
                \cos \theta&  \sin \theta\\
                \sin \theta& -\cos \theta 
            \end{pmatrix},
            0 \le \theta < 2\pi,
            \end{equation}
        \end{solution}
        \item [(e)] 证明V中存在一个标准正交基,使得此基下A的矩阵为分块对角:
        \begin{equation}
        \nonumber
        \mathrm{diag}
        \left\{
            \lambda_1, \cdots, \lambda_r,
            \begin{pmatrix}
            \cos \theta_1& -\sin \theta_1\\
            \sin \theta_1&  \cos \theta_1
            \end{pmatrix},
            \cdots,
            \begin{pmatrix}
            \cos \theta_m& -\sin \theta_m\\
            \sin \theta_m&  \cos \theta_m
            \end{pmatrix}
        \right\}
        \end{equation}
        其中$\lambda_i = 1$或$-1$, $0 < \theta_i < \pi$.
        
        \begin{solution}
            对维数进行数学归纳
        \end{solution}
    \end{itemize}
\end{itemize}