\paragraph{7.3}
\begin{itemize}
  \item 最大公因式: $K[x]$中任意两个多项式都有最大公因式,且可以表示为$f,g$的和式
  \item 互素
\end{itemize}
\paragraph{例题}
\begin{itemize}
%   \item[1.] 设$f(x),g(x) \in K[x]$, $a,b,c,d\in K$而且$ad-bc\ne 0$.证明,
%   $(af(x)+bg(x), cf(x)+dg(x)) = (f(x),g(x))$.
%   \item[2.] 设$A\in M_n(K)$, $f(x),g(x)\in K[x]$, 证明, 如果$d(x) = (f(x), g(x))$, 那么
%   $d(A)x=0$的解空间是$f(A)x=0$解空间和$g(A)x=0$解空间的交.
    \item[1.] $f(x)=x^3+(1+t)x^2+2x+u,\, g(x)=x^3+tx+u$的最大公因式为二次,求$t,u$的值
    \vspace{2cm}
    
    \item[2.] $f(x) = x^3-2x^2+2x-1, g(x)=x^4-x^3+2x^2-x+1$, 求$(f,g)$,并表示为$f,g$的和式
    \vspace{2cm}

    \item[3.] 在$K[x]$中, 如果$(f(x),g(x))=1$, 并且$\deg f >0, \deg g>0$, 那么在
    $K[x]$中存在唯一的多项式$u(x), v(x)$, $\deg u< \deg g, \deg v<\deg f$,
    s.t.
    $$u(x)f(x) +v(x)g(x) = 1.$$
    \vspace{2cm}

    \item[4.] 在$K[x]$中, 如果$(f(x),g(x))=d(x)$, 那么在
    $K[x]$中存在唯一的多项式$u(x)$和$v(x)$, $\deg u < \deg g - \deg d,\,
    \deg v<\deg f - \deg d$,
    s.t.
    $$u(x)f(x) +v(x)g(x) = d(x).$$
    \vspace{1cm}

%   \item[5.] 在$K[x]$中的两个非零多项式$f(x)$和$g(x)$不互素的充分必要条件是,存在两个非零多项式
%   $u(x),v(x)$, s.t. $u(x)f(x) = v(x)g(x)$且$deg\,u <deg\,g$, $deg\,v<deg\,f$.
    \item[5.] 在$K[x]$中,$f_1(x),f_2(x),\cdots,f_s(x)$两两互素,证明对于任意的
    $r_1(x),r_2(x),\cdots, r_s(x)\in K[x]$, 同余方程组
    \begin{equation}
    \nonumber
    \left\{
      \begin{aligned}
      g(x) &\equiv r_1(x) \quad \mathrm{mod}\, f_1(x)\\
      g(x) &\equiv r_2(x) \quad \mathrm{mod}\, f_2(x)\\
      \vdots&\\
      g(x) &\equiv r_s(x) \quad \mathrm{mod}\, f_s(x)\\
      \end{aligned}
    \right.
    \end{equation}
    在$K[x]$中有解,且若$c(x),d(x)$均为解,则$c(x) \equiv d(x),\,\,\mathrm{mod}\,f_1 f_2\cdots f_s$.
    \vspace{2cm}
\end{itemize}

\paragraph{7.4}
\begin{itemize}
  \item 不可约多项式
  \item 唯一因式分解定理;这里唯一性的意义是?
\end{itemize}
\paragraph{例题}
\begin{itemize}
%   \item[1.] 在$K[x]$中,设$(f,g_i)=1,i=1,2$. 证明$(fg_1,g_2)=(g_1,g_2)$.
%   \item[2.] 在$K[x]$中, 证明对于任意的正整数$m$,有$(f^m(x), g^m(x)) = (f(x),g(x))^m$.
%   \item[3.] 分别在复数域、实数域和有理数域上分解 $x^4+1$为不可约多项式的乘积 

\item[1.] 证明,数域$K$上的一个次数大于零的多项式$f$与$K[x]$中某一
不可约多项式的正整数次幂相伴的充分必要条件是,对于任意$g(x)\in K[x]$,
必有$(f(x), g(x)) = 1,$或者存在一个整数数$m$, 使得$f(x)|g^m(x)$.
\vspace{2cm}
\end{itemize}

\paragraph{7.5等}
\begin{itemize}
    \item 重因式
    \item 复数域上的不可约多项式只有一次的
    \item 实数域上的不可约多项式都是一次的,或判别式小于零的二次多项式
    \item 有理数域上的不可约多项式可以是任意次数的
\end{itemize}

\paragraph{例题}
\begin{itemize}
\item[1.] 证明, $K[x]$中一个$n$次($n\ge 1$)多项式$f(x)$, 能被它的导数整除的充分必要条件是
$f(x)$与一个一次因式的$n$次幂相伴.
\vspace{2cm}

\item[2.] 设$K$是一个数域,$R$是$K$的一个交换扩环,设$a\in R,$ 其中$J_a \ne {0},$
$$ J_a = \{f(x) \in K[x] | f(a) = 0\}$$ 
证明:
\begin{itemize}
    \item [(1)] $J_a$中存在唯一的首项系数为$1$的多项式$m(x)$, 使得$J_a$的元素都是$m(x)$
    的倍式.
    \vspace{2cm}

    \item [(2)] 如果$R$是无零因子环, 则$m(x)$在$K[x]$中不可约.
    \vspace{2cm}

    \item [(3)] 取$K=C, R=C[A]$, 其中
    \begin{equation}
        \nonumber
        A=\begin{bmatrix}
            1& -1\\
            1& 1\\
        \end{bmatrix}
    \end{equation}
    求$J_A$中的$m(x)$, 并判断$C[A]$是无零因子环么?
    \vspace{2cm}
\end{itemize}

\item[3.] 在复数域上求下述循环矩阵的全部特征值以及行列式
\begin{equation}
    \nonumber
    A=\begin{bmatrix}
        a_1& a_2& a_3& \cdots& a_n\\
        a_n& a_1& a_2& \cdots& a_{n-1}\\
        \vdots& \vdots& \vdots& &\vdots\\
        a_3& a_4& a_5& \cdots& a_{2}\\
        a_2& a_3& a_4& \cdots& a_{1}\\
    \end{bmatrix}
\end{equation}
\vspace{2cm}
\end{itemize}


