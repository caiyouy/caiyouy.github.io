\documentclass{article}

%%%% Chinese
% fontset = none 可以为后续自定义字体
\usepackage[UTF8, scheme=plain]{ctex}
% % 设置英文字体
% \setmainfont[BoldFont={Ubuntu Bold}, ItalicFont={Ubuntu Italic}]{Ubuntu}
% \setmainfont{Microsoft YaHei}
% \setsansfont{Comic Sans MS}
% \setmonofont{Courier New}
% % 设置中文字体
% \setCJKmainfont[Mapping = fullwidth-stop]{SimSun} %选项将。映射为.
% \setCJKmainfont{SimSun}
% \setCJKmonofont{Source Code Pro}
% \setCJKsansfont{YouYuan}

%%% 习题课讲义排版
\usepackage{xparse}
\newtoks\patchtoks    % helper token register
\def\longpatch#1%     % worker macro
  {\let\myoldmac#1%
   \long\def#1##1{\patchtoks={##1}\myoldmac{\the\patchtoks}}}
\longpatch{\phantom}
\NewDocumentEnvironment{solution}{ +b}{%
  \ifsolution
      \textbf{解答}\quad #1
  \else
      \phantom{\parbox{\textwidth}{#1}}
  \fi
  }{\par}
\newif\ifsolution
% \solutiontrue %添加此句将输出答案,否则输出答案所需的空白

%%%%% MATH
\usepackage{amsmath, amssymb, amsthm}
\usepackage{physics}
\newtheorem{theorem}{Theorem} % 定义定理环境等
\newtheorem{lemma}{Lemma}
\newtheorem{definition}{Definition}
\newtheorem*{remark}{Remark}
\numberwithin{equation}{section} % 公式分节标号

%%%%% Paper size
\usepackage{geometry}
\geometry{a4paper,left=2cm,right=2cm,bottom=2.5cm,top=2.5cm}

%%%%% Text
\usepackage{bm}
\usepackage{ulem} % underlining
\usepackage{enumitem} % customizing lists
\usepackage{titling} % 标题调整
\pretitle{           % 标题左对齐
  \begin{flushleft}
}
\posttitle{
  \end{flushleft}
}
\preauthor{
  \begin{flushleft}
}
\postauthor{
  \end{flushleft}
}
\predate{
  \begin{flushleft}
}
\postdate{
  \end{flushleft}
  \noindent\vrule height 1.0pt width \textwidth
  \vskip .75em plus .25em minus .25em% increase the vertical spacing a bit, make this particular glue stretchier
}
\usepackage{color}
\usepackage{setspace}
\renewcommand{\baselinestretch}{1.3}

%%%%% Ref
\usepackage[
  giveninits=true,       % 作者姓名首单词缩写, 并大写
  date = year,           % 日期只是年份
  % bibstyle = mystyle,  % 参考文献样式(文章最后的文献条目)
]{biblatex}
\renewbibmacro{in:}{} % 去掉文章前面的in
\DeclareFieldFormat[article,incollection,unpublished]{title}{#1} % No quotes for article titles
\DeclareFieldFormat[thesis]{title}{\mkbibemph{#1}} % Theses like book titles
\DeclareFieldFormat{journaltitle}{#1\isdot}
\DeclareFieldFormat{pages}{#1} % 去掉页码前面的pp
% 使用shortjournal代替journaltitle
\DeclareSourcemap{
  \maps[datatype=bibtex]{
    \map[overwrite]{ % Notice the overwrite: replace one field with another
      \step[fieldsource=shortjournal,fieldtarget=journaltitle]
    }
  }
}
\renewcommand*{\bibpagespunct}{\addspace} % This tells biblatex to only put a space right before the pages, no other punctuation.
\renewcommand*{\newunitpunct}{\addcomma\space}
\addbibresource{report.bib}
\usepackage[colorlinks,
            linkcolor=red,
            anchorcolor=red,
            citecolor=blue
            ]{hyperref}

%%%% Graphic
\usepackage{graphics} %color
\usepackage{tikz}
\usetikzlibrary{decorations.pathreplacing,calligraphy} % tikz 花括号

%%%%% Table
\usepackage{tabularx} %table column fix width
\newcolumntype{L}[1]{>{\raggedright\arraybackslash}p{#1}}
\newcolumntype{C}[1]{>{\centering\arraybackslash}p{#1}}
\newcolumntype{R}[1]{>{\raggedleft\arraybackslash}p{#1}}
\usepackage{booktabs} % three lines table
\usepackage{multirow} % multi row; multi column self-contained

%%%%% Figure
\usepackage{graphicx}  % minipage
\usepackage{subfig}    % subfig

%%%%% Code and other
% \usepackage[numbered,framed]{matlab-prettifier} %matlab
% \lstset{
%   style              = Matlab-editor,
%   basicstyle         = \mlttfamily,
%   escapechar         = ",
%   mlshowsectionrules = true,
% }
\usepackage{listings}
\usepackage{algorithm}
\usepackage{algorithmic}
\algsetup{
  indent=3em,
  linenosize=\small,
  linenodelimiter=.
}

%%%%%%% New command
%\newcommand{\char}[0]{\mathrm{char}\,}
\newcommand{\Ker}[0]{\mathrm{Ker}\,}
\newcommand{\Img}[0]{\mathrm{Im}\,}

\title{{\huge \bfseries
线性代数A(II)习题课讲义08
       }}
\author{Caiyou Yuan}
\date{\today}

\begin{document}
\maketitle
% include, new page

约定这里所提及的线性空间均是域$F$上的有限维线性空间.
\section{线性空间的张量积}
\begin{definition}
    对于线性空间$U,V$, 若存在线性空间$T$和双线性映射$\otimes: U\times V \rightarrow T$,
    满足: 
    \begin{enumerate}
        \item[$*$] 对于任一线性空间$W$, 以及双线性映射$f: U\times V \rightarrow W$, 
    都存在唯一的线性映射$F: T\rightarrow W$, 使得$f = F \circ \otimes$,
    \end{enumerate}
    则称二元组$(T, \otimes)$是$U,V$的张量积.
\end{definition}

\begin{remark}
    说明$U, V$的张量积在同构的意义下是唯一的, 即
    若$(T_1, \otimes_1)$, $(T_2, \otimes_2)$均为$U,V$的张量积,
    则$T_1$和$T_2$同构.
    所以我们用$U\otimes V$表示$U,V$的张量积, 
    这里省略了双线性映射$\otimes$.
\end{remark}
\vspace{6cm}

\begin{remark}
说明$U,V$的张量积是存在的.
取$T$为$U,V$上的双线性函数空间, 即$T = L(U^*,V^*;F)$.
\end{remark}
\vspace{5cm}

\begin{remark}
    证明上述性质$*$ 等价于
    \begin{enumerate}
        \item [$*_1$] $T = span( \Img \otimes)$
        \item [$*_2$] 对于任一线性空间$W$, 以及双线性映射$f: U\times V \rightarrow W$, 
    都存在线性映射$F: T\rightarrow W$, 使得$f = F \circ \otimes$,
    \end{enumerate}
\end{remark}
\vspace{6cm}

\begin{remark}
    $Im \otimes$是$T$的子空间么?若是给出证明,若不是举出反例.
\end{remark}
\vspace{4cm}

\begin{remark}
    证明$U \otimes V$和$V \otimes U$同构.
\end{remark}
\vspace{5cm}

\section{多个线性空间的张量积}
\begin{definition}
    对于线性空间$U_i(i=1,\cdots, N)$, 
    若存在线性空间$T$和$N$重线性映射$\otimes: U_1\times \cdots \times U_N
     \rightarrow T$,
    满足: 
    \begin{enumerate}
        \item[$*$] 对于任一线性空间$W$, 以及$N$重线性映射$f: U_1\times \cdots \times U_N
        \rightarrow W$, 都存在唯一的线性映射$F: T\rightarrow W$, 使得$f = F \circ \otimes$,
    \end{enumerate}
    则称二元组$(T, \otimes)$是$U_i(i=1,\cdots, N)$的张量积.
\end{definition}
\begin{remark}
    这里的存在唯一性,和上一节$N=2$的情形说明方式相同.
\end{remark}

\begin{remark}
    证明$U_1 \otimes U_2 \otimes U_3$ 和$(U_1 \otimes U_2) \otimes U_3$同构.
    其中$U_1, U_2, U_3$是三个线性空间.
\end{remark}
\vspace{6cm}


\section{线性变换的张量积}
\begin{definition}
    对于线性空间$U,V$, 设$A,B$分别是$U,V$上的线性变换, 则存在唯一的
    $U\otimes V$上的线性变换, 记为$A\otimes B$, 使得
    \begin{equation}
    (A\otimes B)(\alpha \otimes \beta) = A\alpha \otimes B\beta, 
    \quad \forall \alpha \in V, \beta \in U.
    \label{equ:tensor_AB}
    \end{equation}
    称$A\otimes B$为$A,B$的张量积.
\end{definition}
\begin{remark}
    证明满足\eqref{equ:tensor_AB}的线性映射是存在唯一的.
\end{remark}
\vspace{3cm}

\begin{remark}
    若$A_1, A_2$是$U$上的线性变换, $B_1, B_2$是$V$上的线性变换,
    说明$(A_1 \otimes B_1)(A_2 \otimes B_2) = A_1A_2 \otimes B_1 B_2$
\end{remark}
\vspace{3cm}

\begin{remark}
    说明$\Img (A\otimes B) = \Img A \otimes \Img B$,
        $\rank (A\otimes B) = (\rank A)(\rank B)$
    %    $\Ker (A\otimes B) = \Ker$
\end{remark}
\vspace{5cm}

\begin{remark}
    证明$(A\otimes B)_{\Phi_1 \otimes \Phi_2} = A_{\Phi_1} \otimes B_{\Phi_2}$,
    其中$\Phi_1 = \{\alpha_1, \cdots \alpha_n\}$,
    $\Phi_2 = \{\beta_1, \cdots \beta_m\}$分别是
    $U,V$的一组基. $A_{\Phi_1}, B_{\Phi_2}$分别是$A,B$在基$\Psi_1,\Psi_2$
    下的矩阵, $A_{\Phi_1} \otimes B_{\Phi_2}$中的$\otimes$表示矩阵的Kronecker
    乘积.
\end{remark}
\vspace{6cm}

\begin{remark}
    设$A,B$分别是域$F$上的$n,m$级矩阵,证明$A\otimes B$和$B\otimes A$相似.
\end{remark}
\end{document}
