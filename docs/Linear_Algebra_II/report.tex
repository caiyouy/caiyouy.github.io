\documentclass{article}
% \documentclass{book} // final aim

%%%% Chinese
% fontset = none 可以为后续自定义字体
\usepackage[UTF8, scheme=plain]{ctex}
% % 设置英文字体
% \setmainfont[BoldFont={Ubuntu Bold}, ItalicFont={Ubuntu Italic}]{Ubuntu}
% \setmainfont{Microsoft YaHei}
% \setsansfont{Comic Sans MS}
% \setmonofont{Courier New}
% % 设置中文字体
% etCJKmainfont[Mapping = fullwidth-stop]{SimSun} %选项将。映射为.
% \setCJKmainfont{SimSun}
% \setCJKmonofont{Source Code Pro}
% \setCJKsansfont{YouYuan}

%%% 习题课讲义排版
\usepackage{xparse}
\newtoks\patchtoks    % helper token register
\def\longpatch#1%     % worker macro
  {\let\myoldmac#1%
   \long\def#1##1{\patchtoks={##1}\myoldmac{\the\patchtoks}}}
\longpatch{\phantom}
\NewDocumentEnvironment{solution}{ +b}{%
  \ifsolution
      \textbf{解答}\quad #1
  \else
      \phantom{\parbox{\textwidth}{#1}}
  \fi
  }{\par}
\newif\ifsolution
% \solutiontrue %添加此句将输出答案,否则输出答案所需的空白

%%%%% MATH
\usepackage{amsmath, amssymb, amsthm}
\usepackage{physics}
\newtheorem{theorem}{Theorem} % 定义定理环境等
\newtheorem{lemma}{Lemma}
\newtheorem{definition}{Definition}
\newtheorem*{remark}{Remark}
\numberwithin{equation}{section} % 公式分节标号

%%%%% Paper size
\usepackage{geometry}
\geometry{a4paper,left=2cm,right=2cm,bottom=2.5cm,top=2.5cm}

%%%%% Text
\usepackage{bm}
\usepackage{ulem} % underlining
\usepackage{enumitem} % customizing lists
\usepackage{titling} % 标题调整
\pretitle{           % 标题左对齐
  \begin{flushleft}
}
\posttitle{
  \end{flushleft}
}
\preauthor{
  \begin{flushleft}
}
\postauthor{
  \end{flushleft}
}
\predate{
  \begin{flushleft}
}
\postdate{
  \end{flushleft}
  \noindent\vrule height 1.0pt width \textwidth
  \vskip .75em plus .25em minus .25em% increase the vertical spacing a bit, make this particular glue stretchier
}
\usepackage{color}
\usepackage{setspace}
\renewcommand{\baselinestretch}{1.3}

%%%%% Ref
\usepackage[
  giveninits=true,       % 作者姓名首单词缩写, 并大写
  date = year,           % 日期只是年份
  % bibstyle = mystyle,  % 参考文献样式(文章最后的文献条目)
]{biblatex}
\renewbibmacro{in:}{} % 去掉文章前面的in
\DeclareFieldFormat[article,incollection,unpublished]{title}{#1} % No quotes for article titles
\DeclareFieldFormat[thesis]{title}{\mkbibemph{#1}} % Theses like book titles
\DeclareFieldFormat{journaltitle}{#1\isdot}
\DeclareFieldFormat{pages}{#1} % 去掉页码前面的pp
% 使用shortjournal代替journaltitle
\DeclareSourcemap{
  \maps[datatype=bibtex]{
    \map[overwrite]{ % Notice the overwrite: replace one field with another
      \step[fieldsource=shortjournal,fieldtarget=journaltitle]
    }
  }
}
\renewcommand*{\bibpagespunct}{\addspace} % This tells biblatex to only put a space right before the pages, no other punctuation.
\renewcommand*{\newunitpunct}{\addcomma\space}
\addbibresource{report.bib}
\usepackage[colorlinks,
            linkcolor=red,
            anchorcolor=red,
            citecolor=blue
            ]{hyperref}

%%%% Graphic
\usepackage{graphics} %color
\usepackage{tikz}
\usetikzlibrary{decorations.pathreplacing,calligraphy} % tikz 花括号

%%%%% Table
\usepackage{tabularx} %table column fix width
\newcolumntype{L}[1]{>{\raggedright\arraybackslash}p{#1}}
\newcolumntype{C}[1]{>{\centering\arraybackslash}p{#1}}
\newcolumntype{R}[1]{>{\raggedleft\arraybackslash}p{#1}}
\usepackage{booktabs} % three lines table
\usepackage{multirow} % multi row; multi column self-contained

%%%%% Figure
\usepackage{graphicx}  % minipage
\usepackage{subfig}    % subfig

%%%%% Code and other
% \usepackage[numbered,framed]{matlab-prettifier} %matlab
% \lstset{
%   style              = Matlab-editor,
%   basicstyle         = \mlttfamily,
%   escapechar         = ",
%   mlshowsectionrules = true,
% }
\usepackage{listings}
\usepackage{algorithm}
\usepackage{algorithmic}
\algsetup{
  indent=3em,
  linenosize=\small,
  linenodelimiter=.
}

%%%%%%% New command
%\newcommand{\char}[0]{\mathrm{char}\,}
\newcommand{\Ker}[0]{\mathrm{Ker}\,}
\newcommand{\Img}[0]{\mathrm{Im}\,}

\title{{\huge \bfseries
主理想整环上的有限生成模
% Modules over Principal Ideal Domains
       }}
\author{Caiyou Yuan}
\date{\today}

\begin{document}
\maketitle
% include, new page
\section{出发点}
当研究线性空间$V$上的线性映射时, 
域$F$上的线性空间$V$, 也可以看成是多项式环$F[x]$(主理想整环)上的模.
通过模的角度,可以自然推导出线性映射的有理/Jordan标准型.
这里主要参考了Steven Roman的Advanced Linear Algebra.

\section{预备知识}
关于群/环/域的定义, 这里略去.

\subsection{主理想整环}
\begin{definition}
	对于环$R$,非零元$r \in R$被称为零因子(zero divisor),
	如果存在非零元$s \in R,st = 0$. 
	带单位元的交换环R, 如果没有零因子, 则被称为整环.
\end{definition}

\begin{definition}
	环R的子集I, 如果满足
	\begin{itemize}
		\item[1.] $a - b \in I, \quad \forall a,b \in I$
		\item[2.] $ab \in I, ba \in I, \quad \forall a \in R, b \in I$
	\end{itemize}
	则被称为理想. 
\end{definition}
\begin{remark}
	对于$a \in R$, $I = \{ra, r \in R\}$是理想.
	这种由单个元素生成的理想,被称为主理想.
\end{remark}

\begin{definition}
主理想整环,即是所有理想均为主理想的整环.
\end{definition}
\begin{remark}
	域$F$上的多项式$F[x]$是一个主理想整环.
\end{remark}

\subsection{模}
\begin{definition}
R是一个有单位元的环, 集合$M$以及$M$上定义的两种运算,
加法$(+)$和数乘$(\cdot)$, 如果满足
\begin{itemize}
	\item[1.] $(M, +)$ 是一个交换群
	\item[2.] $1v = v, (rs)v = r(sv), \quad \forall r,s \in R, v \in M$
	\item[3.] $(r+s)u = ru + su, r(u+v) = ru + rv, 
	            \quad \forall r,s \in R, u,v \in M.$
\end{itemize}
则称$M$为一个R上的模, 简称R模.
\end{definition}
因为它和线性空间的定义仅仅区别在,这里$R$只是一个单位环, 而非域,
所以,
模可以大致理解为"环上的线性空间".

\begin{remark}
	当$R$是域时, M即为线性空间; 环R自身就是一个R模.
\end{remark}

和线性空间类似,同理可以引入例如子模,线性相关/无关, 基等概念.
但因为模只是定义在环上,所以有一些和线性空间完全不同的情形.

\paragraph{没有基的模}
环$Z_n$是一个整数环$Z$上的模, 任意$v \in Z_n$, 都有$nv = 0$.
所以$Z_n$中任意一个向量都是线性相关的, 所以也没有基.
特殊的, 把有基的模称为自由模(free module).

\paragraph{自由模但有不自由的子模}
$Z\times Z$是$Z \times Z$上的模, 基是$(1,1)$.
张成整个$Z\times Z$: $(n,m)=(n,m)(1,1)$;
线性无关性:$(n,m)(1,1)=(0,0) \Rightarrow (n,m)=(0,0)$.
但是子模$Z \times \{0\}$没有基,因为$(0,1)(n,0) = (0,0)$.

\paragraph{自由模的基的元素个数可以任意}
$V$是数域$F$上的线性空间,有可数多个元素的基$B = \{b_1, b_2, \cdots\}$.
$R=L(V)$是一个非交换环, $R$自身是$R$上的一个模, 单位映射是$R$的一个基.
对于任意的正整数$n$, 将$B$分成$n$个部分, 对于$s=0,1,\cdots,n-1,$ 
$B_s = \{b_i \,|\, i \equiv s \,\mathrm{mod}\, n\}$.
定义$\beta_s$,
\begin{equation}
\nonumber
\beta_s(b_{kn+t}) = 
\left\{
\begin{array}[]{cc}
	b_k& \text{if } t = s\\
	0  & \text{if } t \ne s\\
\end{array}
\right.
\end{equation}
可以验证$C=\{\beta_0, \beta_1, \cdots, \beta_{n-1}\}$张成$L(V)$,
且线性无关,即为$L(V)$的$n$个元素的基.

\section{主要结果}
主理想整环上的有限生成模的分解定理如下
\begin{theorem}
如果$R$是主理想整环, $M$是一个有限生成的$R$模,则$M$可以分解为有限多个循环模的直和,
也就是
\begin{equation}
M \cong R^n \oplus R/\langle a_1 \rangle \oplus \cdots 
\oplus R/\langle a_d \rangle
\label{equ:decomp_0}
\end{equation}
其中$n\ge 0$, 非零不可逆元$a_1,a_2,\cdots, a_d \in R$满足$a_1|a_2|\cdots |a_d$,
被称为不变因子.
\end{theorem}

\section{定理证明}
\begin{definition}
	假设R是一个交换环,M是一个R模,对于$M$中的元素$v$, 如果存在R中元素$r \ne 0$,
	使得$rv = 0$, 则称$v$是挠元(torsion element). 
	$M_{tor}$是M中所有挠元构成的集合,这是一个子模.
	如果$M_{tor} = M$, 则称M为挠模,反正$M_{tor}=\{0\}$, 称$M$为无挠模.
\end{definition}

\begin{theorem}
	$M$是一个有限生成的主理想整环$R$上的模, 则$M$可以分解为
	\begin{equation}
	M = M_{tor} \oplus M_{free}
	\label{equ:decom_1}
	\end{equation}
\end{theorem}
\begin{proof}
	可以说明$M/M_{tor}$有限生成,且无挠,进而可以说明是有基的.
\end{proof}

进一步的,考虑对$M_{tor}$的分解.
定义$ann(v) = \{r\in R\,|\, rv = 0\}, 
ann(M) = \{r\in R \,|\, rM = \{0\}\}$.
对于有限生成的挠模$M_{tor}$, $ann(M) \ne 0$,
且$ann(M)$是主理想整环R上的理想, 则可由单个元素张成,
这被称为$M$的阶(order).

\begin{theorem}
	$M$是一个有限生成的主理想整环$R$上的挠模, $M$的阶$\mu$可以质因子分解为
	$$\mu = p_1^{e_1}p_2^{e_2}\cdots p_n^{e_n}$$
	则$M$可以分解为
	\begin{equation}
	M = M_{p_1} \oplus M_{p_2} \oplus \cdots \oplus M_{p_n}
	\label{equ:decom_2}
	\end{equation}
	其中$M_{p_i} = \{v\in M\, |\, p_i^{e_i} v = 0\}$.
\end{theorem}

\begin{theorem}
	$M$是一个有限生成的主理想整环$R$上的挠模, $M$的阶$\mu = p^e$,
	则$M$可以分解为
	\begin{equation}
	M = C_1 \oplus C_2 \oplus \cdots \oplus C_k
	\label{equ:decom_3}
	\end{equation}
	其中$C_i$为循环模,阶为$p^{e_i}$, 且$e = e_1 \ge e_2 \ge \cdots \ge e_k$.
\end{theorem}

综合分解式\eqref{equ:decom_1}, \eqref{equ:decom_2}, \eqref{equ:decom_3},
则对于
一个有限生成的主理想整环$R$上的模$M$,
其初等因子版本的循环分解为
\begin{equation}
M = M_{free} 
    \oplus (C_{1,1}\oplus \cdots C_{1,k_1})
    \oplus (C_{2,1}\oplus \cdots C_{2,k_2})
	\oplus \cdots
    \oplus (C_{n,1}\oplus \cdots C_{n,k_n})
\end{equation}
其中$C_{i,j}$是循环模,阶为$p_i^{e_{i,j}}$, 被称为初等因子.
记$D_j = C_{1,j} \oplus \cdots \oplus C_{n,j}$, 
则$D_j$是循环子模,阶$q_j = \prod_{j} p_j^{e_{ij}}$,
$M$不变因子版本的循环分解为
\begin{equation}
M = M_{free} 
    \oplus D_1 
    \oplus D_2 
	\oplus \cdots
    \oplus D_m
\end{equation}
其中$D_j$是循环模,阶为$q_j = \prod_{j} p_j^{e_{ij}}$, 被称为不变因子.
注意到,对于秩为$n$的$R$模$M$, $M \cong R^n$,
而对于$ann(M) = \langle a \rangle$的循环模$M$, $M \cong R/\langle a \rangle$,
故有分解式\eqref{equ:decomp_0}成立.

\section{应用}
对于有限维线性空间$V$上的线性变换$T$, $V$可以看作
是多项式环$F[x]$上的一个模, 其中数乘的定义为
\begin{equation}
\nonumber
p(x)v = p(T)v, \quad v \in V
\end{equation}
不难验证, $V$是主理想整环$F[x]$上的有限生成无挠模.
所以
\begin{equation}
\nonumber
V = D_1 \oplus D_2 \oplus \cdots \oplus D_m
\end{equation}
假设
$ann(D_1) = 
\langle p_1\rangle$,
$p_1 = a_0 + a_1 x + \cdot + a_{n-1} x^{n-1} + x^n$,
且$D_1  = \langle v \rangle$.
则若选择$D_1$的一组基为
$$v, Tv, \cdots, T^{n-1}v,$$
则$T$在此组基下的
矩阵为
\begin{equation}
T(v, Tv, \cdots, T^{n-1}v)
= (v, Tv, \cdots, T^{n-1}v)
\begin{bmatrix}
	0& 0& \cdots &0 & -a_0\\
	1& 0& \cdots &0 & -a_1\\
	0& 1& \cdots &0 & -a_2\\
	\vdots& \vdots& &\vdots& -a_{n-2}\\
	0& 0& \cdots& 1& -a_{n-1}\\
\end{bmatrix}
\end{equation}
在$D_2, \cdots, D_m$中同样处理,即得到线性映射$T$的有理标准形.

若$p_i^{e_ij} = (x-\lambda_i)^{e_{ij}}$, 
则选择
则若选择$D_1$的一组基为
$$v, (T-\lambda_i I)v, \cdots, (T-\lambda I)^{e_{ij}-1}v,$$
则$T$在此组基下的
矩阵为
\begin{equation}
T(v, (T-\lambda_i I)v, \cdots, (T-\lambda I)^{e_{ij}-1}v)
= (v, (T-\lambda_i I)v, \cdots, (T-\lambda I)^{e_{ij}-1}v)
\begin{bmatrix}
	\lambda_i& 0& \cdots &0 &0 \\
	1& \lambda_i& \cdots &0 &0\\
	0& 1& \cdots &0 & 0\\
	\vdots& \vdots& &\vdots& 0\\
	0& 0& \cdots& 1& \lambda_i\\
\end{bmatrix}
\end{equation}
在$D_2, \cdots, D_m$中同样处理,即得到线性映射$T$的Jordan标准形.


\end{document}
