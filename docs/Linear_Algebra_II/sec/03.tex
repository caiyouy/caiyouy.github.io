\paragraph{8.4}
\begin{itemize}
    \item 商空间
    \item $\dim{(V/W)} = \dim V - \dim W$
\end{itemize}

\paragraph{例题}
\begin{itemize}
    \item [1.] 设$U,W$都是域F上线性空间$V$的子空间, 证明$(U+W)/W \cong U/(U\cap W)$
    \vspace{2cm}
    \item [2.] 设$V$是域F上的一个$n$维线性空间, ($n\ge 3$). $U$是$V$的一个2维子空间,用
    $\Omega_1$表示V中包含U的所有$n-1$维子空间组成的集合,用$\Omega_2$表示商空间$V/U$的所有
    $n-3$维子空间组成的集合,令
    \begin{equation}
        \nonumber
        \begin{aligned}
        \sigma:\quad &\Omega_1 \longrightarrow \Omega_2\\
                     & W \longrightarrow W/U.
        \end{aligned}
    \end{equation}
    证明: $\sigma$是双射.
    \vspace{3cm}
\end{itemize}

\paragraph{9.1-4}
\begin{itemize}
    \item 线性映射,线性变换,线性函数
    \item 线性映射的核与像:
    \begin{itemize}
        \item[1.] $\dim{(\Ker A)} + \dim{(\Img A)} = \dim{V}$
        \item[2.] 有限维线性变换, 单等价于满
    \end{itemize}
    \item 线性映射的矩阵表示:
    \begin{itemize}
        \item[1.] $Hom(V,V') \cong M_{s \times n}(F)$
        \item[2.] 相似矩阵 $\Longleftrightarrow$ 线性变换在不同基下的表示 
    \end{itemize}
    \item 线性映射的行列式,秩,迹,特征值,特征向量等
\end{itemize}

\paragraph{例题}
\begin{itemize}
    \item [1.] (Frobenius秩不等式) 设$V,U,W,M$都是域F上的线性空间,
    并且$V,U$都是有限维的,设$A \in Hom(V,U), B \in Hom(U,W), C \in Hom(W,M)$.
    证明,$$\rank(CBA) \ge \rank(CB) + \rank(BA) - \rank(B).$$
    \vspace{2cm}
    \item [2.] (幂零矩阵的矩阵表示)设V是域$F$上的n维线性空间,A是V上的一个线性变换. 如果
    $A^{n-1} \ne 0, A^n = 0$, 那么在V中存在一个基,使得A在此基下的矩阵为
    \begin{equation}
    \nonumber
    \begin{bmatrix}
        0& 1& 0& \cdots &0\\
        0& 0& 1& \cdots &0\\
        \vdots& \vdots& \vdots& \cdots & 1\\
        0& 0& 0& \cdots &0\\
    \end{bmatrix}
    \end{equation}
    \vspace{3cm}
    \item [3.] 设$V$和$V'$分别是域$F$上n维,s维线性空间,A是$V$到$V'$
    的一个线性映射,证明存在$V$的一个基和$V'$的一个基,使得A在这对基下的矩阵为,
    \begin{equation}
    \nonumber
    \begin{bmatrix}
        I_r& 0\\
        0&   0\\
    \end{bmatrix}
    \end{equation}
    其中$r = \rank(A)$.
    \vspace{3cm}
    \item [4.](两两正交的幂等变换的充要条件)
    设$A_i \in M_n(K)$, $i=1,2,\cdots,s$,其中$K$是数域.
    令$A=A_1 + A_2 +\cdots + A_s$.
    \begin{itemize}
        \item[(1) ] 证明: 如果
              $$ \rank(A) = \rank(A_1) + \rank(A_2) \cdots + \rank(A_s),$$
              那么
              $$AM_n(K) = A_1M_n(K) \oplus A_2M_n(K) \oplus \cdots \oplus A_s M_n(K).$$
        \item [(2) ] 证明: $A_1, A_2, \cdots, A_n$是两两正交的幂等矩阵,
              当且仅当$A$是幂等矩阵,并且
              $$ \rank(A) = \rank(A_1) + \rank(A_2) \cdots + \rank(A_s).$$
    \end{itemize}
    \vspace{3cm}
    \item [5. ] (对角化和数域相关) 设$f(x)=x^n +a_{n-1}x^{n-1} +\cdots +a_1 x +a_0$ 是有理数域Q上的一个不可约
    多项式,$n>1$,$\omega$是$f(x)$的一个复根. 把$C$看成是$Q$上的线性空间,令
    \begin{equation}
        \nonumber
        \begin{aligned}
        \mathbf{B}:\quad &Q[x] \longrightarrow C\\
                         &g(x) \longrightarrow g(\omega).
        \end{aligned}
    \end{equation}
    \begin{itemize}
        \item[(1) ] $\mathbf{B}$是不是一线性映射?若是,给出$\Img \mathbf{B}$的一个基和维数;
        \item[(2) ] 令$\mathbf{A}(z) = \omega z, \forall z \in \Img \mathbf{B}$, 
                    $\mathbf{A}$是不是$\Img \mathbf{B}$上的线性变换?如果是,求$\mathbf{A}$在(1)中基下的矩阵A;
        \item[(3) ] $\mathbf{A}$是否可以对角化?
        \item[(4) ] 把矩阵$A$看成是复数域上的矩阵,该矩阵可以对角化么?
    \end{itemize}
    \vspace{3cm}
\end{itemize}
