\paragraph{7.1}
\begin{itemize}
  \item 一元多项式的概念和运算
  \item 环的基本概念; 关于加法构成交换群,乘法具有结合律,加法乘法具有左右分配律
  \item 一元多项式环$K[x]$的通用性质
\end{itemize}

\paragraph{例题}
\begin{itemize}
  \item[1.] $R$是有单位元1($\ne 0$)的环,若对于$a\in R$, $\exists b \in R$, s.t.
$ab = ba = 1$, 则称$b$为$a$的逆. 证明$a$的逆是唯一的.
  \item[2.] 设$R$是一个环,证明: $0a = a0 = 0, \forall a\in R$;
  $\forall a, b \in R, a(-b)=-ab$.  
  \item[3.] 设$A\in M_n(C)$, 设
  $$|\lambda I - A| = (\lambda - \lambda_1)^{l_1}(\lambda - \lambda_2)^{l_2}\cdots(\lambda - \lambda_s)^{l_s},$$
  其中$\lambda_1,\cdots,\lambda_s$是两两不同的复数,$l_1 + l_2 + \cdots + l_s = n$.
  证明,对于$k \in C$, $k \ne 0$, 矩阵$kA$的特征多项式为
  $$|\lambda I - kA| = (\lambda - k\lambda_1)^{l_1}(\lambda - k\lambda_2)^{l_2}\cdots(\lambda - k\lambda_s)^{l_s},$$
  $A^3$的特征多项式为
  $$|\lambda I - A^3| = (\lambda - \lambda_1^3)^{l_1}(\lambda - \lambda_2^3)^{l_2}\cdots(\lambda - \lambda_s^3)^{l_s}.$$
\end{itemize}

\vspace{0.5cm}
\paragraph{7.2}
\begin{itemize}
  \item 整除关系
  \item 带余除法; 主要思路:对被除式的次数用数学归纳法
\end{itemize}
\paragraph{例题}
\begin{itemize}
  \item[1.] 设$d,n\in N^*$, 则$K[x]$中,$x^d-1|x^n-1 \Leftrightarrow d|n$.
\end{itemize}

\vspace{0.5cm}
\paragraph{7.3}
\begin{itemize}
  \item 最大公因式: $K[x]$中任意两个多项式都有最大公因式,且可以表示为$f,g$的和式;
  \item 互素
\end{itemize}
\paragraph{例题}
\begin{itemize}
  \item[1.] 设$f(x),g(x) \in K[x]$, $a,b,c,d\in K$而且$ad-bc\ne 0$.证明,
  $(af(x)+bg(x), cf(x)+dg(x)) = (f(x),g(x))$.
  \item[2.] 设$A\in M_n(K)$, $f(x),g(x)\in K[x]$, 证明, 如果$d(x) = (f(x), g(x))$, 那么
  $d(A)x=0$的解空间是$f(A)x=0$解空间和$g(A)x=0$解空间的交.
  \item[3.] 在$K[x]$中, 如果$(f(x),g(x))=1$, 并且$deg\,f >0, deg\,g>0$, 那么在
  $K[x]$中存在唯一的多项式$u(x), v(x)$, $deg\, u< deg\, g, \,deg\, v<deg\, f$,
  s.t.
  $$u(x)f(x) +v(x)g(x) = 1.$$
  \item[4.] 在$K[x]$中, 如果$(f(x),g(x))=d(x)$, 那么在
  $K[x]$中存在唯一的多项式$u(x)$和$v(x)$, $deg\,u < deg\,g - deg\,d,\,
  deg\,v<deg\,f - deg\,d$,
  s.t.
  $$u(x)f(x) +v(x)g(x) = d(x).$$
  \item[5.] 在$K[x]$中的两个非零多项式$f(x)$和$g(x)$不互素的充分必要条件是,存在两个非零多项式
  $u(x),v(x)$, s.t. $u(x)f(x) = v(x)g(x)$且$deg\,u <deg\,g$, $deg\,v<deg\,f$.
  \item[6.] 在$K[x]$中,$f_1(x),f_2(x),\cdots,f_s(x)$两两互素,证明对于任意的
  $r_1(x),r_2(x),\cdots, r_s(x)\in K[x]$, 同余方程组
  \begin{equation}
  \nonumber
  \left\{
    \begin{aligned}
    g(x) &\equiv r_1(x) \quad \mathrm{mod}\, f_1(x)\\
    g(x) &\equiv r_2(x) \quad \mathrm{mod}\, f_2(x)\\
    \vdots&\\
    g(x) &\equiv r_s(x) \quad \mathrm{mod}\, f_s(x)\\
    \end{aligned}
  \right.
  \end{equation}
  在$K[x]$中有解,且若$c(x),d(x)$均为解,则$c(x) \equiv d(x),\,\,\mathrm{mod}\,f_1 f_2\cdots f_s$.
\end{itemize}

\vspace{0.5cm}
\paragraph{7.4}
\begin{itemize}
  \item 不可约多项式
  \item 唯一因式分解定理;这里唯一性的意义是?
\end{itemize}
\paragraph{例题}
\begin{itemize}
  \item[1.] 在$K[x]$中,设$(f,g_i)=1,i=1,2$. 证明$(fg_1,g_2)=(g_1,g_2)$.
  \item[2.] 在$K[x]$中, 证明对于任意的正整数$m$,有$(f^m(x), g^m(x)) = (f(x),g(x))^m$.
  \item[3.] 分别在复数域、实数域和有理数域上分解 $x^4+1$为不可约多项式的乘积 
\end{itemize}