\paragraph{通向Jordan标准型的两条路}
\begin{itemize}
    \item[(1)] $\lambda$矩阵
        \begin{itemize}
            \item[] 不变因子,行列式因子,初等因子
        \end{itemize} 
    \item[(2)] 不变子空间直和分解: 
        \begin{itemize}
            \item[] 最小多项式,特征多项式
        \end{itemize}
\end{itemize}

\paragraph{两条路的一点关联}
\begin{itemize}
    \item Jordan标准型中Jordan块的最小多项式即是不变因子.
    \item n级矩阵的最小多项式等于其最后一个不变因子$d_n(\lambda)$,
          特征多项式等于最后一个行列式因子$D_n(\lambda)$.
    % \item 数域$K$上的两矩阵相似当且仅当把它们看成复矩阵后相似.
\end{itemize}

\paragraph{关于有理标准形的推广}
若最小多项式无法在$F[x]$上分解为一次因子的幂次乘积?
\paragraph{不变子空间直和分解}
$C_j = A|W_j$是$n_j$维线性空间$W_j$上的线性变换,其最小多项式为$p_j^{l_j}(\lambda)$,
其中$p_j(\lambda)$是不可约多项式.
\begin{itemize}
    \item [(1)] 若存在$\alpha \in W_j$和正整数$t$,使得
    $\alpha, C_j\alpha, \cdots, C_j^{t-1} \alpha$线性无关,
    且$C_j^t \alpha$可以由$\alpha, C_j\alpha, \cdots, C_j^{t-1} \alpha$
    线性表出, 则称$\alpha, C_j\alpha, \cdots, C_j^{t-1} \alpha$
    张成了由$\alpha$生成的$C_j$-循环子空间.

    $W_j$能分解为$\frac{1}{r} \dim \widehat{W}_j$个$C_j$-循环子空间的直和,
    其中$r=\deg p(\lambda)$, $\widehat{W}_j = \ker p(C_j)$.
    \vspace{1cm}
    \item [(2)] 证明$W_j$中存在一个基,使得$C_j$在此基下的矩阵为
    有理块组成的块对角矩阵, 每个有理块级数是$r$的倍数,且不超过$rl_j$,
    有理块的总数为$\frac{1}{r}\left(n_j - \rank\, p_j(C_j)\right)$.
    $rt$级有理块的个数$N(rt)$为,
    $$N(rt) = \frac{1}{r} \left(
        \rank\,p^{t-1}(C_j) + \rank\,p^{t+1}(C_j) 
        - 2\rank\,p^{t}(C_j).
    \right)$$
    \vspace{1cm}
\end{itemize}

\paragraph{$\lambda$-矩阵}
\begin{itemize}
    \item[(1)] 
     $A(\lambda)$的每个次数大于零的不变因子在$F[x]$上做不可约分解,所有这些不可约多项式的方幂
     (相同的按照出现的次数计算)称为$A(\lambda)$的初等因子.
    \item[(2)] 由有理块组成的块对角矩阵的初等因子由它全部有理块的初等因子组成,
    于是矩阵的有理标准型由它的初等因子唯一确定(除去有理块的排列次序).  
    \item[(3)] 设域$E$包含域$F$, 域$F$上的两矩阵相似当且仅当把他们当成$E$上的矩阵相似. 
\end{itemize}

\paragraph{例题}
$A$是域$F$上线性空间$V$上的线性变换, 且$A$的最小多项式在$F[\lambda]$中的不可约分解为
$$m(\lambda) = (\lambda-\lambda_1)^{l_1}(\lambda-\lambda_2)^{l_2}\cdots(\lambda-\lambda_s)^{l_s}$$
\begin{itemize}
\item[(1)] 令$V_i = \{\alpha \in V |\,\exists r > 0,
(\lambda_i I - A)^r \alpha = 0\}, i = 1,2,\cdots,s.$
说明$V_i$是A不变的,且$V = V_1 \oplus V_2 \oplus \cdots \oplus V_s$ 
\begin{solution}
说明$V_i = W_i$
\end{solution}
% \item[(2)] 若当$i \ne j$时,$\lambda_i \ne \lambda_j$,
% 证明$A$的Jordan标准型恰好由$s$个Jordan块组成当且仅当$\dim V_{\lambda_j} = 1, j = 1,\cdots,s.$
\item[(2)] 
\begin{itemize}
    \item[a.] 证明: A的Jordan标准型中主对角元为$\lambda_i$的Jordan块的个数为
    $N_j = \dim V - \rank (A - \lambda_j I)$.
    其中$J_t(\lambda_j)$的个数为$N_j(t) = \rank (A-\lambda_j I)^{t+1} +
    \rank (A-\lambda_j I)^{t-1} - 2\rank (A-\lambda_j I)^t$.

    \begin{solution}
        前面已说明$N_j = \dim \Ker (A|W_j - \lambda_j I)$以及
        $N_j(t) = \rank (A|W_j-\lambda_j I)^{t+1} +
        \rank (A|W_j-\lambda_j I)^{t-1} - 2\rank (A|W_j-\lambda_j I)^t$,
        说明对于$i=1,2,\cdots, l_j+1$,
        $$\Ker (A-\lambda_j I)^i = \Ker(A|W_j - \lambda_j I)^i$$即可.
    \end{solution}

    \item[b.] 证明:A可对角化,当且仅当$$ \rank (A - \lambda_iI)^2 = 
    \rank (A - \lambda_iI),\ i=1,\cdots,s. $$ 
    \begin{solution}
    \begin{itemize}
        \item [(a)]
        $\rank(\lambda_iI - A) = \rank(\lambda_i I - A)^2 =\cdots =\rank(\lambda_i - A)^{l_i}$
        即特征空间$V_{\lambda_i}$等同于$W_i$,故 
        $V = V_{\lambda_1} \oplus \cdots \oplus V_{\lambda_s}.$
        \item [(b)] 说明$N_j = N_j(1)$, 即$\dim V - \rank(A-\lambda_j I) = 
        \rank(A-\lambda_j I)^0 + \rank(A-\lambda_j I)^2 - 2\rank(A-\lambda_j I)$即可.
    \end{itemize}
    \end{solution}
\end{itemize}

\item[(3)]如果$\forall i,\ \lambda_i = 1,$ 证明$A$和$A^k\,(k\ge 1)$相似
\begin{solution}
\end{solution}
\vspace{3cm}

\item[(4)] 如果$l_1 + l_2 +\cdots l_s = \dim V$, 说明A的Jordan标准型中各个Jordan块主对角元互不相同. 
\vspace{3cm}
\end{itemize}

\paragraph{Jordan标准型的应用}
\begin{itemize}
    \item [1. ] 计算矩阵多项式
    \item [2. ] 计算矩阵幂级数,特别是指数函数
    $$e^{A} = \sum_{i=0}^{\infty} \frac{A^i}{i!}$$
    及其在求解ODE中的应用.
    \begin{itemize}
    %     \item [(1)] 设$f(t) = \sum_{k=0}^{\infty} a_k t^k$, 收敛半径为$r$,
    %     设矩阵$A$的Jordan标准型为
    %     \begin{equation}
    %     \nonumber
    %     J =
    %     \begin{pmatrix}
    %         J_{n_1}(\lambda_1)&&&\\
    %         &J_{n_2}(\lambda_2)&&\\
    %         &&\ddots&\\
    %         &&&J_{n_s}(\lambda_s)\\
    %     \end{pmatrix}
    %     \end{equation}
    %     即$A = PJP^{-1}$, 若$\forall i$, $|\lambda_i| < r$,
    %     即$\rho(A) < r$, 则矩阵幂级数$\sum_{k=0}^{\infty} a_k A^k$收敛,
    %     $$f(A) = P
    %     \begin{pmatrix}
    %         f(J_{n_1}(\lambda_1))&&&\\
    %         &f(J_{n_2}(\lambda_2))&&\\
    %         &&\ddots&\\
    %         &&&f(J_{n_s}(\lambda_s))\\
    %     \end{pmatrix}
    %     $$
    %     其中$f(J_{n_i}(\lambda_i))$该如何表达?
    %     \vspace{3cm}

        \item [(a)]齐次一阶线性常系数常微分方程组
        \begin{equation}
        \nonumber
        \left\{
        \begin{aligned}
        &\frac{\mathrm{d} x}{\mathrm{d} t} = Ax,\\
        &x(0) = x_0.
        \end{aligned}
        \right.
        \end{equation}
        的唯一解为$x(t)=e^{At}x_0$. 例如求解
        \begin{equation}
            \nonumber
            \frac{\mathrm{d} x}{\mathrm{d} t} = 
            \begin{pmatrix}
                2& 1& 4\\
                0& 2& 0\\
                0& 3& 1\\
            \end{pmatrix}
            x
        \end{equation}
        的通解.
        \vspace{3cm}
        \item [(b)]齐次高阶线性常系数常微分方程
        \begin{equation}
        \nonumber
        \left\{
        \begin{aligned}
        &x^{(n)} + a_{n-1} x^{(n-1)} + \cdots + a_1 x^{(1)} + a_0 x = 0\\
        &x(0) = x_0\\
        &x^{(1)}(0) = x_0^{(1)}\\
        &\vdots\\
        &x^{(n-1)}(0) = x_0^{(n-1)}\\
        \end{aligned}
        \right.
        \end{equation}
        其中$x^{(i)}(t) = \frac{\mathrm{d}^i x}{\mathrm{d} t^i}(t).$ 可以转化为方程组情形.
        例如求解$x^{(3)} - 3x^{(2)}-6x^{(1)} + 8x = 0$的通解.
        \vspace{3cm}
    \end{itemize}
    \item [2. ] 计算矩阵平方根
    \begin{itemize}
        \item[(1)] 设$a$是域F中的非零元, 求$J_r(a)^2$的标准型.
        \begin{solution}
            由于
            $J_r(a)^2 = (aI + J_r(0))^2
                      = a^2 I + 2aJ_r(0) + J_r(0)^2$,
            $\rank J_r(a)^2 = r-1$, 所以$\rank J_r(a)^2 \sim J_r(a^2)$.
        \end{solution}
        \vspace{3cm}
        \item[(2)] 任意的可逆复矩阵都有平方根.
        \vspace{3cm}
        \item[(3)]
        \begin{equation}
        \nonumber
            A = 
            \begin{pmatrix}
               -2& 1& 0\\
               -4& 2& 0\\
               -2& 1& 1\\
            \end{pmatrix}
        \end{equation} 
        是否有平方根,若有给出一个.
        \vspace{3cm}
        \item[(4)] 证明:不可逆的复矩阵有平方根,当且仅当其标准型中主对角元为0的
        Jordan块或是$J_1(0)$,或是$J_r(0), J_{r}(0)$成对出现,或是$J_r(0), J_{r+1}(0)$成对出现. 
        \vspace{3cm}
    \end{itemize}
\end{itemize}