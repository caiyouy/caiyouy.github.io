\documentclass[cn,10pt,
               result=answer,
               math=cm,
            %    chinesefont=founder,
               citestyle=gb7714-2015,bibstyle=gb7714-2015]{elegantbook}

\title{线性代数A 习题课讲义集}
\author{Caiyou Yuan}
% \institute{Elegant\LaTeX{} Program}
\date{October 5, 2021}
% \version{0.1}
% \bioinfo{自定义}{信息}
% \extrainfo{各人自扫门前雪,休管他人瓦上霜。—— 陈元靓}
\setcounter{tocdepth}{3}

% \logo{logo-blue.png}
\cover{The_four_subspaces.jpg}

% 本文档命令
\usepackage{array}
\newcommand{\ccr}[1]{\makecell{{\color{#1}\rule{1cm}{1cm}}}}
\definecolor{customcolor}{RGB}{32,178,170}
\colorlet{coverlinecolor}{customcolor}

\newcommand{\rank}[1]{\mathrm{rank}(#1)}

\begin{document}

\maketitle
\frontmatter

\chapter*{特别声明}
\markboth{Introduction}{前言}
这里主要整理了作者在担任北京大学线性代数A这门课程的助教期间,
在习题课上讲解的一些题目,
主要参考了,
\begin{enumerate}
  \item 高等代数(第三版)丘维声,高等教育出版社
  \item 高等代数学习指导书(第二版)丘维生,清华大学出版社
\end{enumerate}
以及一些其他书籍。

作者能力有限,难免有考虑不周和疏漏之处,欢迎批评指正。

\begin{flushright}
Caiyou Yuan\\
October 5, 2021
\end{flushright}

\tableofcontents

\mainmatter
\section{线性方程组}
\begin{itemize}
    \item 矩阵的初等行变换
    \item 阶梯形矩阵,简化阶梯形矩阵
    \item Guass-Jordan算法, 无解/有唯一解/有无数解
    \item 数域, $Q,R,C$
\end{itemize}

\subsection*{例题}
\begin{itemize}
    \item[1.] 证明任意一个矩阵都可以经过一系列初等行变换化为(简化)阶梯形矩阵
    \vspace{2.5cm}

    \item[2.] 解如下线性方程组
    \begin{equation}
    \nonumber
        \left\{
        \begin{aligned}
            &x_1 + x_2 + \cdots + x_n = 1\\
            &x_2 + x_3 + \cdots + x_{n+1} = 2\\
            &x_3 + x_4 + \cdots + x_{n+2} = 3\\
            &\vdots\\
            &x_{n+1}+x_{n+2}+ \cdots + x_{2n} = n+1\\
        \end{aligned}
        \right.
    \end{equation} 
    \vspace{2.5cm}

    \item[3.] 解如下线性方程组
    \begin{equation}
    \nonumber
        \left\{
        \begin{aligned}
            &x_1 +2x_2 +\cdots +(n-1)x_{n-1} +nx_{n} =b_1\\
            &nx_1 +x_2 +\cdots +(n-2)x_{n-1} +(n-1)x_{n} =b_2\\
            &(n-1)x_1 +nx_2 +\cdots +(n-3)x_{n-1} +(n-2)x_{n} =b_3\\
            &\vdots\\
            &2x_1 +3x_2 +\cdots +nx_{n-1} +x_{n} =b_n\\
        \end{aligned}
        \right.
    \end{equation} 
    其中$b_1,\cdots b_n$为给定常数
    \vspace{2.5cm}
\end{itemize}

\section{行列式}
\subsection{n元排列}
\begin{itemize}
    \item 逆序数
    \item 奇/偶排列
\end{itemize}

\subsection*{例题}
这里使用$\tau(a_1a_2\cdots a_k)$表示排列$a_1a_2\cdots a_k$的逆序数.
\begin{itemize}
    \item[1.] 设$1,2,\cdots,n$的n元排列
    $a_1a_2\cdots a_k b_1b_2\cdots b_{n-k}$
    有$$\tau(a_1a_2\cdots a_k) = \tau(b_1b_2\cdots b_{n-k}) = 0$$
    那么$\tau(a_1a_2\cdots a_k b_1b_2\cdots b_{n-k})$是多少?
    $n,k,a_1,\cdots,a_k, b_1, \cdots, b_{n-k}$
    均已知.
    \vspace{2.5cm}

    \item[2.] 说明n(n>1)元排列中,奇偶排列各占一半
    \vspace{1.5cm}

    \item[3.]
    \begin{itemize}
        \item[(1)] 若$\tau(a_1a_2\cdots a_n) = r$, 那么
        $\tau(a_na_{n-1}\cdots a_1)$ 是多少?
        \vspace{1.5cm}
        \item[(2)] 计算所有n元排列的逆序数之和
        \vspace{1.5cm}
    \end{itemize} 
\end{itemize}

\subsection{n阶行列式的定义}
\begin{equation}
\nonumber
\begin{aligned}
\det{A} &= \sum_{j_1 j_2\cdots j_n} (-1)^{\tau(j_1 j_2\cdots j_n)}a_{1j_1}a_{2j_2}\cdots a_{nj_n}\\
        &= \sum_{i_1 i_2\cdots i_n} (-1)^{\tau(i_1 i_2\cdots i_n)}a_{i_11}a_{i_22}\cdots a_{i_nn}
\end{aligned}
\end{equation}

\subsection*{例题}
\begin{itemize}
    \item[1.] 下列行列式是$x$的几次多项式,求出$x^4$项和$x^3$项的系数
    \begin{equation}
    \nonumber
    \left|
        \begin{array}{rrrr}
        5x &x &1 &x\\
        1  &x &1 &-x\\
        3  &2 &x &1\\
        3  &1 &1 &x\\
        \end{array}
    \right|
    \end{equation} 
    \vspace{2.5cm}

    \item[2.] 计算下列行列式
    \begin{equation}
    \nonumber
    \left|
        \begin{array}{rrrrr}
        a_1 &a_2 &a_3 &\cdots &a_n\\
        b_2 &1   &0   &\cdots &0\\
        b_3 &0   &1   &\cdots &0\\
        \vdots &\vdots   &\vdots   & &\vdots\\
        b_n &0   &0   &\cdots &1\\
        \end{array}
    \right|
    \end{equation}
    \vspace{2.5cm}
\end{itemize}

\subsection{行列式的性质}
\subsection*{例题}
\begin{itemize}
    \item 计算下列行列式
    \begin{equation}
    \nonumber
    \left|
        \begin{array}{ccccc}
        x_1-a_1 &x_2     &x_3     &\cdots & x_n\\
        x_1     &x_2-a_2 &x_3     &\cdots & x_n\\
        x_1     &x_2     &x_3-a_3 &\cdots & x_n\\
        \vdots  &\vdots  &\vdots  &       & \vdots\\
        x_1     &x_2     &x_3     &\cdots & x_n-a_n\\
        \end{array}
    \right|
    \end{equation} 
    其中$a_i \ne 0, i = 1,\cdots,n.$
    
    \begin{solution}
    (方法一)第一行的-1倍加到2到n行,化为箭形矩阵行列式;
    (方法二)非对角元素补上减去0,每列拆分为两列, $2^n$个行列式中经有n个非零;
    (方法三)和方法二类似,但只对第一列拆分, 找到n阶结果和n-1阶结果的关系
    \end{solution}
    \vspace{2cm}
\end{itemize}

\subsection{行列式按一行(列)展开}
\begin{itemize}
    \item 代数余子式, 行列式按行展开
    \item Vandermonde行列式 
\end{itemize}

\subsection*{例题}
\begin{itemize}
    \item[1.] 计算下列行列式
    \begin{itemize}
        \item[(1)]
        \begin{equation}
        \nonumber
        \left|
            \begin{array}{cccccccc}
            2a      &a^2     &0       &0      &\cdots & 0 &0 &0\\
            1       &2a      &a^2     &0      &\cdots & 0 &0 &0\\
            0       &1       &2a      &a^2    &\cdots & 0 &0 &0\\
            \vdots  &\vdots  &\vdots  &\vdots &       & \vdots & \vdots &\vdots\\
            0       &0       &0       &0      &\cdots & 1 &2a &a^2\\
            0       &0       &0       &0      &\cdots & 0 &1 &2a\\
            \end{array}
        \right|
        \end{equation} 
        \vspace{2cm}

        \item[(2)]
        \begin{equation}
        \nonumber
        \left|
            \begin{array}{cccccccc}
            a+b      &ab     &0       &0      &\cdots & 0 &0 &0\\
            1       &a+b     &ab     &0      &\cdots & 0 &0 &0\\
            0       &1       &a+b      &ab    &\cdots & 0 &0 &0\\
            \vdots  &\vdots  &\vdots  &\vdots &       & \vdots & \vdots &\vdots\\
            0       &0       &0       &0      &\cdots & 1 &a+b &ab\\
            0       &0       &0       &0      &\cdots & 0 &1 &a+b\\
            \end{array}
        \right|
        \end{equation} 
        \vspace{2cm}

        \item[(3)]
        \begin{equation}
        \nonumber
        \left|
            \begin{array}{cccccccc}
            a       &b     &0       &0      &\cdots & 0 &0 &0\\
            c       &a     &b     &0      &\cdots & 0 &0 &0\\
            0       &c     &a      &b    &\cdots & 0 &0 &0\\
            \vdots  &\vdots  &\vdots  &\vdots &       & \vdots & \vdots &\vdots\\
            0       &0       &0       &0      &\cdots & c &a &b\\
            0       &0       &0       &0      &\cdots & 0 &c &a\\
            \end{array}
        \right|
        \end{equation} 
        \vspace{2cm}
    \end{itemize}

    \item[2.] 计算如下行列式
    \begin{equation}
    \nonumber
    \left|
        \begin{array}{ccccc}
        1       &1       &\cdots &1 & 1\\
        x_1     &x_2     &\cdots &x_{n-1} & x_n\\
        x_1^2   &x_2^2   &\cdots &x_{n-1}^2 & x_n^2\\
        \vdots  &\vdots  &       &\vdots & \vdots\\
        x_1^{n-2} &x_2^{n-2}     &\cdots &x_{n-1}^{n-2} & x_n^{n-2}\\
        x_1^{n} &x_2^{n}     &\cdots &x_{n-1}^{n} & x_n^{n}\\
        \end{array}
    \right|
    \end{equation} 
\end{itemize}
\paragraph{7.5等}
\begin{itemize}
    \item 重因式
    \item 复数域上的不可约多项式只有一次的
    \item 实数域上的不可约多项式都是一次的,或判别式小于零的二次多项式
    \item 有理数域上的不可约多项式可以是任意次数的
\end{itemize}

\paragraph{多项式理论的应用: $\lambda$-矩阵}
即矩阵的每个元素都是多项式环$K[\lambda]$中元素;矩阵乘法,加法以及行列式等概念,和数字矩阵类似

\paragraph{$\lambda$-矩阵的初等变换}
\begin{itemize}
  \item [(a)] 矩阵的两行/列互换位置
  \item [(b)] 矩阵的某一行/列乘以非零常数c
  \item [(c)] 矩阵的某一行/列加上另一行/列的$p(\lambda)$倍,其中$p(\lambda) \in K[\lambda]$.
\end{itemize}
如果$A(\lambda)$可以经过一系列行和列的初等变换化为$B(\lambda)$,
则称$A(\lambda)$和$B(\lambda)$等价.

\paragraph{例题}
\begin{itemize}
  \item[1.] 证明: 设$A(\lambda)$的左上角元素$a_{11}(\lambda) \ne 0$, 并且$A(\lambda)$中
至少有一个元素不能被它除尽,那么一定可以找到和$A(\lambda)$等价的$B(\lambda)$,
它的左上角元素也不为零,但是次数小于$a_{11}(\lambda)$的次数
  \vspace{3cm}
  \item[2.] 证明:任意一个非零的$s\times n$的$\lambda$-矩阵$A(\lambda)$都等价于如下形式的矩阵(被称为标准形式)
\begin{equation}
\nonumber
\begin{pmatrix}
  d_1(\lambda)&&&&&&\\
  &d_2(\lambda)&&&&&\\
  &&\ddots&&&&\\
  &&&d_r(\lambda)&&&\\
  &&&&0&&\\
  &&&&&\ddots&\\
  &&&&&&0\\
\end{pmatrix}
\end{equation}
其中$r\ge 1$, $d_i(\lambda)$是首一的多项式(被称为不变因子),且
$$d_i(\lambda)\,|\,d_{i+1}(\lambda)\quad i=1,2,\cdots, r-1.$$
\vspace{4cm}
\item[3.]用初等变换化$\lambda$-矩阵
\begin{equation}
\nonumber
A(\lambda) =
\begin{bmatrix}
1-\lambda& 2\lambda-1& \lambda\\
\lambda&   \lambda^2&  -\lambda\\
1+\lambda^2& \lambda^3+\lambda-1& -\lambda^2\\
\end{bmatrix}
\end{equation}
为标准型.
\vspace{3cm}
\end{itemize}

\paragraph{$\lambda$-矩阵标准形式的唯一性}
下面来借助行列式因子的概念来说明$\lambda$-矩阵标准形式的唯一性.
\begin{itemize}
  \item[1.] ($\lambda$-矩阵的秩)如果$A(\lambda)$中有一个$r\ge 1$级子式不为零,而所有的$r+1$级子式(如果有的话)全为零,
则称$A(\lambda)$的秩为$r$. 零矩阵的秩记为0.
  \item[2.] ($\lambda$-矩阵的行列式因子) 设$A(\lambda)$的秩为r,对于$1 \le k \le r$, $A(\lambda)$中全部$k$级子式的首一最大公因式
$D_{k}(\lambda)$称为$A(\lambda)$的$k$阶行列式因子.
\end{itemize}

\paragraph{例题}
\begin{itemize}
  \item[1.] 等价的$\lambda$-矩阵具有相同的秩和相同的各阶行列式因子.
  \vspace{2cm}
  \item[2.] 证明$\lambda$-矩阵的不变因子和行列式因子有如下关系
  $$
  d_1(\lambda) = D_1(\lambda),\quad
  d_2(\lambda) = \frac{D_2(\lambda)}{D_1(\lambda)},\quad
  \dots, \quad d_r = \frac{D_r(\lambda)}{D_{r-1}(\lambda)}
  $$ 
  \vspace{3cm}
  \item[3.] 说明$\lambda$-矩阵的标准形式是唯一的.
  \vspace{2cm} 
\end{itemize}

\paragraph{8.1}
\begin{itemize}
    \item 域
    \item 域$F$上的线性空间的定义
    \item 基
    \item 维数
    \item 过渡矩阵
\end{itemize}

\paragraph{例题}
\begin{itemize}
  \item[1.]
  \begin{itemize}
    \item [(a)] 把域$F$看成是$F$上的线性空间,求它的一个基和维数;
    \item [(b)] 把复数域$C$看成是实数域$R$上的线性空间,求它的一个基和维数;
  \end{itemize}
\end{itemize}
\vspace{1cm}

\begin{itemize}
  \item[2.]
  \begin{itemize}
    \item [(a)] 把实数域$R$看成是有理数域$Q$上的线性空间,证明:对于任意大于1的正整数,
  $$1, \sqrt[n]{3}, \sqrt[n]{3^2}, \cdots, \sqrt[n]{3^{n-1}}$$
  是线性无关的.\,(提示: 已知$g(x) = x^n-3$是Q上的不可约多项式)
    \item [(b)] 证明: 实数域$R$作为有理数域$Q$上的线性空间是无穷维的.
  \end{itemize}
\end{itemize}
\vspace{1.5cm}

\begin{itemize}
  \item[3.] 设$V$是域$F$上的n维线性空间,域$F$包含域$E$,\,$F$可看作域$E$
  上的$m$维线性空间
  \begin{itemize}
    \item [(a)] 求证:V可以成为域$E$上的线性空间
    \item [(b)] 证明: 求V作为域E上线性空间的维数
  \end{itemize}
\end{itemize}
\vspace{1.5cm}

\begin{itemize}
  \item[4.] (Complexification of real vector space)
  设$V$是数域R上的$n$维线性空间,设$V_C = \{(u,v), u,v\in V\}$,
  定义$V_C$上的加法
  $$(u_1,v_1) +(u_2, v_2) = (u_1+u_2, v_1+v_2)$$
  以及C上的数乘
  $$(a+bi)(u,v)=(au-bv, av+bu)$$
  \begin{itemize}
    \item [(a)] 求证:$V_C$是一个复线性空间
    \item [(b)] 计算$V_C$的维数
  \end{itemize}
\end{itemize}
\vspace{1cm}

\begin{itemize}
  \item[5.] (对偶空间)设$V$是数域K上的$n$维线性空间,
  考虑复数域$C$上的线性空间$C^V$(从$V$到$R$的函数全体)中具有下述性质的函数组成的子集$W$:
  $$f(\alpha+\beta)=f(\alpha) + f(\beta),\quad \forall \alpha, \beta \in V,$$
  $$f(k\alpha)=kf(\alpha),\quad \forall \alpha \in V, k \in K.$$
  \begin{itemize}
    \item [(a)] 求证:$W$是一个复线性空间
    \item [(b)] 求$W$的一个基和维数;设$f\in W$, 求$f$在这个基下的坐标
  \end{itemize}
\end{itemize}
\vspace{1cm}

\begin{itemize}
  \item[6.] (零化多项式和最小多项式)
  设A是数域K上的一个非零n阶矩阵,说明$K[A]$是$K$上的一个线性空间.
  $K[A]$至多多少维?\footnote{Hamilton-Cayley定理告诉我们,K[A]至多n维.}
\end{itemize}
\vspace{1cm}

\begin{itemize}
  \item[7.] 设递推方程
  $$u_n = au_{n-1} +bu_{n-2}, \quad n \ge 2,$$
  其中$a,b$都是非零复数. 若N上的一个复值数列$u_n$满足上述递推关系,则称为上述递推方程的解.
  一元多项式$f(x)=x^2 - ax -b$称为上述递推方程的特征多项式.
  求证
  \begin{itemize}
    \item [(a)] 上述递推方程的解集$W$是一个复线性空间
    \item [(b)] 设$\alpha$是一个非零复数,则$\alpha^n \in W$ 当且仅当$f(\alpha)=0$
    \item [(c)] 设$\alpha$是一个非零复数,则$n\alpha^n \in W$ 当且仅当$f(\alpha)=0, f'(\alpha)=0$.
    \item [(d)] 若$f(x)$有两不同的根$\alpha_1, \alpha_2$, 则任意$u_n \in W$, 可以表示为
    $$ u_n = C_1 \alpha_1^n +C_2 \alpha_2^n$$
    其中$C_1, C_2$是常数;
    \item [(e)] 若$f(x)$有二重根$\alpha$, 则任意$u_n \in W$, 可以表示为
    $$ u_n = C_1 \alpha^n +C_2 n\alpha^n$$
    其中$C_1, C_2$是常数;
  \end{itemize}
\end{itemize}
\vspace{6cm}


\paragraph{8.2}
\begin{itemize}
    \item 子空间
    \item 子空间的维数定理
    $$\dim V_1 + \dim V_2 = \dim (V_1 + V_2) + \dim (V_1 \cap V_2)$$
    \item 直和
\end{itemize}

\paragraph{例题}
\begin{itemize}
\item[1.] 设$V_1, V_2, V_3$都是域F上的有限维线性空间$V$的子空间
\begin{itemize}
    \item[(a)] 求证:
    $$V_3 \cap V_1 + V_3 \cap V_2 \subset V_3 \cap (V_1 + V_2)$$ 
    \item[(b)] 如果$V_3 \subset V_1 + V_2$,
试问$(V_3\cap V_1) + (V_3\cap V_2) = V_3$是否总成立?如果再加上条件$V_1 \subset V_3$呢?
    \item[(c)] 求证,
    \begin{equation}
    \nonumber
    \begin{aligned}
    \dim V_1 + \dim V_2 + \dim V_3 \ge
    &\dim(V_1 + V_2 + V_3)\\
    &+\dim(V_1 \cap V_2) +\dim(V_1 \cap V_3) +\dim(V_2 \cap V_3)\\
    &-\dim(V_1 \cap V_2 \cap V_3)
    \end{aligned}
    \end{equation}  
  \end{itemize}
\end{itemize}
\vspace{3cm}

\begin{itemize}
  \item[2.] 设$V_1, \cdots, V_s$都是域F上线性空间V的子空间,
  证明$V_1 + V_2 + \cdots + V_s$是直和
  \begin{itemize} 
    \item[(a)] 当且仅当,
    $$V_i \cap \left(\sum_{j\ne i} V_j \right) = 0, \quad i=1,2,\dots,s$$
    \item[(b)] 当且仅当,$V$中有一向量$\alpha$可以唯一表示为
    $$\alpha = \sum_{i=1}^s \alpha_i, \quad \alpha_i \in V_i$$
    \item[(c)] 当且仅当,
    $$\dim(V_1 + V_2 + \cdots + V_s) = \dim V_1 + \cdots \dim V_s$$ 
  \end{itemize}
\end{itemize}
\vspace{3cm}

\begin{itemize}
  \item[3.] 设A是数域K上的一个n阶矩阵,$\lambda_1, \lambda_2, \cdots, \lambda_s$
  是A的全部不同的特征值,用$V_{\lambda_i}$表示A的属于$\lambda_i$的特征子空间. 证明:
  A可以对角化的充分必要条件是
  $$K^n = V_{\lambda_1} \oplus V_{\lambda_2} \oplus \cdots \oplus V_{\lambda_s}$$
\end{itemize}
\vspace{1cm}

\begin{itemize}
  \item[4.] 在数域K上的线性空间$K^{M_n(K)}$中,如果$f$满足,
  对于任意的$A=(\alpha_1, \alpha_2,\cdots, \alpha_n)$,任意的n维列向量$\alpha$, 以及任意$k\in K$, $j\in \{1,2,\cdots, n\}$,
  有 
  $$f(\alpha_1, \cdots, \alpha_{j-1}, \alpha_{j} + \alpha, \alpha_{j+1}, \cdots, \alpha_n)
  = f(\alpha_1, \cdots, \alpha_{j-1}, \alpha_{j}, \alpha_{j+1}, \cdots, \alpha_n) + 
    f(\alpha_1, \cdots, \alpha_{j-1}, \alpha, \alpha_{j+1}, \cdots, \alpha_n)$$
  $$f(\alpha_1, \cdots, \alpha_{j-1}, k\alpha_{j}, \alpha_{j+1}, \cdots, \alpha_n)
  = kf(\alpha_1, \cdots, \alpha_{j-1}, \alpha_{j}, \alpha_{j+1}, \cdots, \alpha_n)
  $$
  那么称$f$是$M_n(K)$上的列线性函数. 同理如果$g(A^T)$是列线性函数,则称$g(A)$是行线性函数。
  记所有的列/行线性函数组成的集合分别记为$V_1$和$V_2$.
  \begin{itemize} 
    \item[(a)] 证明: $V_1, V_2$都是$K^{M_n(K)}$的子空间
    \item[(b)] 分别求$V_1, V_2$的一个基和维数
    \item[(c)] 分别求$V_1 \cap V_2,\,V_1 + V_2$的一个基和维数 
  \end{itemize}
\end{itemize}
\vspace{6cm}

\paragraph{8.3}
\begin{itemize}
    \item 线性空间的同构
    \item 有限维线性空间同构的充要条件
\end{itemize}

\paragraph{例题}
\begin{itemize}
  \item[1.]令
  \begin{equation}
  \nonumber
  H = \left\{
    \begin{bmatrix}
      z_1& z_2\\
     -z_2& z_1\\
    \end{bmatrix},
    z_1, z_2 \in C
      \right\}
  \end{equation}
  \begin{itemize}
    \item[(a)] H对于矩阵的加法,以及实数和矩阵的乘法构成一个实线性空间
    \item[(b)] 给出H的一个基和维数
    \item[(c)] 证明: $H$与$R^4$同构,并写出$H$到$R^4$的一个同构映射  
  \end{itemize}
\end{itemize}
\vspace{1cm}

\begin{itemize}
  \item[2.]设$A \in M_n(K)$, 令$AM_n(K) = \{AB, B\in M_n(K) \}$.
  \begin{itemize}
    \item[(a)] 证明$AM_n(K)$是数域$K$上线性空间$M_n(K)$的子空间
    \item[(b)] 设A的列向量组$\alpha_1, \cdots, \alpha_n$的一个极大线性无关组为
    $\alpha_{j_1}, \cdots, \alpha_{j_r}$,证明$AM_n(K)$和$M_{r\times n}(K)$同构,并
    写出一个同构映射. 
    \item[(c)] 证明: $\dim [AM_n(K)] = \rank(A)n$.
  \end{itemize}
\end{itemize}
\vspace{1cm}
\paragraph{8.1}
\begin{itemize}
    \item 域, 以及域$F$上的线性空间
    \item 基和维数
    \begin{itemize}
		\item[a.] 所有非零线性空间均有基
		\item[b.] 线性空间中的任意一组线性无关的向量可以扩充为基
	\end{itemize}
    \item 过渡矩阵
\end{itemize}

\paragraph{例题}
\begin{itemize}
  \item[1.]
  \begin{itemize}
    \item [(a)] 把域$F$看成是$F$上的线性空间,求它的一个基和维数;
    \item [(b)] 把复数域$C$看成是实数域$R$上的线性空间,求它的一个基和维数;
    \item [(c)] 把实数域$R$看成是有理数域$Q$上的线性空间,证明:对于任意大于1的正整数$n$,
  $$1, \sqrt[n]{3}, \sqrt[n]{3^2}, \cdots, \sqrt[n]{3^{n-1}}$$
  是线性无关的.\,(提示: 已知$g(x) = x^n-3$是Q上的不可约多项式)
    \item [(d)] 证明: 实数域$R$作为有理数域$Q$上的线性空间是无穷维的.
\end{itemize}
\end{itemize}
\vspace{2cm}

\begin{itemize}
  \item[2.] 设$V$是域$F$上的n维线性空间,域$F$包含域$E$,\,$F$可看作域$E$
  上的$m$维线性空间
  \begin{itemize}
    \item [(a)] 求证:V可以成为域$E$上的线性空间
    \item [(b)] 证明: 求V作为域E上线性空间的维数
  \end{itemize}
\end{itemize}
\vspace{1.5cm}

\begin{itemize}
  \item[3.] (Complexification of real vector space)
  设$V$是数域R上的$n$维线性空间,设$V_C = \{(u,v), u,v\in V\}$,
  定义$V_C$上的加法
  $$(u_1,v_1) +(u_2, v_2) = (u_1+u_2, v_1+v_2)$$
  以及C上的数乘
  $$(a+bi)(u,v)=(au-bv, av+bu)$$
  \begin{itemize}
    \item [(a)] 求证:$V_C$是一个复线性空间
    \item [(b)] 计算$V_C$的维数
  \end{itemize}
\end{itemize}
\vspace{1cm}

\begin{itemize}
  \item[4.] (对偶空间)设$V$是数域K上的$n$维线性空间,
  考虑复数域$C$上的线性空间$C^V$(从$V$到$R$的函数全体)中具有下述性质的函数组成的子集$W$:
  $$f(\alpha+\beta)=f(\alpha) + f(\beta),\quad \forall \alpha, \beta \in V,$$
  $$f(k\alpha)=kf(\alpha),\quad \forall \alpha \in V, k \in K.$$
  \begin{itemize}
    \item [(a)] 求证:$W$是一个复线性空间
    \item [(b)] 求$W$的一个基和维数;设$f\in W$, 求$f$在这个基下的坐标
  \end{itemize}
\end{itemize}
\vspace{2cm}

\begin{itemize}
  \item[5.] (零化多项式和最小多项式)
  设A是数域K上的一个非零n阶矩阵,说明$K[A]$是$K$上的一个线性空间.
  $K[A]$至多多少维?\footnote{Hamilton-Cayley定理告诉我们,K[A]至多n维.}
\end{itemize}
\vspace{1cm}

\begin{itemize}
  \item[6.] 设递推方程
  $$u_n = au_{n-1} +bu_{n-2}, \quad n \ge 2,$$
  其中$a,b$都是非零复数. 若N上的一个复值数列$u_n$满足上述递推关系,则称为上述递推方程的解.
  一元多项式$f(x)=x^2 - ax -b$称为上述递推方程的特征多项式.
  求证
  \begin{itemize}
    \item [(a)] 上述递推方程的解集$W$是一个复线性空间
    \item [(b)] 设$\alpha$是一个非零复数,则$\alpha^n \in W$ 当且仅当$f(\alpha)=0$
    \item [(c)] 设$\alpha$是一个非零复数,则$n\alpha^n \in W$ 当且仅当$f(\alpha)=0, f'(\alpha)=0$.
    \item [(d)] 若$f(x)$有两不同的根$\alpha_1, \alpha_2$, 则任意$u_n \in W$, 可以表示为
    $$ u_n = C_1 \alpha_1^n +C_2 \alpha_2^n$$
    其中$C_1, C_2$是常数;
    \item [(e)] 若$f(x)$有二重根$\alpha$, 则任意$u_n \in W$, 可以表示为
    $$ u_n = C_1 \alpha^n +C_2 n\alpha^n$$
    其中$C_1, C_2$是常数;
  \end{itemize}
\end{itemize}
\vspace{6cm}


\paragraph{8.2}
\begin{itemize}
    \item 子空间
    \item 子空间的维数定理
    $$\dim V_1 + \dim V_2 = \dim (V_1 + V_2) + \dim (V_1 \cap V_2)$$
    \item 直和
\end{itemize}

\paragraph{例题}
% \begin{itemize}
% \item[1.] 设$V_1, V_2, V_3$都是域F上的有限维线性空间$V$的子空间
% \begin{itemize}
%     \item[(a)] 求证:
%     $$V_3 \cap V_1 + V_3 \cap V_2 \subset V_3 \cap (V_1 + V_2)$$ 
%     \item[(b)] 如果$V_3 \subset V_1 + V_2$,
% 试问$(V_3\cap V_1) + (V_3\cap V_2) = V_3$是否总成立?如果再加上条件$V_1 \subset V_3$呢?
%     \item[(c)] 求证,
%     \begin{equation}
%     \nonumber
%     \begin{aligned}
%     \dim V_1 + \dim V_2 + \dim V_3 \ge
%     &\dim(V_1 + V_2 + V_3)\\
%     &+\dim(V_1 \cap V_2) +\dim(V_1 \cap V_3) +\dim(V_2 \cap V_3)\\
%     &-\dim(V_1 \cap V_2 \cap V_3)
%     \end{aligned}
%     \end{equation}  
%   \end{itemize}
% \end{itemize}
% \vspace{3cm}

% \begin{itemize}
%   \item[2.] 设$V_1, \cdots, V_s$都是域F上线性空间V的子空间,
%   证明$V_1 + V_2 + \cdots + V_s$是直和
%   \begin{itemize} 
%     \item[(a)] 当且仅当,
%     $$V_i \cap \left(\sum_{j\ne i} V_j \right) = 0, \quad i=1,2,\dots,s$$
%     \item[(b)] 当且仅当,$V$中有一向量$\alpha$可以唯一表示为
%     $$\alpha = \sum_{i=1}^s \alpha_i, \quad \alpha_i \in V_i$$
%     \item[(c)] 当且仅当,
%     $$\dim(V_1 + V_2 + \cdots + V_s) = \dim V_1 + \cdots \dim V_s$$ 
%   \end{itemize}
% \end{itemize}
% \vspace{3cm}

% \begin{itemize}
%   \item[3.] 设A是数域K上的一个n阶矩阵,$\lambda_1, \lambda_2, \cdots, \lambda_s$
%   是A的全部不同的特征值,用$V_{\lambda_i}$表示A的属于$\lambda_i$的特征子空间. 证明:
%   A可以对角化的充分必要条件是
%   $$K^n = V_{\lambda_1} \oplus V_{\lambda_2} \oplus \cdots \oplus V_{\lambda_s}$$
% \end{itemize}
% \vspace{1cm}

\begin{itemize}
  \item[1.] 设$V_1, V_2, \cdots, V_s$是域$F$上线性空间V的$s$个真子空间,证明:如果$char F \ne 0$,
  $V_1\cup V_2\cup \cdots \cup V_s \ne V$.
  \vspace{2cm}

  \item[2.] 在数域K上的线性空间$K^{M_n(K)}$中,如果$f$满足,
  对于任意的$A=(\alpha_1, \alpha_2,\cdots, \alpha_n)$,任意的n维列向量$\alpha$, 以及任意$k\in K$, $j\in \{1,2,\cdots, n\}$,
  有 
  $$f(\alpha_1, \cdots, \alpha_{j-1}, \alpha_{j} + \alpha, \alpha_{j+1}, \cdots, \alpha_n)
  = f(\alpha_1, \cdots, \alpha_{j-1}, \alpha_{j}, \alpha_{j+1}, \cdots, \alpha_n) + 
    f(\alpha_1, \cdots, \alpha_{j-1}, \alpha, \alpha_{j+1}, \cdots, \alpha_n)$$
  $$f(\alpha_1, \cdots, \alpha_{j-1}, k\alpha_{j}, \alpha_{j+1}, \cdots, \alpha_n)
  = kf(\alpha_1, \cdots, \alpha_{j-1}, \alpha_{j}, \alpha_{j+1}, \cdots, \alpha_n)
  $$
  那么称$f$是$M_n(K)$上的列线性函数. 同理如果$g(A^T)$是列线性函数,则称$g(A)$是行线性函数。
  记所有的列/行线性函数组成的集合分别记为$V_1$和$V_2$.
  \begin{itemize} 
    \item[(a)] 证明: $V_1, V_2$都是$K^{M_n(K)}$的子空间
    \item[(b)] 分别求$V_1, V_2$的一个基和维数
    \item[(c)] 分别求$V_1 \cap V_2,\,V_1 + V_2$的一个基和维数 
  \end{itemize}
\end{itemize}
\vspace{6cm}

\paragraph{8.3}
\begin{itemize}
    \item 线性空间的同构
    \item 有限维线性空间同构的充要条件
\end{itemize}

\paragraph{例题}
\begin{itemize}
  \item[1.]令
  \begin{equation}
  \nonumber
  H = \left\{
    \begin{bmatrix}
      z_1& z_2\\
     -\bar{z}_2& \bar{z}_1\\
    \end{bmatrix},
    z_1, z_2 \in C
      \right\}
  \end{equation}
  \begin{itemize}
    \item[(a)] H对于矩阵的加法,以及实数和矩阵的乘法构成一个实线性空间
    \item[(b)] 给出H的一个基和维数
    \item[(c)] 证明: $H$与$R^4$同构,并写出$H$到$R^4$的一个同构映射  
  \end{itemize}
\end{itemize}
\vspace{2cm}

\begin{itemize}
  \item[2.]设$A \in M_n(K)$, 令$AM_n(K) = \{AB, B\in M_n(K) \}$.
  \begin{itemize}
    \item[(a)] 证明$AM_n(K)$是数域$K$上线性空间$M_n(K)$的子空间
    \item[(b)] 设A的列向量组$\alpha_1, \cdots, \alpha_n$的一个极大线性无关组为
    $\alpha_{j_1}, \cdots, \alpha_{j_r}$,证明$AM_n(K)$和$M_{r\times n}(K)$同构,并
    写出一个同构映射. 
    \item[(c)] 证明: $\dim [AM_n(K)] = \rank(A)n$.
  \end{itemize}
\end{itemize}
\vspace{4cm}

\paragraph{8.4}
\begin{itemize}
    \item 商空间
    \item $\dim{(V/W)} = \dim V - \dim W$
\end{itemize}

\paragraph{例题}
\begin{itemize}
    \item [1.] 设$U,W$都是域F上线性空间$V$的子空间, 证明$(U+W)/W \cong U/(U\cap W)$
    \vspace{2cm}
    % \item [2.] 设$V$是域F上的一个$n$维线性空间, ($n\ge 3$). $U$是$V$的一个2维子空间,用
    % $\Omega_1$表示V中包含U的所有$n-1$维子空间组成的集合,用$\Omega_2$表示商空间$V/U$的所有
    % $n-3$维子空间组成的集合,令
    % \begin{equation}
    %     \nonumber
    %     \begin{aligned}
    %     \sigma:\quad &\Omega_1 \longrightarrow \Omega_2\\
    %                  & W \longrightarrow W/U.
    %     \end{aligned}
    % \end{equation}
    % 证明: $\sigma$是双射.
    % \vspace{3cm}
\end{itemize}

\paragraph{9.1-4}
\begin{itemize}
    \item 线性映射,线性变换,线性函数
    \item 线性映射的核与像:
    \begin{itemize}
        \item[1.] $\dim{(\Ker A)} + \dim{(\Img A)} = \dim{V}$
        \item[2.] 有限维线性变换, 单等价于满
    \end{itemize}
    \item 线性映射的矩阵表示:
    \begin{itemize}
        \item[1.] $Hom(V,V') \cong M_{s \times n}(F)$
        \item[2.] 相似矩阵 $\Longleftrightarrow$ 线性变换在不同基下的表示 
    \end{itemize}
    \item 线性映射的行列式,秩,迹,特征值,特征向量等
\end{itemize}

\paragraph{例题}
\begin{itemize}
	\item [1.] 设$A \in Hom(V,V)$, 证明对于任意的$k$, 
	$$\rank{A^k} - \rank{A^{k+1}} \ge \rank{A^{k+1}} - \rank{A^{k+2}}$$
	\vspace{3cm}

	\item [2.] 设$f$是$M_n(K)$上的线性函数,且对于任意的$A,B \in M_n(K),
	f(AB)=f(BA)$, 求证$f=c \mathrm{tr}$, 其中$c$是某一常数,
	$\mathrm{tr}$是迹算子.
	\vspace{2cm}

	\item [3.] (Frobenius秩不等式) 设$V,U,W,M$都是域F上的线性空间,
    并且$V,U$都是有限维的,设$A \in Hom(V,U), B \in Hom(U,W), C \in Hom(W,M)$.
    证明,$$\rank(CBA) \ge \rank(CB) + \rank(BA) - \rank(B).$$
    \vspace{2cm}

    \item [4.] (幂零矩阵的矩阵表示)设V是域$F$上的n维线性空间,A是V上的一个线性变换. 如果
    $A^{n-1} \ne 0, A^n = 0$, 那么在V中存在一个基,使得A在此基下的矩阵为
    \begin{equation}
    \nonumber
    \begin{bmatrix}
        0& 1& 0& \cdots &0\\
        0& 0& 1& \cdots &0\\
        \vdots& \vdots& \vdots& \cdots & 1\\
        0& 0& 0& \cdots &0\\
    \end{bmatrix}
    \end{equation}
    \vspace{3cm}

    \item [5.] 设$V$和$V'$分别是域$F$上n维,s维线性空间,A是$V$到$V'$
    的一个线性映射,证明存在$V$的一个基和$V'$的一个基,使得A在这对基下的矩阵为,
    \begin{equation}
    \nonumber
    \begin{bmatrix}
        I_r& 0\\
        0&   0\\
    \end{bmatrix}
    \end{equation}
    其中$r = \rank(A)$.
    \vspace{3cm}

    \item [6.](两两正交的幂等变换的充要条件)
    设$A_i \in M_n(K)$, $i=1,2,\cdots,s$,其中$K$是数域.
    令$A=A_1 + A_2 +\cdots + A_s$.
    \begin{itemize}
        \item[(1) ] 证明: 如果
              $$ \rank(A) = \rank(A_1) + \rank(A_2) \cdots + \rank(A_s),$$
              那么
              $$AM_n(K) = A_1M_n(K) \oplus A_2M_n(K) \oplus \cdots \oplus A_s M_n(K).$$
        \item [(2) ] 证明: $A_1, A_2, \cdots, A_n$是两两正交的幂等矩阵,
              当且仅当$A$是幂等矩阵,并且
              $$ \rank(A) = \rank(A_1) + \rank(A_2) \cdots + \rank(A_s).$$
    \end{itemize}
    \vspace{5cm}

    % \item [7. ] (对角化和数域相关) 设$f(x)=x^n +a_{n-1}x^{n-1} +\cdots +a_1 x +a_0$ 是有理数域Q上的一个不可约
    % 多项式,$n>1$,$\omega$是$f(x)$的一个复根. 把$C$看成是$Q$上的线性空间,令
    % \begin{equation}
    %     \nonumber
    %     \begin{aligned}
    %     \mathbf{B}:\quad &Q[x] \longrightarrow C\\
    %                      &g(x) \longrightarrow g(\omega).
    %     \end{aligned}
    % \end{equation}
    % \begin{itemize}
    %     \item[(1) ] $\mathbf{B}$是不是一线性映射?若是,给出$\Img \mathbf{B}$的一个基和维数;
    %     \item[(2) ] 令$\mathbf{A}(z) = \omega z, \forall z \in \Img \mathbf{B}$, 
    %                 $\mathbf{A}$是不是$\Img \mathbf{B}$上的线性变换?如果是,求$\mathbf{A}$在(1)中基下的矩阵A;
    %     \item[(3) ] $\mathbf{A}$是否可以对角化?
    %     \item[(4) ] 把矩阵$A$看成是复数域上的矩阵,该矩阵可以对角化么?
    % \end{itemize}
    % \vspace{5cm}
\end{itemize}

\chapter{矩阵}
\section{矩阵的运算和特殊矩阵}
\begin{itemize}
\item 相同行列数目的矩阵全体在加法和数乘下构成线性空间
\item 矩阵的乘法(线性映射的复合,有限维线性映射的典型)
\end{itemize}

\subsection*{例题}
\begin{itemize}
	\item[1.] (矩阵的幂次)
	\begin{itemize}
		\item [(a)]
		\begin{equation}
		\nonumber
		A = \begin{bmatrix}
			0& 1& 0& \cdots& 0\\
			0& 0& 1& \cdots& 0\\
			\vdots& \vdots& \vdots& & \vdots\\
			0& 0& 0& \cdots& 1\\
			0& 0& 0& \cdots& 0\\
		\end{bmatrix}_{n\times n}
		\end{equation}
		计算$A^m$, 其中$m$为正整数.
		\vspace{2cm}

		\item [(b)]
		\begin{equation}
		\nonumber
		J = \begin{bmatrix}
			x& 1& 0& \cdots& 0\\
			0& x& 1& \cdots& 0\\
			\vdots& \vdots& \vdots& & \vdots\\
			0& 0& 0& \cdots& 1\\
			0& 0& 0& \cdots& x\\
		\end{bmatrix}_{n\times n}
		\end{equation}
		计算$J^m$, 其中$m$为正整数.
		\vspace{2cm}
	\end{itemize}
	
	\item[2.] 证明:反对称矩阵的秩为偶数
	\begin{solution}
	取行向量组的极大线性无关组,记为$i_1, i_2, \cdots, i_r$行, 考虑这些行列组成的子矩阵,
	行列式不为零, 且反对称,故$r$为偶数.
	\end{solution}
	\vspace{1cm}
\end{itemize}
	

\section{矩阵相乘的秩与行列式}
\begin{itemize}
\item $\rank{AB} \le \min\{\rank{A}, \rank{B}\}$
\item 实数域上,$\rank{A^TA} = \rank{A}$
\item Binet-Cauchy公式
\end{itemize}

\subsection*{例题}
\begin{itemize}
	\item[1.] 举例说明,对于复矩阵$A$, 有可能$\rank{A^TA} \ne \rank A$
	\vspace{2cm}

	\item[2.] 证明,对于实数域上的任意$s\times n$矩阵$A$, 都有$\rank{AA^TA} = \rank A$
	\vspace{1.5cm}

	\item[3.] 设$A,B$分别是数域$K$上的$s\times n, n\times m$矩阵,
	证明:如果$\rank{AB} = \rank B$, 那么对于任意的数域$K$上的$m\times r$的矩阵
	$C$, 都有 
	$$\rank{ABC} = \rank{BC}.$$
	\begin{solution}
	注意到$ABx=0$和$Bx=0$同解, 证明$ABCx=0$和$BCx=0$同解即可.
	\end{solution}
	\vspace{2cm}

	\item[4.] 计算如下矩阵的行列式
	\begin{equation}
	\nonumber
	A = \begin{bmatrix}
		\cos(\alpha_1 - \beta_1)& \cos(\alpha_1 - \beta_2)& \cdots& \cos(\alpha_1-\beta_n)\\
		\cos(\alpha_2 - \beta_1)& \cos(\alpha_2 - \beta_2)& \cdots& \cos(\alpha_2-\beta_n)\\
		\vdots& \vdots& & \vdots\\
		\cos(\alpha_n - \beta_1)& \cos(\alpha_n - \beta_2)& \cdots& \cos(\alpha_n-\beta_n)\\
	\end{bmatrix}
	\end{equation}
	\begin{solution}
	$\cos(\alpha_i - \beta_j) = \cos(\alpha_i)\cos(\beta_j) + \sin(\alpha_i)\sin(\beta_j)$,
	然后使用Binet-Cauchy公式; $n\ge 2$时为0, 其他时候计算可得.
	\end{solution}
	\vspace{2cm}

\end{itemize}

\section{可逆矩阵}
\begin{itemize}
\item $\det A \ne 0$ 是方阵$A$可逆的充分必要条件
\item 可逆矩阵可以表示为若干初等矩阵的乘积
\end{itemize}

\subsection*{例题}
\begin{itemize}
	\item[1.] (不可约对角占有矩阵)求如下矩阵的逆矩阵
	\begin{equation} \nonumber
	A = \begin{bmatrix}
		2& -1& & & & \\
		-1& 2&-1&& & \\
		& -1& 2& -1& &\\
		&& \ddots& \ddots& \ddots\\
		&&& -1& 2& -1\\
		&&&& -1& 2\\
	\end{bmatrix}_{n \times n}
	\end{equation}
	\begin{solution}
		计算伴随矩阵即可;注意到,伴随矩阵对称,而且当$i\ge j$时,
		$A_{ij}= S_{n-i}S_{j-1} = (n-i+1)j$, 其中$S_{k} = k+1$表示$k$阶
		和$A$同类型矩阵的行列式.
	\end{solution}
	\vspace{2cm}

	\item[2.] ($AB$和$BA$的非零谱相同) $A$,$B$分别为 $n\times m$和$m\times n$
	的矩阵
	\begin{itemize}
		\item [(a).] 证明, 若$I_n - AB$可逆,则$I_m - BA$也可逆.
		\begin{solution}
			验证$(I_m - BA)^{-1} = I_m + B(I_n-AB)^{-1}A$
		\end{solution}
		\vspace{2cm}

		\item [(b).] 证明  $\lambda^{m}\det(\lambda I_n - AB) = 
		                   \lambda^{n}\det(\lambda I_m - BA)$
		\begin{solution}
			考虑分块矩阵
			\begin{equation} \nonumber
			\begin{bmatrix}
				\lambda I_n & A\\
				B& I_m\\
			\end{bmatrix}
			\end{equation}
			使用初等行/列变换,计算其行列式
		\end{solution}
		\vspace{2cm}
	\end{itemize}
\end{itemize}

\section{分块矩阵}
\subsection*{例题}
\begin{itemize}
	\item[1.] (矩阵的LU分解)证明:如果$A$的所有顺序主子式不等于零,则存在可逆的下三角矩阵$L$, 使得
	$LA$是上三角矩阵
	\begin{solution}
		考虑Gauss消去,使用初等变换将矩阵化为阶梯型的过程.
		使用数学归纳法, 和分块矩阵的记号说明.
	\end{solution}
	\vspace{2cm}
\end{itemize}

\section{正交矩阵和欧氏空间}
\begin{itemize}
\item 正交矩阵的定义和性质(实数域)
\item 欧氏空间的内积(对称正定的双线性函数)
\item Schmidt正交化
\end{itemize}

\subsection*{例题}
\begin{itemize}
	\item[1.] (矩阵的QR分解)
	证明:可逆矩阵$A$,可以唯一分解为正交矩阵$Q$与主对角元都是正数的上三角矩阵$R$的乘积
	\begin{solution}
		存在性可以通过Householder变换,Givens变换或者Schimidt正交化来说明;
		唯一性,注意到上三角的正交矩阵为主对角元为$\pm 1$的对角矩阵;
	\end{solution}
	\vspace{2cm}

	\item[2.] $A$是$s\times n$的实矩阵, 说明$A^T$的像空间和$A$的核空间正交, 
	即在两空间中各自任意取一个向量,内积均为零
	\vspace{2cm}

	\item[3.] (最小二乘) $A$是$m\times n$的实矩阵, $m > n$, $b \in \mathbb{R}^n$,
	如果存在$x_0$使得, 对于任意的$x \in \mathbb{R}^n$,
	$$|Ax_0-b|^2 \le |Ax-b|^2$$
	则称$x_0$是$Ax=b$的最小二乘解. 证明$x_0$是最小二乘解,当且仅当$x_0$是如下线性方程组的解
	$$A^TAx=A^Tb$$
	\begin{solution}
	转化为无约束二次优化问题,求梯度可得必要性,充分性可以求Hessian,利用函数的凸性;
	或者利用扰动说明充分性, 
	$$|Ax_0-b|^2 \le |A(x_0+ty)-b|^2$$
	即 $t^2(y^TA^TAy) + 2ty^T(A^TAx_0 - A^Tb) \ge 0.$
	\end{solution}
	\vspace{2cm}
\end{itemize}

\section{线性映射}
\begin{itemize}
\item $\dim(\ker A) + \dim(\mathrm{Im} A) = \dim(K^n)$
\end{itemize}

\subsection*{例题}
\begin{itemize}
	\item[1.] 设$S$是一个有限集合, 映射$f:S \rightarrow S$,
	证明: $f$为单射和$f$为满射互相等价.
	\begin{solution}
		$|f(S)| = |S|$
	\end{solution}
	\vspace{1cm}

	\item[2.] 上一题结论在$S$不是有限集合的时候成立么?若成立,请证明;不成立给出反例
	\vspace{1cm}

	\item[3.] 设$V$是一个有限维线性空间, 线性映射$L:V \rightarrow V$,
	证明: $L$为单射和$L$为满射互相等价.
	\begin{solution}
		映射的线性,保证了$L(V)$也是一个线性空间.
		考虑$V$的一组基$\alpha_1,\dots,\alpha_n$, 以及$L\alpha_1, \dots, L\alpha_n$,
		\begin{enumerate}
		\item [1.] 若$L$为单射, 则$L\alpha_i$也有$n$个, 其线性无关. 所以$\dim L(V) = n, L(V) = V$, 所以是满射. 
		\item [2.] 若$L$为满射, $\dim L(V) = n$, 注意到$L\alpha_1, \dots, L\alpha_n$张成$L(V)$,
		$n \le \rank{\{ L\alpha_1,\dots,L\alpha_n\}} \le n$, 
		故$L\alpha_i$线性无关,故$L$为单射
		\end{enumerate}
	\end{solution}
	\vspace{2cm}

\end{itemize}
\chapter{矩阵相抵和相似}

\section{矩阵相抵}
\begin{itemize}
\item 如果矩阵$A$可以通过初等行/列变换为矩阵$B$, 则称$A,B$相抵
\item 矩阵相抵是$M_{m\times n}(K)$上的一个等价关系
\item $M_{m\times n}(K)$中的两矩阵相抵当且秩相等
\end{itemize}

\subsection*{例题}
\begin{itemize}
	\item[1.] 设$A,B,C$分别是数域$K$上$s\times n, p\times m, s\times m$矩阵,
	证明矩阵方程$AX-YB=C$有解的充分必要条件是
	\begin{equation}
	\nonumber
	\mathrm{rank}
	\begin{bmatrix}
		A&0\\
		0&B
	\end{bmatrix}
	=
	\mathrm{rank}
	\begin{bmatrix}
		A&C\\
		0&B
	\end{bmatrix}
	\end{equation}
	\begin{solution}
		必要性将$C=AX-YB$代入即可; 充分性,
		假设$\mathrm{rank}(A)=a, \mathrm{rank}(B)=b$,
		则存在可逆矩阵$P_a, Q_a, P_b, Q_b$, 使得
		\begin{equation*}
			P_a A Q_a = \begin{bmatrix}
				I_a&0\\
				0&0\\
			\end{bmatrix},\quad
			P_b B Q_b = \begin{bmatrix}
				I_b&0\\
				0&0\\
			\end{bmatrix}
		\end{equation*}
		所以
		\begin{equation*}
		\begin{bmatrix}
			P_a&\\
			&P_b\\
		\end{bmatrix}
		\begin{bmatrix}
			A&C\\
			0&B\\
		\end{bmatrix}
		\begin{bmatrix}
			Q_a&\\
			&Q_b\\
		\end{bmatrix}
		=
		\begin{bmatrix}
			I_a&0& C_{11}& C_{12}\\
			0&0& C_{21}& C_{22}\\
			0&0& I_b& 0\\
			0&0& 0& 0\\
		\end{bmatrix}
		\end{equation*}
		其中
		\begin{equation*}
		P_a C Q_b = \begin{bmatrix}
			C_{11}&C_{12}\\
			C_{21}&C_{22}\\
		\end{bmatrix}
		\end{equation*}
		由已知条件, $C_{22}=0$. 所以
		\begin{equation*}
		\begin{bmatrix}
			I& &-C_{11}& \\
			&I &-C_{21}&\\
			& &I&\\
			& &&I\\
		\end{bmatrix}
		\begin{bmatrix}
			P_a&\\
			&P_b\\
		\end{bmatrix}
		\begin{bmatrix}
			A&C\\
			0&B\\
		\end{bmatrix}
		\begin{bmatrix}
			Q_a&\\
			&Q_b\\
		\end{bmatrix}
		\begin{bmatrix}
			I& & &-C_{12}\\
			&I & &\\
			& &I&\\
			& &&I\\
		\end{bmatrix}
		=
		\begin{bmatrix}
			I_a&0&0&0\\
			0&0&0&0\\
			0&0& I_b& 0\\
			0&0& 0& 0\\
		\end{bmatrix}
		\end{equation*}
		注意右上角的分块计算结果为0, 整理即可获得原矩阵方程的解.
	\end{solution}
	\vspace{5cm}
\end{itemize}


\nocite{*} 
\printbibliography

\appendix

\end{document}
