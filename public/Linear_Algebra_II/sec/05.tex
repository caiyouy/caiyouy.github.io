\paragraph{复Jordan标准型}
\begin{itemize}
    \item[(1)] $\lambda$矩阵
    \item[(2)] 不变子空间直和分解
\end{itemize}

\paragraph{$\lambda$-矩阵}
\begin{itemize}
    \item $\lambda$矩阵等价 $\Longleftrightarrow$ 相同的不变因子/行列式因子
    \item $A,B$相似 $\Longleftrightarrow$ $\lambda I-A$和$\lambda I-B$等价
    \item 初等因子: 将不变因子分解为一次因子方幂的乘积; 同级复矩阵相似 $\Longleftrightarrow$ 初等因子相同
\end{itemize}

\paragraph{例题}
\begin{itemize}
    \item [1.] 求
    \begin{equation}
    \nonumber
    J_n(\lambda) = 
    \begin{pmatrix}
    \lambda&1&&&&\\
    &\lambda&1&&&\\
    &&\ddots&\ddots&&\\
    &&&\lambda&1\\
    &&&&\lambda\\
    \end{pmatrix}_{n\times n}
    \end{equation}
    的所有不变因子和初等因子.
    \vspace{3cm}
    \item [2.]
    \begin{itemize}
        \item [(1)] 设
    \begin{equation}
    \nonumber
    A(\lambda) = \begin{pmatrix}
        f_1(\lambda)g_1(\lambda)&0\\
        0&f_2(\lambda)g_2(\lambda)\\
    \end{pmatrix},\quad
    B(\lambda) = \begin{pmatrix}
        f_2(\lambda)g_1(\lambda)&0\\
        0&f_1(\lambda)g_2(\lambda)\\
    \end{pmatrix},
    \end{equation}
    如果多项式$f_1(\lambda),f_2(\lambda)$都与$g_1(\lambda),g_2(\lambda)$
    互素,则$A(\lambda)$和$B(\lambda)$等价.
    \vspace{3cm}
    
    \item [(2)] 对于对角$\lambda$矩阵$D(\lambda)$, 假设对角元素可以分解为一次因式方幂的乘积,
    证明所有这些一次因式的方幂就是$D(\lambda)$的全部初等因子.
    \vspace{3cm}

    \item [(3)]
    求
    \begin{equation}
    \nonumber
    J = 
    \begin{pmatrix}
    J_{n_1}(\lambda_1)&&&\\
    &J_{n_2}(\lambda_2)&&\\
    &&\ddots&\\
    &&&J_{n_s}(\lambda_s)\\
    \end{pmatrix}
    \end{equation}
    的所有初等因子.
    \vspace{3cm}

    \item [(4)]
    求矩阵
    \begin{equation}
    \nonumber
    A = 
    \begin{pmatrix}
    -1 & -2& 6\\
    -1 &  0& 3\\
    -1 & -1& 4\\
    \end{pmatrix}
    \end{equation}
    在复数域上的Jordan标准型.
    \vspace{3cm}
    \end{itemize}
\end{itemize}

\paragraph{不变子空间直和分解}
\begin{itemize}
    \item [1.] 设$A$是域$F$上线性空间$V$上的线性变换,$A$的最小多项式
    $m(\lambda)$在$F[\lambda]$中的标准分解为
    $$m(\lambda) = p_1^{l_1}(\lambda)p_2^{l_2}(\lambda)\cdots p_s^{l_s}(\lambda)$$
    其中$p_i(\lambda)$是域F上的两两不同的首一不可约多项式.
    \begin{itemize}
        \item [(1)] 证明: 
        $$V = \Ker p_1^{l_1}(A) \oplus \Ker p_2^{l_2}(A) \oplus \cdots \oplus \Ker p_s^{l_s}(A)$$
        \vspace{2cm}
        \item [(2)] 设$B_j = p_j(A|W_j)$, 其中$W_j = \Ker p_{j}^{l_j}$.
        证明: $B_j$是$W_j$上的幂零变换,且它的幂零指数为$l_j$.
        \vspace{3cm}
        \item [(3)] 证明: $\Ker p_j^{k_j}(A) = \Ker p_{j}^{l_j}, \forall k_j \ge l_j$.
        \vspace{3cm}
        \item [(4)] 以下进一步假设$p_j(\lambda) = \lambda - \lambda_j$. 证明:
        如果$A$的特征多项式可以分解为
        $$f(\lambda) = (\lambda-\lambda_1)^{r_1}(\lambda - \lambda_2)^{r_2}\cdots
        (\lambda-\lambda_s)^{r_s}$$
        则$W_j$的维数为$r_j$.
        \vspace{3cm}
    \end{itemize}
    \item [2.] 上面已经说明$B_j$是$r_j$维空间$W_j$上幂零指数为$l_j$的幂零变换. 所以我们需要研究幂零变换的结构.
    \begin{itemize}
        \item [(1)] 记$k_j = \dim \Ker B_j$, 说明$W$能分解为$k_j$个$B_j$强循环子空间的直和.
        \vspace{3cm}
        \item [(2)] 证明$W_j$中存在一个基,使得$B_j$在此基下的矩阵为
        $Jordan$型矩阵, 每个$Jordan$块的主对角元都是0,且级数不超过$l_j$,
        Jordan块的总数是$r_j-\rank B_j$, $t$级$Jordan$块的个数$N(t)$为,
        $$N(t) = \rank B_j^{t+1} + \rank B_j^{t-1} - 2 \rank B_j^t.$$
        \vspace{3cm}
    \end{itemize}
\end{itemize}

\paragraph{例题}
\begin{itemize}
  \item[1.] 设$B$是域$F$上n维线性空间$V$上的幂零变换,且它的幂零指数为$l$,证明:
  $$l \le 1 + \rank(B)$$
  \vspace{3cm}
  \item[2.] 设域F上的$n$级矩阵
  $$A = \mathrm{diag}\left\{J_{n_1}(\lambda_1), J_{n_2}(\lambda_2),
  \cdots,J_{n_s}(\lambda_s)\right\}$$
  其中$\lambda_1,\lambda_2,\cdots,\lambda_s$两两不同,
  证明$\dim C(A) = n, C(A) = F[A].$ 
\end{itemize}


