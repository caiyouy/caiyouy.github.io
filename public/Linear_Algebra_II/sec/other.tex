\paragraph{命题7.6.2(矩阵理论苏育才等)} 设n阶方阵$A$的最小多项式为
$$m(\lambda) = (\lambda-\lambda_1)^{k_1}(\lambda-\lambda_2)^{k_2}\cdots(\lambda-\lambda_s)^{k_s},$$
则
$$g(\lambda)=\sum_{i=1}^s m_i(\lambda)
\left(a_{i0} + a_{i1}(\lambda-\lambda_i) 
+a_{i2}(\lambda-\lambda_i)^2 +\cdots a_{ik_i}(\lambda-\lambda_i)^{k_i-1}\right)$$
满足如下Hermite插值条件
$$g^{(j)}(\lambda_i) = f^{(j)}(\lambda_i),\quad i=1,\cdots,s,\,j=0,\cdots,k_i-1,$$
其中
$$m_i(\lambda) = m(\lambda)/(\lambda-\lambda_i)^{k_i},$$
$$a_{ij} = \frac{1}{j!}\left(\frac{f(\lambda)}{m_i(\lambda)} \right)^{(j)}
\bigg|_{\lambda=\lambda_i}.$$

\paragraph{证明}
只需验证插值条件即可. 注意到, 当$j \le k_i-1$时,
$$g^{(j)}(\lambda)\bigg|_{\lambda=\lambda_i} = 
\left[
m_i(\lambda)
\left(a_{i0} + a_{i1}(\lambda-\lambda_i) 
+a_{i2}(\lambda-\lambda_i)^2 +\cdots a_{ik_i}(\lambda-\lambda_i)^{k_i-1}\right)
\right]^{(j)}\bigg|_{\lambda=\lambda_i}
$$
只需验证上式等于$f^{(j)}(\lambda_i)$即可.

对$j$进行数学归纳.当$j=0$时,易见成立.
当$j=1$时,
\begin{equation}
\nonumber
\begin{aligned}
g^{(1)}(\lambda_i) &=
m_i(\lambda_i)^{(1)}a_{i0}+
m_i(\lambda_i)a_{i1}\\
&=
\left(m_i(\lambda)^{(1)}\frac{f(\lambda)}{m_i(\lambda)}+
m_i(\lambda)
\left(
	\frac{f(\lambda)}{m_i(\lambda)}\right)^{(1)}
\right) \Bigg|_{\lambda = \lambda_i}\\
&=f(\lambda_i)^{(1)}.
\end{aligned}
\end{equation}

假设$j<k$时成立, 其中$k\le k_i-1$.当$j=k$时
\begin{equation}
\nonumber
\begin{aligned}
g^{(j)}(\lambda_i) &=
\sum_{p=0}^{j}C_{j}^p
m_i(\lambda_i)^{(p)}(j-p)!a_{i(j-p)}\\
&=
\left(
\sum_{p=0}^{j}C_{j}^p
m_i(\lambda)^{(p)}
\left(\frac{f(\lambda)}{m_i(\lambda)} \right)^{(j-p)}
\right) \Bigg|_{\lambda = \lambda_i}\\
&=f(\lambda_i)^{(j)}.
\end{aligned}
\end{equation}
证毕.

\paragraph{考试试题} 
\begin{enumerate}
	\item[1.] 设$V_1$和$V_2$是数域$K$上的线性空间,维数分别为$m,n$,
	试通过$V_1$和$V_2$构造$K$上的线性空间$W_1$和$W_2$,使得他们的维数分别为$m+n$和$mn$.
	\item[2.] 设$A$为实方阵,
		\begin{enumerate}
			\item[(1)] 证明存在正交矩阵$T$使得$T^{-1}AT$为分块上三角矩阵,且其主对角线上的块
			$A_{ii}$为1阶或2阶实方阵. 
			\item[(2)] 设$A$的复特征值为$\lambda_1, \lambda_2, \cdots, \lambda_n$, 证明
			$$\mathrm{tr}(AA^T) \ge |\lambda_1|^2 + |\lambda_2|^2 + \cdots + |\lambda_m|^2,$$
			且等号成立当且仅当$AA^T = A^TA$.
		\end{enumerate}
\end{enumerate}

\paragraph{参考做法}
\begin{enumerate}
\item[1.] $W_1 = V_1 \oplus V_2$(外直和), $W_2 = V_1 \otimes V_2$(张量积).  
\item[2.]
	\begin{enumerate}
			\item[(1)] 对维数归纳. 一阶或二阶显然成立, $n$阶时如果有特征值, 则可以将特征向量扩充为标准正交基,
			将问题归纳为$n-1$阶. 如果没有特征值, 可以找到二维不变子空间,同样的方式,将问题归纳为$n-2$ 
			阶.
			\item[(2)] 使用复$Schur$分解
			\begin{equation}
			\nonumber
			\begin{aligned}
				\mathrm{tr}(AA^T) &= \mathrm{tr} (AA^H) \\
				                  &= \mathrm{tr}(URU^HUR^H U^H)\\
				                  &= \mathrm{tr}(RR^H)\\
								  &= \sum_{i\le j } |r_{ij}|^2 \\
								  &\ge \sum_{i} |r_{ii}|^2  = \sum_i |\lambda_i|^2\\
			\end{aligned}
			\end{equation} 
			等号成立当且仅当$\sum_{i<j} |r_{ij}|^2 = 0$, 即$A$可酉对角化, 等价于$AA^H = A^HA$, 即正规矩阵, 证毕.
		\end{enumerate}
\end{enumerate}