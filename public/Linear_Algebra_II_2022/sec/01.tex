\paragraph{7.1}
\begin{itemize}
  \item 一元多项式的概念和运算
  \item 环的基本概念; 关于加法构成交换群,乘法具有结合律,加法乘法具有左右分配律
  \item 一元多项式环$K[x]$的通用性质
\end{itemize}

\paragraph{例题}
\begin{itemize}
  \item[1.] $R$是有单位元$1$($\ne 0$)的环,若对于$a\in R$, $\exists \,b \in R$, s.t.
$ab = ba = 1$, 则称$b$为$a$的逆. 证明$a$的逆是唯一的, 且$a$不为零因子.
  \vspace{1cm}

  \item[2.] 设$R$是一个环,证明: $0a = a0 = 0, \forall a\in R$;
  $\forall a, b \in R, a(-b)=-ab$.
  \vspace{1cm}
  
  \item[3.] 设$B$是数域$K$上的$n$级幂零矩阵,幂零指数为$l$, 令$A=aI+bB$, 且$a,b \ne 0$.
  说明$A$可逆,并计算$A^{-1}$.
  \vspace{2cm}

  \item[4.] 设$A\in M_n(C)$, 设
  $$|\lambda I - A| = (\lambda - \lambda_1)^{l_1}(\lambda - \lambda_2)^{l_2}\cdots(\lambda - \lambda_s)^{l_s},$$
  其中$\lambda_1,\cdots,\lambda_s$是两两不同的复数,$l_1 + l_2 + \cdots + l_s = n$.
  证明,对于$k \in C$, $k \ne 0$, 矩阵$kA$的特征多项式为
  $$|\lambda I - kA| = (\lambda - k\lambda_1)^{l_1}(\lambda - k\lambda_2)^{l_2}\cdots(\lambda - k\lambda_s)^{l_s},$$
  $A^3$的特征多项式为
  $$|\lambda I - A^3| = (\lambda - \lambda_1^3)^{l_1}(\lambda - \lambda_2^3)^{l_2}\cdots(\lambda - \lambda_s^3)^{l_s}.$$
  \vspace{3cm}
\end{itemize}

\vspace{0.5cm}
\paragraph{7.2}
\begin{itemize}
  \item 整除关系
  \item 带余除法; 主要思路: 对被除式的次数用数学归纳法
\end{itemize}
\paragraph{例题}
\begin{itemize}
  \item[1.] 设$d,n\in N^*$, 则$K[x]$中,$x^d-1|x^n-1 \Longleftrightarrow d|n$.
  \vspace{1.5cm}

  \item[2.] 将$f(x)=5x^3-3x+4$表示为$x+2$的幂和/多项式.
  \vspace{1.5cm}

  \item[3.] 设$m \in N^*, a\in K^*$, 证明: 在$K[x]$中, $x-a|x^m-a^m$, 并求商式.
  \vspace{1.5cm}
\end{itemize}

\paragraph{多项式理论的应用: $\lambda$-矩阵}
即矩阵的每个元素都是多项式环$K[\lambda]$中元素;矩阵乘法,加法以及行列式等概念和数字矩阵类似

\paragraph{$\lambda$-矩阵的初等变换}
\begin{itemize}
  \item [(a)] 矩阵的两行/列互换位置
  \item [(b)] 矩阵的某一行/列乘以非零常数c
  \item [(c)] 矩阵的某一行/列加上另一行/列的$p(\lambda)$倍,其中$p(\lambda) \in K[\lambda]$.
\end{itemize}
如果$A(\lambda)$可以经过一系列行和列的初等变换化为$B(\lambda)$,
则称$A(\lambda)$和$B(\lambda)$相抵.

\paragraph{例题}
\begin{itemize}
  \item[1.] 证明: 设$A(\lambda)$的左上角元素$a_{11}(\lambda) \ne 0$, 并且$A(\lambda)$中
至少有一个元素不能被它除尽,那么一定可以找到和$A(\lambda)$相抵的$B(\lambda)$,
它的左上角元素也不为零,但是次数小于$a_{11}(\lambda)$的次数
  \vspace{3cm}
  
  \item[2.] 证明:任意一个非零的$s\times n$的$\lambda$-矩阵$A(\lambda)$都相抵于如下形式的矩阵(被称为标准形式)
\begin{equation}
\nonumber
\begin{pmatrix}
  d_1(\lambda)&&&&&&\\
  &d_2(\lambda)&&&&&\\
  &&\ddots&&&&\\
  &&&d_r(\lambda)&&&\\
  &&&&0&&\\
  &&&&&\ddots&\\
  &&&&&&0\\
\end{pmatrix}
\end{equation}
其中$r\ge 1$, $d_i(\lambda)$是首一的多项式(被称为不变因子),且
$$d_i(\lambda)\,|\,d_{i+1}(\lambda)\quad i=1,2,\cdots, r-1.$$
\vspace{4cm}

\item[3.]用初等变换化$\lambda$-矩阵
\begin{equation}
\nonumber
A(\lambda) =
\begin{bmatrix}
1-\lambda& 2\lambda-1& \lambda\\
\lambda&   \lambda^2&  -\lambda\\
1+\lambda^2& \lambda^3+\lambda-1& -\lambda^2\\
\end{bmatrix}
\end{equation}
为标准型.
\vspace{3cm}
\end{itemize}

\paragraph{$\lambda$-矩阵标准形式的唯一性}
下面来借助行列式因子的概念来说明$\lambda$-矩阵标准形式的唯一性.
\begin{itemize}
  \item[1.] ($\lambda$-矩阵的秩)如果$A(\lambda)$中有一个$r\ge 1$级子式不为零,而所有的$r+1$级子式(如果有的话)全为零,
则称$A(\lambda)$的秩为$r$. 零矩阵的秩记为0.
  \item[2.] ($\lambda$-矩阵的行列式因子) 设$A(\lambda)$的秩为r,对于$1 \le k \le r$, $A(\lambda)$中全部$k$级子式的首一最大公因式
$D_{k}(\lambda)$称为$A(\lambda)$的$k$阶行列式因子.
\end{itemize}

\paragraph{例题}
\begin{itemize}
  \item[1.] 等价的$\lambda$-矩阵具有相同的秩和相同的各阶行列式因子.
  \vspace{2cm}
  \item[2.] 证明$\lambda$-矩阵的不变因子和行列式因子有如下关系
  $$
  d_1(\lambda) = D_1(\lambda),\quad
  d_2(\lambda) = \frac{D_2(\lambda)}{D_1(\lambda)},\quad
  \dots, \quad d_r = \frac{D_r(\lambda)}{D_{r-1}(\lambda)}
  $$ 
  \vspace{3cm}
  \item[3.] 说明$\lambda$-矩阵的标准形式是唯一的.
  \vspace{2cm} 
\end{itemize}

